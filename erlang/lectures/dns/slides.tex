
\usepackage{fancyhdr}
\usepackage{lastpage}
\usepackage[utf8]{inputenc}

% Minted for syntax highliting
\usepackage{minted}
\usemintedstyle{tango}

% Headers/footers styling
\pagestyle{fancy}
\fancyhf{}
\renewcommand{\headrulewidth}{0pt}

% Footer
\lfoot{ID1019}
\cfoot{KTH}
\rfoot{\thepage \hspace{1pt} / \pageref{LastPage}}

%\newcommand{\defaultpagestyle}{\thispagestyle{plain}}
\newcommand{\defaultpagestyle}{\thispagestyle{fancy}}

 
 
\title[ID1019 DNS]{A DNS Resolver}
 

\author{Johan Montelius}
\institute{KTH}
\date{\semester}

\begin{document}

\begin{frame}
\titlepage
\end{frame}

\begin{frame}{Domain Name System}

\end{frame}

\begin{frame}[fragile]{RFC 1035}

\begin{verbatim}
             Local Host                        |  Foreign
                                               |
+---------+               +----------+         |  +--------+
|         | user queries  |          |queries  |  |        |
|  User   |-------------->|          |---------|->|Foreign |
| Program |               | Resolver |         |  |  Name  |
|         |<--------------|          |<--------|--| Server |
|         | user responses|          |responses|  |        |
+---------+               +----------+         |  +--------+
                            |     A            |
            cache additions |     | references |
                            V     |            |
                          +----------+         |
                          |  cache   |         |
                          +----------+         |
\end{verbatim}

\end{frame}


\begin{frame}{the resolver}

\begin{itemize}
\item client: sends request to resolver
\item resolver: receives requests, queries servers/resolvers and caches responses 
\item server: responsible for sub-domain
\end{itemize}

\vspace{10pt}\pause  

{\em The first resolver is most probably running on your laptop.}
\end{frame}


\begin{frame}[fragile]{let's build a DNS resolver}
 
\pause

\begin{verbatim}
-module(dns).
-export([start/2]).
\end{verbatim}

\begin{verbatim}
start(Port, Server) ->
    spawn(fun() ->init(Port, Server) end),
\end{verbatim}

\vspace{10pt}\pause
{\em The server is the DNS server to which queries are routed.}

\pause

\begin{verbatim}
init(Port, Server) ->
   case gen_udp:open(Port, [{active, true}, binary]) of
      {ok, Socket} ->
           dns(Socket, Server);
      {error, Error} ->
           io:format("dns: error opening server socket - ~w~n", [Error])
           ok
   end.
\end{verbatim}


\end{frame}

\begin{frame}[fragile]{the server loop}

\begin{verbatim}
dns(Socket, Server) ->
    receive 
        {udp, Socket, IP, Port, Packet} ->
            handler:handle(Packet, IP, Port, Socket, Server),
            dns(Socket, Server);
        stop ->
            io:format("by by~n", []),
            ok;
        Strange ->
            io:format("strange message ~w~n", [Strange]),
            dns(Socket, Server)
    end.
\end{verbatim}

\end{frame}


\begin{frame}[fragile]{handle a request}

\pause
\begin{verbatim}
-module(handler).

-export([handle/5]).

handle(Packet, IP, Port, _Socket, _Server) ->    
    io:format("received request from ~w/~w: ~w~n", [IP, Port, Packet]).
\end{verbatim}

\vspace{20pt}\pause{\em Let's try.}

\end{frame}


\begin{frame}[fragile]{forward it}

\pause
Let's forward the request to the DNS server above us.

\pause

\begin{verbatim}
handle(Packet, IP, Port, Socket, Server) ->    
    case forward(Packet, Server) of
        {ok, Reply} ->
            gen_udp:send(Socket, IP, Port, Reply);
        {error, Error} ->
            io:format("error in forward ~w~n", [Error])
    end.
\end{verbatim}

\end{frame}

\begin{frame}[fragile]{forward a request}

\begin{verbatim}
forward(Request, Server) ->
    case gen_udp:open(0, [{active, true}, binary]) of
        {ok, Client} ->
            gen_udp:send(Client, Server, ?Port, Request),
            Result = receive 
                         {udp, Client, _IP, _Port, Reply} ->
                             {ok, Reply}
                     after ?Timeout ->
                             {error, timeout}
                     end,
            gen_udp:close(Client),
            Result;
        Error ->
            {error, Error}
    end.
\end{verbatim}

\end{frame}


\begin{frame}{deep inside}


\vspace{10pt}\pause
{\em Deep inside the resolver we hard code that we should use UDP packets.}

\end{frame}



\begin{frame}[fragile]{abstractions}

Hide the details, let the server module decide how things are done.

\vspace{10pt}\pause
 
\begin{verbatim}
        {udp, Socket, IP, Port, Packet} ->
\end{verbatim}\pause
\begin{verbatim}
            Reply = fun(Rep) ->  reply(Rep, Socket, IP, Port) end,
\end{verbatim}\pause
\begin{verbatim}
            Forward =  fun(Req) -> forward(Req, Server) end,
\end{verbatim}\pause
\begin{verbatim}
            handler:handle(Packet, Reply, Forward),
\end{verbatim}\pause
\begin{verbatim}
            dns(Socket, Server);
\end{verbatim}

\end{frame}

\begin{frame}[fragile]{an alternative}

The handler process:
\pause
\begin{verbatim}
handle(Packet, Reply, Forward) ->    
    case Forward(Packet) of
        {ok, Msg} ->
            Reply(Msg);
        {error, Error} ->
            io:format("error in forward ~w~n", [Error])
    end.
\end{verbatim}

\vspace{10pt}\pause
{\em Separation of knowledge - a need to know basis.}

\end{frame}


\begin{frame}{increase throughput}

How do we increase throughput?

\end{frame}


\begin{frame}[fragile]{the server loop}

\begin{verbatim}
dns(Socket, Server) ->
    receive 
        {udp, Socket, IP, Port, Packet} ->
            Reply = fun(Rep) ->  reply(Rep, Socket, IP, Port) end,
            Forward =  fun(Req) -> forward(Req, Server) end,
            handler:start(Packet, Reply, Forward),
            dns(Socket, Server);
        stop ->
            io:format("by by~n", []),
            ok;
        Error ->
            io:format("strange message ~w~n", [Error]),
            dns(Socket, Server)
    end.
\end{verbatim}

\end{frame}


\begin{frame}[fragile]{handle a request}

\pause

\begin{verbatim}
-module(handler).
-export([start/3]).
\end{verbatim}

\pause
\begin{verbatim}
start(Packet, Reply, Forward) ->
   spawn(fun() -> handle(Packet, Reply, Forward) end).
\end{verbatim}

\pause
{\em Let's try.}
\end{frame}

\begin{frame}{encapsulation}

Encapsulation of the handling of a request in a process gives us something more than throughput - what?

\end{frame}


\begin{frame}[fragile]{let's decode the message}


  \begin{bytefield}{32}
    \bitheader{0,7-8,15-16,23-24,31} \\

    \begin{rightwordgroup}{transport \\ header}
      \bitbox{16}{identifier} &  \bitbox{16}{flags}\\
      \bitbox{16}{\# query blocks} &  \bitbox{16}{\# answer blocks }\\
      \bitbox{16}{\# authority blocks} &  \bitbox{16}{\# additional blocks }
    \end{rightwordgroup}\\
    \begin{rightwordgroup}{data \\ fields}
      \wordbox[lrt]{1}{%
        \parbox{0.6\width}{\centering \vspace{10pt} query, answer, authority \\ and additional blocks}} \\
      \skippedwords \\
      \wordbox[lrb]{1}{} 
    \end{rightwordgroup}

  \end{bytefield}

\vspace{10pt}\pause
{\em Query and response messages of the same format.}

\end{frame}

 
\begin{frame}[fragile]{message flags}

\begin{itemize}
\item QR: query or reply
\item Op-code: the operation 
\item AA: authoritative answer (if the server is responsible for the domain)
\item TC: message truncated, more to follow
\item RD: recursion desired by client 
\item RA: recursion available by server
\item Resp-code: ok or error message in response
\end{itemize}

\vspace{10pt}\pause

  \begin{bytefield}[bitwidth=2em]{16}
    \bitheader{0,1,4-5,6,7,8-9,11-12,15} \\
    \bitbox{1}{QR} & \bitbox{4}{Op-code} & \bitbox{1}{AA} & \bitbox{1}{TC} & \bitbox{1}{RD} & \bitbox{1}{RA} &  \bitbox{3}{-} &   \bitbox{4}{Resp-code} 
  \end{bytefield}


\vspace{10pt}\pause
{\em This is getting complicated.}

\end{frame}


\begin{frame}[fragile]{the bit syntax}

\begin{verbatim}
decode(<<Id:16, Flags:2/binary, 
         QDC:16, ANC:16, 
         NSC:16, ARC:16, 
         Body/binary>>=Raw) ->
\end{verbatim}

\pause
\begin{verbatim}
    <<QR:1, Op:4, AA:1, TC:1, RD:1, RA:1, _:3, Resp:4>> = Flags,
\end{verbatim}
\pause
\begin{verbatim}
    Decoded = decode_body(QDC, ANC, NSC, ARC, Body, Raw),
\end{verbatim}
\pause
\begin{verbatim}
    {Id, QR, Op, AA, TC, RD, RA, RCode, Decoded}.
\end{verbatim}


{\em Why passing the raw message to the decoding of the body?}


\end{frame}

\begin{frame}[fragile]{decode the body}

The body consists of a number of: query, response, authoritative (server node) and additional sections.

\vspace{10pt}\pause

The answer, authoritative and additional sections follow the same pattern,
the query is slightly different.

\begin{verbatim}
decode_body(QDC, ANC, NSC, ARC, Body0, Raw) ->
    {Query, Body1} = decode_query(QDC, Body0, Raw),
    {Answer, Body2} = decode_answer(ANC, Body1, Raw),
    {Authority, Body3} = decode_answer(NSC, Body2, Raw),
    {Additional, _} = decode_answer(ARC, Body3, Raw),
    {Query, Answer, Authority, Additional}.
\end{verbatim}

\vspace{10pt}\pause

{\em Note the nestling of the reminder of the body.}

\end{frame}

\begin{frame}{decode a query}

A query consists of a sequence of queries (we know from the header how many).

\begin{grammar}
<query> ::= <name> <query type> <query class>

<name> ::= <empty> | <label> <name>

<empty> ::=  0 

<label> ::=  <byte representing length n> <byte sequence of length n> \\

<query type> ::= {\em 16 bits}

<query class> ::= {\em 16 bits}

\end{grammar}

{\em The length is maximum 63 i.e. the top two bits are set to zero.}

\end{frame}

\begin{frame}[fragile]{decode a query}

\begin{verbatim}
decode_query(0, Body, _) ->
    {[], Body};
decode_query(N, Queries, Raw) ->
    {Name, <<QType:16, QClass:16, Next/binary>>} = 
          decode_name(Queries, Raw),
    {Decoded, Body} = decode_query(N-1, Next, Raw),
    {[{Name, QType, QClass} | Decoded], Body}.
\end{verbatim}

\end{frame}


\begin{frame}[fragile]{decode a name}

\begin{verbatim}
decode_name(Label, Raw) ->
    decode_name(Label, [], Raw).
\end{verbatim}
\vspace{20pt}\pause

\begin{verbatim}
decode_name(<<0:1, 0:1, _:6, _/binary>>=Label, Names, Raw) -> 
    decode_label(Label, Names, Raw).
\end{verbatim}
\end{frame}


\begin{frame}[fragile]{decode a label}

\begin{verbatim}
decode_label(<<0:8, Rest/binary>>, Names, _) ->
    {lists:reverse(Names), Rest};
decode_label(<<N:8, Rest/binary>>, Names, Raw) ->
    decode_label(N, Rest, [], Names, Raw).
\end{verbatim}

\vspace{20pt}\pause

\begin{verbatim}
decode_label(0, Rest, Sofar, Names, Raw) ->
    decode_name(Rest, [lists:reverse(Sofar)|Names], Raw);

decode_label(N, <<Char:8, Rest/binary>>, Sofar, Names, Raw) ->
    decode_label(N-1, Rest, [Char|Sofar], Names, Raw).
\end{verbatim}

\end{frame}

\begin{frame}[fragile]{query example}

  Erlang binary:
\begin{verbatim}
  <<4,12, 1, 0,
    0, 1, 0, 0,
    0, 0, 0, 0,
    3,119,119,119,3,107,116,104,2,115,101,0,
    0,1,0,1>>
\end{verbatim}

\vspace{20pt} \pause
  Decoded query:
\begin{verbatim}
  {1036,0,0,0,0,1,0,0,{[{["www","kth","se"],1,1}],[],[],[]}}
\end{verbatim}  

\end{frame}

\begin{frame}{encoding names by offset}

The names in answers may use a more compact form of encoding.  

\vspace{20pt}\pause

Assume we have encoded {\tt www.kth.se} and need to encode {\tt
  mail.kth.se} - then we can reuse the coding of {\tt kth.se}.

\vspace{20pt} \pause

\begin{grammar}
<label> ::=  <byte representing length n> <byte sequence of length n> | \\
             <two bytes representing an offset \underline{ from the start of the message}>
\end{grammar}


\vspace{20pt} \pause


{\em The length version will always have the top two bits set to {\tt 00} and the offset version will have them set to {\tt 11}.}

\end{frame}

\begin{frame}[fragile]{offset encoding}

\begin{verbatim}
decode_name(<<0:1, 0:1, _:6, _/binary>>=Label, Names, Raw) -> 
    %% regular name encoding
    decode_label(Label, Names, Raw);
      
decode_name(<<1:1, 1:1, N:14, Rest/binary>>, Names, Raw) ->
    %% offset encoding
    Offset = 8*N,    
    <<_:Offset, Section/binary>> = Raw,
    {Name, _} = decode_label(Section, Names, Raw),
    {Name, Rest}.
\end{verbatim}

\end{frame}

\begin{frame}[fragile]{decode an answer}

All answer sections have the same basic structure:

\begin{grammar}
<answer> ::= <name> <type> <class> <ttl> <length> <resource record>
\end{grammar}

\begin{itemize}
\item type 16-bits: A-type, NS-, CNAME-, MX- etc
\item class 16-bits: Internet, ...
\item TTL 32-bits: time in seconds (typical some hours)
\item length 16-bits: the length of the record in bytes
\end{itemize}

\vspace{10pt}\pause
{\em The resource record is coded depending on the type of resource.}

\end{frame}


\begin{frame}{let's try}

Same proxy that forwards the requests but now with an effort to decode the messages.

\end{frame}

\begin{frame}{things to do}

We're wasting UDP ports - a new port is used for each query, could we do better?

\vspace{20pt}\pause

We of course want to cache the replies so that we can generate a respons directly.



\end{frame}




\end{document}
