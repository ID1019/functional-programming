
\usepackage{fancyhdr}
\usepackage{lastpage}
\usepackage[utf8]{inputenc}

% Minted for syntax highliting
\usepackage{minted}
\usemintedstyle{tango}

% Headers/footers styling
\pagestyle{fancy}
\fancyhf{}
\renewcommand{\headrulewidth}{0pt}

% Footer
\lfoot{ID1019}
\cfoot{KTH}
\rfoot{\thepage \hspace{1pt} / \pageref{LastPage}}

%\newcommand{\defaultpagestyle}{\thispagestyle{plain}}
\newcommand{\defaultpagestyle}{\thispagestyle{fancy}}


\title[ID1019 Processes]{Processes}


\author{Johan Montelius}
\institute{KTH}
\date{\semester}

\begin{document}

\begin{frame}
\titlepage
\end{frame}

\begin{frame}[fragile]{processes}

\begin{columns}
   \begin{column}{.5\linewidth}
     \begin{block}{create and send}
       \begin{verbatim}
start() ->
   Pid = spawn(fun() -> 
                server(10) 
               end),
   Pid ! {add, 5},
   Pid ! stop.
       \end{verbatim}
       \vfill
     \end{block}
   \end{column}
   \pause
   \begin{column}{.5\linewidth}
     \begin{block}{receive}
       \begin{verbatim}
server(S) ->
   receive
    {add, X} ->
       server(S+X);
    stop ->
       ok
   end
       \end{verbatim}
     \end{block}
   \end{column}
  \end{columns}
\end{frame}

\begin{frame}[fragile]{processes}
\begin{verbatim}
first() ->
    receive
      tic -> io:format("tic~n", []), second()
    end.
\end{verbatim}
\pause
\begin{verbatim}
second() ->
    receive
      tac -> io:format("tac~n", []), last();
      toe -> io:format("toe~n", []), last()
    end.
\end{verbatim}
\pause
\begin{verbatim}
last() ->
    receive 
      X -> io:format("~w~n", [X])
    end.
\end{verbatim}
\pause
\begin{verbatim}
 P = spawn(fun()->first() end), P ! toe, P ! tac, P ! tic.
\end{verbatim}
\end{frame}

\begin{frame}[fragile]{why}

Erlang checks the first message in the queue against all clauses before
moving to the second element. 

Messages that do not match a clause are left in the queue.

\pause\vspace{10pt}
Why not the other way around?

\pause\vspace{20pt}
\begin{verbatim}
priority(State) ->
    receive 
       {alarm, the, house, is, on, fire} -> 
           call_112(State);
       {harmless, X} -> 
           take_it_easy(X, State)
    end.
\end{verbatim}

\end{frame}

\begin{frame}[fragile]{what now}

\begin{block}{dead process}
  \begin{verbatim}
start() ->
   Pid = spawn(fun() -> ok end),
   Pid ! {add, 5},
   Pid ! stop.
       \end{verbatim}
\end{block}
\pause
\begin{block}{no match}
   \begin{verbatim}
start() ->
   Pid = spawn(fun() -> server(10) end),
   Pid ! {gurka, 5},
   Pid ! stop.
   \end{verbatim}
\end{block}

\end{frame}

\begin{frame}[fragile]{example account}

\begin{columns}
   \begin{column}{.4\linewidth}
     \begin{verbatim}
acc(Saldo) ->
  recieve
     {dep, M} ->
        acc(Saldo + M);
     {wdr, M} ->
        acc(Saldo - M)
  end.
     \end{verbatim}
   \end{column}
   \begin{column}{.6\linewidth}
     \begin{verbatim}
doit() ->
  Acc = spawn(fun()->acc(0) end),
  Acc ! {dep, 20},
  Acc ! {wdr, 10}.
  Acc.
     \end{verbatim}
     \vfill
   \end{column}
\end{columns}


\end{frame}

\begin{frame}[fragile]{example account}

\begin{columns}
   \begin{column}{.45\linewidth}
\begin{verbatim}
acc(Saldo) ->
  recieve
    {dep, M} ->
       acc(Saldo + M);
    {wdr, M} ->
       acc(Saldo - M);
    {req, Pid} ->
       Pid ! {reply, Saldo},
       acc(Saldo)
  end.
\end{verbatim}
\end{column}
   \begin{column}{.45\linewidth}
\begin{verbatim}
check(Acc) ->
  Acc ! {req, self()},
  receive
    {reply, Saldo} ->
        Saldo
  end.
\end{verbatim}
\vfill
\end{column}
\end{columns}
\end{frame}


\begin{frame}[fragile]{request reply}

\begin{verbatim}
client(Server) ->
   Server ! {query, 42, self()},
   receive 
     {reply, X} ->
        continue(X, Server)
   end.
\end{verbatim}
\pause\vspace{10pt}
\begin{verbatim}
server() ->
   receive 
    {query, N, Client} ->
         Client ! {reply, N+1},
         server()
   end.
\end{verbatim}

\end{frame}


\begin{frame}[fragile]{unique ref}

\begin{verbatim}
client(Server) ->
   Ref = make_ref(),
   Server ! {query, 42, Ref, self()},
   receive 
     {reply, Ref, X} ->
        continue(X, Server)
   end.
\end{verbatim}
\pause\vspace{10pt}
\begin{verbatim}
server() ->
   receive 
    {query, N, Ref, Client} ->
         Client ! {reply, Ref, N+1},
         server()
   end.
\end{verbatim}

\end{frame}

\begin{frame}[fragile]{what if...}

What if we send a request and never get a reply?

\pause\vspace{20pt}
\begin{verbatim}
client(Server) ->
   Server ! {query, 42, self()},
   receive 
     {reply, X} ->
        continue(X, Server)
   after 
     4000 -> something(Server)
   end.
\end{verbatim}

\pause\vspace{20pt}
{\em Why is this dangerous?}
\end{frame}



\begin{frame}[fragile]{process pattern}

\begin{verbatim}
process(This, That) ->
   receive
      {foo, F} -> 
           Upd = zot(This, F),
           process(Upd, That);
      {bar, B, A} ->
           Ipd = zat(That, B, A),
           process(This, Ipd);
      Strange ->
           underliga_saker_handa(Strange)
   end.
\end{verbatim}

\pause\vspace{20pt}
{\em Tail recursion!}

\end{frame}

\begin{frame}[fragile]{spawn}

Assume we have mod:foo/1 defined, we can write:

\vspace{20pt}
\begin{verbatim}
 Pid = spawn(fun() -> mod:foo(42) end)
\end{verbatim}

\pause
 ... or:
\vspace{10pt}

\begin{verbatim}
 Pid = spawn(mod, foo, [42])
\end{verbatim}

\pause
 ... why not:
\vspace{10pt}
\begin{verbatim}
 Pid = spawn(mod:foo(42))
\end{verbatim}

\end{frame}

\begin{frame}[fragile]{linking}

What if ...

\begin{verbatim}
    :
  Pid = spawn(fun() -> start(42) end),
  X = 5 div 0,
    :
\end{verbatim}

\pause\vspace{10pt}

\begin{verbatim}
    :
  %% if I crash, you die
  Pid = spawn_link(fun() -> start(42) end),
    :
\end{verbatim}

\pause\vspace{10pt}

{\em ... but then, if you crash I die ... hmmm}
\end{frame}

\begin{frame}[fragile]{exit signals}

An exception, division by zero:

\begin{tikzpicture}[>=stealth', node distance=8cm, semithick, auto]
    \node[state]                     (mother)   []                  {mother};
    \node[state]                     (child)  [right of=mother]     {child};

    \path[-, dashed]  (mother)   edge [bend left] node {link}                            (child);
    \pause
    \path[<-]         (mother)   edge             node {\{'EXIT', <0.55.0>, \{badarith, ...\}\}}       (child);
\end{tikzpicture}

\pause
\begin{verbatim}
> 
=ERROR REPORT==== 26-Nov-2013::23:23:05 ===
Error in process <0.55.0> with exit value: 
           {badarith,[{erlang,'/',[1,0],[]}]}
\end{verbatim}
\end{frame}

\begin{frame}[fragile]{exit signals}

A normal termination:

\begin{tikzpicture}[>=stealth', node distance=8cm, semithick, auto]
    \node[state]                     (mother)   []                  {mother};
    \node[state]                     (child)  [right of=mother]     {child};

    \path[-, dashed]  (mother)   edge [bend left] node {link}                            (child);
    \pause
    \path[<-]         (mother)   edge             node {\{'EXIT', <0.55.0>, normal\}}       (child);
\end{tikzpicture}

\pause

{\em the child dies but the mother survives}

\end{frame}

\begin{frame}[fragile]{exit signals}

Propagating exit signals:

\begin{tikzpicture}[>=stealth', node distance=4cm, semithick, auto]
    \node[state]                     (mother)   []                  {mother};
    \node[state]                     (child1)   [above right of=mother]     {child1};
    \node[state]                     (child2)   [below right of=mother]     {child2};

    \path[-, dashed]  (mother)   edge [bend left]  node {link}                            (child1);
    \path[-, dashed]  (mother)   edge [bend right] node {link}                            (child2);
    \pause
    \path[->]         (child1)   edge  node {\{'EXIT', <0.55.0>, \{badarith, ..\}\}}  (mother);
    \pause
    \path[->]         (mother)   edge  node {\{'EXIT', <0.45.0>, \{badarith, ..\}\}}  (child2);

\end{tikzpicture}

\pause

{\em only exceptions i.e. non-normal termination}

\end{frame}


\begin{frame}[fragile]{trap exit signals}

Do we have to die?

\pause\vspace{10pt}

\begin{verbatim}
   process_flag(trap_exit, true),
   Pid = spawn_link(fun() -> start(42) end),
     :
\end{verbatim}

\pause\vspace{10pt}

\begin{verbatim}
     :
   receive
      {regular, Msg} -> 
            do_this(Msg);
      {'EXIT', From, Reason} ->
            %% Someone crached
            what_should_we_do(Reason)
   end.
\end{verbatim}


\end{frame}

\begin{frame}{link and trap}

\begin{itemize}
\item Link processes that need each other - recommended.
\pause
\item Don't use trap\_exit if you don't know what you're doing.
\end{itemize}

\end{frame}


\begin{frame}[fragile]{monitor}

There is an alternative to link/trap, called {\em monitoring}.

\pause\vspace{10pt}
Monitors are more dynamic and should be used when you want to keep
track of a process without actually linking to it.
\pause\vspace{10pt}

\begin{verbatim}
  Ref = erlang:monitor(process, Pid),
  receive 
     {expected, Reply} ->
            erlang:demonitor(Ref, [flush]),
            ok(Reply);
     {'DOWN', Ref, process, Pid, Reason} ->
            something()
  end
\end{verbatim}

\end{frame}

\begin{frame}[fragile]{finding processes}

How do you get hold of process identifiers?

\pause\vspace{10pt}
You know your own ... and the identifiers of your children, ... and?

\pause\vspace{10pt}
\begin{verbatim}
init(State) ->
  receive 
     {logger, Logger} ->
          server(State, Logger)
  end
\end{verbatim}

\pause\vspace{10pt}
{\em Hmm, this might become hairy}

\end{frame}

\begin{frame}[fragile]{process registry}

Register processes under {\em names} in a process registry.

\pause\vspace{10pt}
The name, an atom, can be used to find the registered process.

\pause\vspace{10pt}
\begin{verbatim}
    :
  Logger = logger:start(),
  register(logger, Logger)
    :
\end{verbatim}

\pause\vspace{10pt}
\begin{verbatim}
    :
  case X of
    ok -> 
       dothis();
    ono -> 
       logger ! {error, ono},
       dothat()
  end
\end{verbatim}

\pause\vspace{10pt}
{\em Hmm, this might become hairy}

\end{frame}


\begin{frame}{summary}

\begin{itemize}
\item Processes created using spawn or better spawn\_link.
\item Process state represented by procedure arguments.
\item Messages handled by selective receive and implicit deferral.
\item Exit signals propagate through links.
\item Non-normal exit signals will kill a process.
\item Exit signals can be trapped and handled as messages.
\item Monitors more dynamic way of keeping track of a process.
\item Processes can be registered under names (pros and cons).
\end{itemize}
\end{frame}

\end{document}



