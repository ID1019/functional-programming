
\usepackage{fancyhdr}
\usepackage{lastpage}
\usepackage[utf8]{inputenc}

% Minted for syntax highliting
\usepackage{minted}
\usemintedstyle{tango}

% Headers/footers styling
\pagestyle{fancy}
\fancyhf{}
\renewcommand{\headrulewidth}{0pt}

% Footer
\lfoot{ID1019}
\cfoot{KTH}
\rfoot{\thepage \hspace{1pt} / \pageref{LastPage}}

%\newcommand{\defaultpagestyle}{\thispagestyle{plain}}
\newcommand{\defaultpagestyle}{\thispagestyle{fancy}}


\title[ID1019 Types]{Types}


\author{Johan Montelius}
\institute{KTH}
\date{\semester}

\begin{document}

\begin{frame}
\titlepage
\end{frame}

\begin{frame}[fragile]{Java}

\pause 
\begin{verbatim}
        :
    public static int fibonacci(int n) {

        int current = 0;
        int prevprev = 0, prev = 1;

        for(int i = 0; i < n; i++) { 
            current = prev + prevprev;  
            prevprev = prev;
            prev = current; 
        }
        return current;
    };
        :
\end{verbatim}

\end{frame} 

\begin{frame}[fragile]{Erlang}

\begin{verbatim}
fib(0) -> 0;
fib(1) -> 1;
fib(N) -> fib(N-1) + fib(N-2).
\end{verbatim}
\pause\vspace{10pt}
\begin{verbatim}
-spec fib(integer()) -> integer().
\end{verbatim}
\pause\vspace{10pt}
{\em .. or, is this the whole truth?}
\end{frame}

\begin{frame}[fragile]{Erlang}

{\em ... better:}
\vspace{20pt}

\begin{verbatim}
fib(0) -> 0;
fib(1) -> 1;
fib(N) when is_integer(N) and N > 1 -> 
   fib(N-1) + fib(N-2).
\end{verbatim}
\pause\vspace{10pt}
\begin{verbatim}
-spec fib(integer()) -> integer().
\end{verbatim}


\end{frame}

\begin{frame}[fragile]{the compiler}

Why does this compile?

\begin{verbatim}
test() ->
   fib(3.5).

fib(0) -> 0;
fib(1) -> 1;
fib(N) when is_integer(N) and N > 1 -> 
   fib(N-1) + fib(N-2).
\end{verbatim}

\pause\vspace{20pt}
{\em ... hmm, if the compiler only knew about types}

\end{frame}

\begin{frame}{types in Erlang}

What types do we have:
\pause\vspace{10pt}
Singletons, the types of individual data structures:
\begin{itemize}
\item  1, 2 or 42
\item 'foo', 'bar' or 'atom'
\item  \{foo, 42\}
\end{itemize}
\pause\vspace{10pt}
Unions of singletons, what we normally refer to as ``types'':
\begin{itemize}
\item integer(): any integer value 
\item float(): any floating point value 
\item atom(): any atom 
\item pid(): a process identifier
\item ref(): a reference
\item fun(): a function 
\item .. and many more
\end{itemize}

{\em ... why atom(), why not simply call the type atom?}
\end{frame}

\begin{frame}{types in Erlang}
Types for compound data structures:
\pause\vspace{10pt}
\begin{itemize}
\item  tuple(): a tuple of any form
\item  list(): a proper list of any length
\end{itemize}

\end{frame}


\begin{frame}[fragile]{type declarations}

We have seen this:
\begin{verbatim}
-spec fib(integer()) -> integer().
\end{verbatim}

\pause\vspace{10pt}
Cards are represented as  {\tt \{card, Suit,
Value\}}, where the suit is represented using the atoms {\tt spade},
{\tt heart}, {\tt diamond} and {\tt clubs}.

\vspace{10pt}
How do we specify the type for {\em suit/1}:

\begin{verbatim}
suit({card, Suit, _}) -> Suit.
\end{verbatim}
\pause\vspace{10pt}
\begin{verbatim}
-spec suit(tuple()) -> atom().
\end{verbatim}

\end{frame}

\begin{frame}[fragile]{defining types}

We would like to define our own type that specifies what a card looks like.

\begin{verbatim}
-type value() :: 1..13.
\end{verbatim}
\pause

\begin{verbatim}
-type suit() :: spade | heart | diamond | clubs.
\end{verbatim}

\pause
\begin{verbatim}
-type card() :: {card, suit(), value()}.
\end{verbatim}

\pause
\begin{verbatim}
-spec suit(card()) -> suit().
\end{verbatim}

\end{frame}

\begin{frame}[fragile]{defining types}

\pause
\begin{verbatim}
-type boolean() :: true | false.
\end{verbatim}

\pause
\begin{verbatim}
-type byte() :: 1..255.
\end{verbatim}

\pause
\begin{verbatim}
-type number() :: integer() | float().
\end{verbatim}
\pause
The type {\tt any()}, defines the union of all types.

\end{frame}


\begin{frame}[fragile]{defining types}

The type list(T) is the type of lists containing elements of type T.
\pause

\begin{verbatim}
-type list(T) :: [] | [T|list(T)].
\end{verbatim}

\pause
\begin{verbatim}
-type string() :: list(char()).
\end{verbatim}

Define the type of a deck of cards.
\pause
\begin{verbatim}
-type deck() :: list(card()).
\end{verbatim}

\end{frame}

\begin{frame}[fragile]{types of lists}

\begin{verbatim}
-type list(T) :: [] | [T|list(T)].
\end{verbatim}

\begin{verbatim}
-type nonempty_list(T) :: [T|list(T)].
\end{verbatim}

\begin{verbatim}
-type improper_list(T,Nil) :: 
      Nil | [T|improper_list(T,Nil)]
\end{verbatim}

\begin{verbatim}
-type maybe_improper_list(T,Nil) :: 
      Nil | [] | 
      [T|maybe_improper_list(T,Nil)]
\end{verbatim}

\end{frame}


\begin{frame}{the lattice of types}

\begin{tikzpicture}[>=stealth', node distance=2cm, semithick, auto]
    \node[state]        (any)                              {any()};
    \node[state]        (number) [below left of=any]       {number()};
    \node[state]        (integer)[below right of=number]   {integer()};
    \node[state]        (float)  [below left  of=number]     {float()};
    \node[state]        (atom)   [right of = integer]        {atom()};
    \node[state]        (list)   [right of = atom]         {list()};
    \node[state]        (tuple)   [right of = list]         {tuple()};
    \node[state]        (bottom) [below of = integer]          {$\perp$};

    \path[->]   (integer) edge       []    node      {}    (number);
    \path[->]   (float)   edge       []    node      {}    (number);
    \path[->]   (number)  edge       []    node      {}    (any);
    \path[->]   (atom)    edge       []    node      {}    (any);
    \path[->]   (list)    edge       []    node      {}    (any);
    \path[->]   (tuple)   edge       []    node      {}    (any);

    \path[->]   (bottom)  edge       []    node      {}    (integer);
    \path[->]   (bottom)  edge       []    node      {}    (float);
    \path[->]   (bottom)  edge       []    node      {}    (atom);
    \path[->]   (bottom)  edge       []    node      {}    (list);
    \path[->]   (bottom)  edge       []    node      {}    (tuple);

\end{tikzpicture}

\end{frame}

\begin{frame}{program annotation}

Type specifiers are used for:
\begin{itemize}
\item documentation of intended usage
\pause
\item automatic detection of type errors
\end{itemize}

\pause\vspace{20pt}
{\em the compiler does not check types}

\end{frame}

\begin{frame}[fragile]{exporting types}

We can export types from a module:

\begin{verbatim}
-module(cards).

-export_type([card/0, deck/0, suit/0, value/0]).
\end{verbatim}

\pause\vspace{20pt}

\begin{verbatim}
-module(poker).

-type hand() :: hand(cards:card()).
-type hand(C) :: {hand, C, C, C, C, C}.

\end{verbatim}

\end{frame}


\begin{frame}{TypEr and Dialyzer}

Why annotate programs if the annotations are not used?

\pause\vspace{20pt}

TypEr:  
\begin{itemize}
\item {\em infers} types, i.e. tries to determine the function specifications
\pause
\item checks that given specifications agree with the inferred types
\end{itemize}

Dialyzer:

\begin{itemize}
\item checks that given specifications agree with call patterns
\pause
\item detects exceptions and dead code 
\end{itemize}


\end{frame}

\begin{frame}[fragile]{TypEr example}

\begin{verbatim}
fib(0) -> 0;
fib(1) -> 1;
fib(N) -> fib(N-1) + fib(N-2).
\end{verbatim}

\pause
\begin{verbatim}
%% File: "fib.erl"
%% ---------------
-spec fib(non_neg_integer()) -> non_neg_integer().
\end{verbatim}

\pause {\em TypEr is not wrong but could give a more general proposal.}

\end{frame}

\begin{frame}[fragile]{Dialyzer example}

\begin{verbatim}
-module(test).

test() -> fib(a).
\end{verbatim}

\pause\vspace{20pt}

\begin{verbatim}
   :
test.erl:4: Function test/0 has no local return
test.erl:4: The call test:fib('a') will never return ...
   :
\end{verbatim}

\end{frame}

\begin{frame}[fragile]{problems}

What is the type of {\tt and/2}:

\begin{verbatim}
and(false, _) -> false;
and(_, false) -> false;
and(true, true) -> true.
\end{verbatim}

\end{frame}

\begin{frame}[fragile]{function overloading}

\begin{verbatim}
gr({card,_,V1}, {card, _, V2}) -> 
   V1 > V2;
gr(V1, V2) -> 
   V1 > V2.
\end{verbatim}

\pause\vspace{20pt}
\begin{verbatim}
   :
 case gr({card, heart,4}, {card, spdes, 2}) of
   :
 end.
\end{verbatim}

\pause\vspace{10pt}

\begin{verbatim}
   :
 case gr(7, 4) of
   :
 end.
\end{verbatim}

\end{frame}

\begin{frame}{dynamically typed}

Erlang is a {\em dynamically typed} language: types are checked and handled at run time.

{\em other dynamically typed languages: PHP, Python, Lisp}

\vspace{20pt}

Java is a {statically typed} language: types are checked and handled at compile time.

{\em other statically typed languages: C++, Haskell, Scala}

\end{frame}

\begin{frame}{dynamic vs static}

The pros and cons of dynamically typed languages:

\begin{itemize}
\item quick to write code
\item compiling an easier task
\item induces an overhead at run-time
\item errors detect first at run-time (and maybe very late)
\end{itemize}

{\em So why is Erlang dynamically typed?}

\end{frame}

\begin{frame}[fragile]{types and concurrency}

What is the type of {\tt server/1}:

\begin{verbatim}
server(State) ->
   receive 
       {query, X, Pid} ->
           Pid ! {reply, query(X, State)},
           server(State)
   end.
\end{verbatim}

\end{frame}

\begin{frame}[fragile]{types and dynamic code}

Erlang allows code to be updated while running.

\pause\vspace{10pt}

Very hard (limiting the functionality) to ensure that all type
information can be checked at compile time.

\pause\vspace{10pt}

A static type system would cover 90 percent of an Erlang system, but
it is the 10 percent that makes Erlang unique.

\end{frame}

\begin{frame}{strong vs weak}

There is no precise definition of strong vs weak type system. In
general a type system is said to be weak if you can change the types
of variables either at compile time and/or at run-time.

\pause\vspace{20pt}

Erlang is a {\em strongly typed} programming language: variables can not be reinterpreted.

\vspace{10pt}

{\em other strongly typed languages: Haskell, Lisp, Java, Python}

\pause\vspace{20pt}

C++ is a {\em weakly typed} programming language: variables can be reinterpreted.

\vspace{10pt}

{C++ inherits its most dangerous constructs from  C}

\end{frame}

\begin{frame}{Summary}

\begin{itemize}
\item Erlang is a dynamically typed programming language
\item External tools (TypEr, Dialyzer) can check for type errors.
\item Type specification, if correct, helps in understanding the code.
\item Dynamic vs static type systems - pros and cons.
\end{itemize}


\end{frame}

\end{document}


