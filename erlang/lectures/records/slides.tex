
\usepackage{fancyhdr}
\usepackage{lastpage}
\usepackage[utf8]{inputenc}

% Minted for syntax highliting
\usepackage{minted}
\usemintedstyle{tango}

% Headers/footers styling
\pagestyle{fancy}
\fancyhf{}
\renewcommand{\headrulewidth}{0pt}

% Footer
\lfoot{ID1019}
\cfoot{KTH}
\rfoot{\thepage \hspace{1pt} / \pageref{LastPage}}

%\newcommand{\defaultpagestyle}{\thispagestyle{plain}}
\newcommand{\defaultpagestyle}{\thispagestyle{fancy}}


\title[ID1019 Records]{Records}


\author{Johan Montelius}
\institute{KTH}
\date{\semester}

\begin{document}

\begin{frame}
\titlepage
\end{frame}

\begin{frame}[fragile]{name and age}

\pause 
\begin{verbatim}
-module(person).

new(Name, Age) ->
  {person, Name, Age}.

name({person, Name, _}) -> Name.

age({person, _, Age}) -> Age.
\end{verbatim}

\pause\vspace{20pt}
{\em Now let's add a phone number to our structure.}

\end{frame} 

\begin{frame}[fragile]{name, age and phone}

\pause 
\begin{verbatim}
-module(person).

new(Name, Age, Phone) ->
  {person, Name, Age, Phone}.

name({person, Name, _, _}) -> Name.

age({person, _, Age, _}) -> Age.

phone({person, _, _, Phone}) -> Phone.
\end{verbatim}

\pause\vspace{20pt}
{\em There should be a better way.}

\end{frame} 

\begin{frame}[fragile]{if we had types}

Warning - this is not Erlang

\begin{verbatim}
-type person :: {name :: string(), age :: integer()}


-spec new(string(), intger()) -> person().
new(Name, Age) ->
   person#{name=Name, age=Age}.

-spec name(person()) -> string().
name(P) -> P.name.

-spec age(person()) -> integer().
age(P) -> P.age.
\end{verbatim}
\pause
{\em Things would be easier with a static type system.}

\end{frame}

\begin{frame}{records}

\begin{center}
Records gives us some of the benefits of a static type system.
\end{center}
\pause\vspace{30pt}
\begin{center}
It is not a proper type system.
\end{center}
\end{frame}

\begin{frame}[fragile]{records}

\begin{verbatim}
-module(person).

-record(person, {name, age}).

new(Name, Age) ->
  #person{name=Name, age=Age}.
\end{verbatim}
\pause
\begin{verbatim}
name(#person{name=Name}) -> Name.

age(#person{age=Age}) -> Age.
\end{verbatim}

\end{frame} 

\begin{frame}[fragile]{defining records}

\begin{verbatim}
-record(Name, {Field [=Default], ...}
\end{verbatim}

\pause\vspace{20pt}
\begin{itemize}
\item Name : an atom 
\item Field : an atom naming the field
\item Default : an optional default value  (if not given 'undefined' is used)
\end{itemize}

\end{frame}

\begin{frame}[fragile]{constructing records}

\begin{verbatim}
      :
    Sune = person#{name = "Sune"},
      :
\end{verbatim}

\pause\vspace{20pt}
{\em It is not necessary to speciefy all fields.}
\end{frame}

\begin{frame}[fragile]{pattern matching}

\begin{verbatim}
name(#person{name=Name}) -> Name.
\end{verbatim}

\pause\vspace{10pt}

\begin{verbatim}
       :
    #person{name=Name} = Person,
       :
\end{verbatim}

\pause\vspace{10pt}

\begin{verbatim}
       :
    Name = P#person.name,
       :
\end{verbatim}

\end{frame}

\begin{frame}[fragile]{internal representation}

To know but never to use!
\pause\vspace{20pt}

Since we do not have a regular type declarations in Erlang we have to
fake it. 

\begin{verbatim}
-record(person, {name, age}).

new(Name, Age) ->
  #person{name=Name, age=Age}.
\end{verbatim}

\pause\vspace{20pt}

\begin{verbatim}
> person:new("Sune", 24).
{person,"Sune",24}
\end{verbatim}

\pause\vspace{20pt}
{\em A record structure is simply a tuple with the name of the type as it's first element.}

\end{frame}

\begin{frame}[fragile]{pre-processing}

\begin{verbatim}
>compile:file(person,['E']).
{ok, person}
\end{verbatim}

\pause\vspace{20pt}
\begin{block}{person.E}
\begin{verbatim}
  :
new(Name, Age) ->
    {person,Name,Age}.

name({person,Name,_}) ->
    Name.
  :
\end{verbatim}
\end{block}

\end{frame}

\begin{frame}{smart kid - not}

\begin{center}
Ahh, since I know that a record structure is just a tuple - then I can ....
\end{center}

\pause\vspace{20pt}
\begin{center}
   \includegraphics[width=0.5\linewidth]{badidea.jpg}
\end{center} 

\end{frame}

\begin{frame}[fragile]{a trick}

\begin{verbatim}
     :
   Name = element(#person.name, P),
     :
\end{verbatim}

\pause 
\begin{verbatim}
       :
    case lists:keysearch(49, #person.age, Persons) of
       {value, #person{name=Name}} -> 
             Name ++ " is soon turning 50.";
        false ->
             "No one will turn 50 anytime soon."
    end.
\end{verbatim}

\end{frame}

\begin{frame}[fragile]{how do we work with records}

Record declarations in a source file are local to the module.
\begin{block}{person.hrl}
\begin{verbatim}
-record(person, {name, age}).
\end{verbatim}
\end{block}

\pause
\begin{block}{party.erl}
\begin{verbatim}
   -include("person.hrl").

     Sune = person:new("Sune", 24),
       :
     "Hi " ++ Sune#person.name.

\end{verbatim}
\end{block}

{\em Hmmm, should we allow the party module to know anything about our representation?}
\end{frame}


\begin{frame}[fragile]{abstract data types}


\begin{block}{party.erl}
\begin{verbatim}
-include("person.hrl").

  Sune = person:new("Sune", 24),
    :
  "Your ref is: " ++ Sune#person.ref.
\end{verbatim}
\end{block}

\pause

\begin{block}{party.erl}
\begin{verbatim}

  Sune = person:new("Sune", 24),
    :
  "Hi " ++ person:name(Sune) ++ " you're invited!"

\end{verbatim}
     \end{block}
\pause\vspace{10pt}
{\em Only exporting functions gives us more control over what is used outside of the module.}

\end{frame}

\begin{frame}[fragile]{let's only export functions}

Assume that your waiting for a message from the {\em person process} (that we have not define):

\pause\vspace{20pt}

\begin{block}{party.erl}
\begin{verbatim}
    :
  receive 
      P -> 
           person:name(P) ++ " let's go!"
  end.
\end{verbatim}
     \end{block}

\end{frame}

\begin{frame}[fragile]{deferring messages}


\begin{block}{party.erl}
\begin{verbatim}
-include("person.hrl").
    
    :
  receive 
      #person{name=Name} -> 
           Name ++ ``let's go!''
  end.
\end{verbatim}
\end{block}

\end{frame}

\begin{frame}{records and messages}

Since we want to defer messages we need to expose the structure.

\pause\vspace{20pt}
Making a record declaration visible, and restricting usage to a few
fields (that will not change) is much better than announcing the full
tuple representation.

\pause\vspace{20pt}

\end{frame}

\begin{frame}[fragile]{updating message structure}

How do we update a message structure of a process without changing the
receivers?

\pause\vspace{20pt}

Records help us, but not all the way.

\pause\vspace{20pt}

Optional information in a list?

\end{frame}


\begin{frame}[fragile]{updating message structure}

\begin{block}{person.hrl}
\begin{verbatim}
-record(person, {name, age, opt=[]}).
\end{verbatim}
\end{block}

\pause\vspace{10pt}
\begin{block}{party.erl}
\begin{verbatim}
-include("person.hrl").
    
    :
  receive 
      #person{name=Name, opt=Opt} -> 
           case lists:keysearch(phone, 1, Opt) of
             {found, {phone, Nr}} ->
                   Name ++ ``: phone: '' ++ Nr;
             false ->
                   Name ++ " no phone"
           end
  end.
\end{verbatim}
\end{block}


\end{frame}



\begin{frame}{Summary}

Records are good!

\pause\vspace{20pt}
Use them as much as possible.

\pause\vspace{20pt}
It is not a type system :-(


\end{frame}


\end{document}


