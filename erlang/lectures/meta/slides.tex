
\usepackage{fancyhdr}
\usepackage{lastpage}
\usepackage[utf8]{inputenc}

% Minted for syntax highliting
\usepackage{minted}
\usemintedstyle{tango}

% Headers/footers styling
\pagestyle{fancy}
\fancyhf{}
\renewcommand{\headrulewidth}{0pt}

% Footer
\lfoot{ID1019}
\cfoot{KTH}
\rfoot{\thepage \hspace{1pt} / \pageref{LastPage}}

%\newcommand{\defaultpagestyle}{\thispagestyle{plain}}
\newcommand{\defaultpagestyle}{\thispagestyle{fancy}}


\title[ID1019 Meta Interpreter]{Meta Interpreter}


\author{Johan Montelius}
\institute{KTH}
\date{\semester}

\begin{document}

\begin{frame}
\titlepage
\end{frame}

\begin{frame}
\begin{figure}
  \includegraphics[width=\linewidth]{breaking.jpg}
\end{figure}

\end{frame}

\begin{frame}{interpreter}

Let's write a program that evaluates an expression in our language. 

\pause\vspace{20pt}

Since the language we have defined is close to Erlang and we will
implement the interpreter in Erlang, we call it a {\em meta
interpreter}. 

\pause\vspace{20pt}

{\em This is more common than you might think, gcc - a very popular
C/C++ compiler is written in ... C/C++. The Java compiler javac is
written in ... Java.}

\end{frame}


\begin{frame}{a recap}

  \begin{tabular}{r l l}
   {\em Atoms} & = & \{a, b, c, \ldots\} \\
   {\em Structures} & = & {\em Atoms} $\cup$ \{ \{a, b\} \textbar a $\in$ {\em Structures}  $\wedge$  b $\in$ {\em Structures} \}
  \end{tabular}

\end{frame}


\begin{frame}{expressions and patterns}

\begin{grammar}
<atom> ::= a | b | c | \ldots

<variable> ::= X | Y | Z | \ldots

<expression> ::= <atom> | <variable> | '\{' <expression> ',' <expression> '\}'

<pattern> ::= <atom> | <variable> | '\_' | '\{' <pattern> ',' <pattern> '\}'

<match> ::=  <pattern> '=' <expression>

<sequence> ::=  <expression> | <match> ',' <sequence>
\end{grammar}
\end{frame}

\begin{frame}{operational semantics}

 \begin{itemize}
   \pause \item $E\sigma(a) \rightarrow s$ if $a \equiv s$ 
   \pause \item $E\sigma(v) \rightarrow s$ if $v/s \in \sigma$
   \pause \item $E\sigma(\lbrace e_1 , e_2\rbrace) \rightarrow \lbrace E\sigma(e_1), E\sigma(e_2) \rbrace$
 \end{itemize}

\begin{itemize}
  \pause \item $E\sigma(p = e, {\rm sequence}) \rightarrow E\theta({\rm sequence})$ if  $E\sigma(e) \rightarrow s \wedge P\sigma(p, s) \rightarrow \theta$
\end{itemize}

\end{frame}

\begin{frame}{operational semantics}

\pause
\begin{itemize}
  \pause \item $P\sigma(a, s) \rightarrow \sigma$  if  $a \equiv s$
  \pause \item $P\sigma(\_, s) \rightarrow \sigma$  
  \pause \item $P\sigma(v, s) \rightarrow \sigma$  if \pause $ v/s \in \sigma $
  \pause \item $P\sigma(v, s) \rightarrow \lbrace v/s \rbrace \cup \sigma$ if \pause $ v/t \not\in \sigma$
  \pause \item $P\sigma(\lbrace p_1, p_2 \rbrace , \lbrace s_1, s_2 \rbrace) \rightarrow \theta$ if
      $P\sigma(p_1 , s_1) \rightarrow \theta' \wedge P\theta'(p_2 , s_2) \rightarrow \theta$
\end{itemize}
\pause
\begin{itemize}
  \pause \item $P\sigma(t = s) \rightarrow$ \pause $\  \mathrm{fail}\ $ if no other rule apply
\end{itemize}

\end{frame}


\begin{frame}{outline}

\pause Our language:

\begin{itemize}
\pause \item Data structures: atoms and compound structures
\pause \item Expressions: terms, patterns, matching, sequences ...
\end{itemize}

\pause Operational semantics:

\begin{itemize}
\pause \item Environment: a mapping from variables to values
\pause \item Eval: evaluation of an expression, given an environment
\pause \item Match: pattern matching of pattern and data structure given an environment
\end{itemize}

\pause {\em Our implementation must be {\em sound} and (hopefully) {\em complete}}.
\end{frame}

\begin{frame}{things to do}

\begin{itemize}
 \pause \item define a representation 
  \begin{itemize}
     \pause \item of data structures (this is the easy part)
     \pause \item of a sequence of expressions (also easy)
     \pause \item of an environment (simple)
  \end{itemize}
 \pause \item implement functions
  \begin{itemize}
    \pause \item that can map variables to data structures given an environment (even easier)
     \pause \item that takes an expression and returns a data structure (hmmm, yes easy)
     \pause \item that takes a pattern and a data structure and returns an updated environment (actually easy)
     \pause \item that takes a sequence and an environment and returns a structure (you would not beleive me, but it's easy)
  \end{itemize}
\end{itemize}

\end{frame}

\begin{frame}[fragile]{high level description}

\begin{itemize}
 \pause \item \verb+eval_expr(Exr, Env)+ :  evaluate the expression, given the environment
 \pause \item \verb+eval_seq(Sequence, Env)+ :  evaluate the sequence, given the environment
 \pause \item \verb+eval_match(Patter, Sructure, Env)+ :  match the pattern to the structure
\end{itemize}

\pause\vspace{20pt}{Sound easy - let's go!}
\end{frame}

\begin{frame}{data structures}

  \begin{tabular}{r l l}
   {\em Atoms} & = & \{a, b, c, \ldots\} \\
   {\em Structures} & = & {\em Atoms} $\cup$ \{ \{a, b\} \textbar a $\in$ {\em Structures}  $\wedge$  a $\in$ {\em Structures} \}
  \end{tabular}

\pause  \vspace{20pt}

Let's represent {\em atoms} as Erlang atoms and compound {\em structures} as Erlang tuples.

\end{frame}


\begin{frame}{term expressions}

\begin{grammar}
<atom> ::= a | b | c | \ldots

<variable> ::= X | Y | Z | \ldots

<literal> ::= <atom>

<pattern> ::= <literal> | <variable> | '\_' | '\{' <pattern> ',' <pattern> '\}'

<expression> ::= <literal> | <variable> |  '\{' <expression> ',' <expression> '\}'
\end{grammar}

\pause We need to be able to look at an element an determine if it is an atom, variable or compound structure.

\end{frame}

\begin{frame}[fragile]{representing terms and patterns}

\pause\vspace{20pt}
\begin{itemize}
 \pause \item atoms: {\tt \{atm, a\}, \{atm, b\}, \ldots} 
 \pause \item variables: {\tt \{var, x\}, \{var, y\}, \ldots} 
 \pause \item \_:  \verb+ignore+.
 \pause \item compound: {\tt \{cons, \{atm, c\}, \{var, z\}\}, \ldots}
\end{itemize}

\end{frame}

\begin{frame}[fragile]{representing sequences}

\begin{grammar}
<pattern matching> ::=  <pattern> '=' <expression>

<sequence> ::=  <expression> | <pattern matching> ',' <sequence>
\end{grammar}

\pause\vspace{20pt} Why not represent it as a list of tuples, \verb+{match, Pattern, Expression}+, ending in a single expression.

\end{frame}

\begin{frame}[fragile]{parse and interpret}

\pause A parser would take a source program (or in our case a source sequence):

\pause\vspace{10pt}\hspace{40pt}\verb+ X = a, Y = {X,b}, {c, Y}+

\pause \vspace{10pt} and transform it to a representation

\pause\vspace{10pt}
\begin{verbatim}
   [{match, {var, x}, {atm, a}}, 
    {match, {var, y}, {cons, {var, x},{atm, b}}, 
    {cons, {atm, c}, {var, y}}]
\end{verbatim}
\pause \vspace{10pt} that is evaluated to return a data structure

\pause\vspace{10pt}\hspace{40pt}\verb+{c {a,b}}+

\end{frame}


\begin{frame}[fragile]{an environment}

The environment will map variables to values.

\pause\vspace{20pt}
A list of tuples \verb+{Variable, Structure}+ will do fine.

\pause\vspace{20pt}
\hspace{40pt}\verb+[{x, a}, {y, {b,c}}, {z,d}]+

\pause\vspace{20pt}
.... ehh, why not a tree, $O(lg(n))$ is better than $O(n)$, ... 

\end{frame}

\begin{frame}[fragile]{abstraction}

Separate the implementation of the environment in its own module.

\pause\vspace{20pt}Export a set of functions and only use these in the interpreter.


\begin{itemize}
  \pause\item \verb+new()+ : return a new environment
  \pause\item \verb+add(Var, Str, Env)+ : return an environment where the variable is bound to the structure
  \pause\item \verb+lookup(Var, Env)+ : return either \verb+{Var, Str}+ or \verb+false+ 
\end{itemize}


\pause\vspace{20pt}You have implemented an ``abstract data type''.

\end{frame}

\begin{frame}[fragile]{evaluate an expression}

\begin{verbatim}
eval_expr({atm, A}, Env) -> ...
\end{verbatim}
\begin{verbatim}
eval_expr({var, V}, Env) -> ...
\end{verbatim}
\begin{verbatim}
eval_expr({cons, H, T}, Env) -> ...
\end{verbatim}

\end{frame}


\begin{frame}[fragile]{pattern matching}

We're matching a pattern to a data structure, returning either \verb+{ok, Env}+ or \verb+fail+.

\begin{verbatim}
eval_match({ignore, _, Env) -> ...
\end{verbatim}
\pause
\begin{verbatim}
eval_match({var, V}, S, Env) -> ... 
\end{verbatim}
\pause
\begin{verbatim}
eval_match({atm, A}, A, Env) -> ...
\end{verbatim}
\pause
\begin{verbatim}
eval_match({cons, P1, P2},{S1,S2}, Env) -> ... 
\end{verbatim}
\pause
\begin{verbatim}
eval_match(_, _, Env) -> ...
\end{verbatim}

\pause{\em We're so close.}
\end{frame}

\begin{frame}[fragile]{evaluating a sequence}

\begin{verbatim}
eval_seq([Expresion], Env) -> ...
\end{verbatim}

\begin{verbatim}
eval_seq([{match, Pattern, Expresion}|Sequence], Env) ->
    case ... of
        error -> 
            ...;
        {ok, Str}  ->
            case ... of
                fail ->
                    error;
                {ok, ...} ->
                    ...
            end
    end.
\end{verbatim}

\pause{\em let's take her for a spin}

\end{frame}

\begin{frame}{benchmark (full interpreter)}

\pause Naive reverse of a list of 20 element (an $O(n^2)$ algorithm).

\begin{columns}
 \begin{column}{0.5\linewidth}
  \begin{itemize}
   \pause \item The interpreter
   \pause \item Erlang using tuples
   \pause \item Erlang using lists
   \pause \item Erlang using '++'
  \end{itemize}
 \end{column}
 \begin{column}{0.5\linewidth}
  \begin{itemize}
   \pause \item 3.9 s
   \pause \item 84 ms (x46 speed up)  
   \pause \item 36 ms (x110 speed up) 
   \pause \item 21 ma (x190 speed up) 
  \end{itemize}
 \end{column}
\end{columns}

\vspace{20pt}
\pause{\em Actually not that bad for an hour of work.}

\end{frame}

\begin{frame}[fragile]{case expression}

\begin{verbatim}
   case  X  of 
       a ->  foo;
       b ->  bar
   end
\end{verbatim}

\end{frame}

\begin{frame}{case expression}

\begin{grammar}
     <expression> ::=  <case expression> | ...  

     <case expression> ::= 'case' <expression> 'of' <clauses>  'end' 

     <clauses> ::=   <clause> | <clause> ';' <clauses>

     <clause> ::=  <pattern> '->' <sequence>
\end{grammar}

\pause How should we represent a case expression? 

\pause How should we represent a sequence of clauses?

\pause How should we represent a clause?

\end{frame}


\begin{frame}{evaluation of case expression}
 
$E\sigma({\tt case}\ e \ {\tt of}\ {\rm clauses} \ {\tt end}) \rightarrow C\sigma(s, {\rm clauses})$ if $ E\sigma(e) \rightarrow s$

\pause\vspace{10pt}
$C\sigma(s, p \rightarrow  {\rm sequence} ; {\rm clauses}) \rightarrow E\sigma'({\rm sequence})$ if $P\sigma(p = s) \rightarrow \sigma'$

\pause\vspace{10pt}otherwise 

\pause\vspace{10pt}
$C\sigma(s, {\rm clause} ; {\rm clauses}) \rightarrow C\sigma(s, {\rm clauses})$ 

\pause\vspace{20pt}
If no successful clause is found, return an error ($\perp$).

\end{frame}


\begin{frame}{why}

\pause\vspace{40pt}\hspace{60pt}Why do we do this?

\end{frame}

\end{document}
