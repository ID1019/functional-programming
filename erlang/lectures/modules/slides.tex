
\usepackage{fancyhdr}
\usepackage{lastpage}
\usepackage[utf8]{inputenc}

% Minted for syntax highliting
\usepackage{minted}
\usemintedstyle{tango}

% Headers/footers styling
\pagestyle{fancy}
\fancyhf{}
\renewcommand{\headrulewidth}{0pt}

% Footer
\lfoot{ID1019}
\cfoot{KTH}
\rfoot{\thepage \hspace{1pt} / \pageref{LastPage}}

%\newcommand{\defaultpagestyle}{\thispagestyle{plain}}
\newcommand{\defaultpagestyle}{\thispagestyle{fancy}}


\title[ID1019 Modules]{Modules}

\author{Johan Montelius}
\institute{KTH}
\date{\semester}

\begin{document}

\begin{frame}
\titlepage
\end{frame}

\begin{frame}[fragile]{the module}

\begin{verbatim}
%%% A comment at the begining of a row.

-module(set).

-export([new/0,  union/2, intersection/2,]).

new() -> []. % a coment at the end of a line

union(A, B) -> 
         %% a comment that is intended
         lists:append(A, B).
\end{verbatim}
\end{frame}

\begin{frame}{a module}

What is a module: 

\begin{itemize}
\pause
\item an abstraction layer
\pause
\item hiding of implementation details
\pause
\item the unit of compilation
\pause
\item the unit of loading
\end{itemize}

\pause\vspace{40pt}
{\em The file name should be the same as the module name.}

\end{frame}

\begin{frame}[fragile]{exporting an interface}

Exported functions are visible utside of the module \ldots the API.
\pause

\begin{verbatim}
-export([foo/1, bar/2]).
\end{verbatim}

\pause {\em A way of hiding the internals, allows for changes and compiler optimizations.}
\end{frame}

\begin{frame}[fragile]{exporting an interface}
Imported functions can be used in the module without using the module prefix.

\begin{verbatim}
-import(lists, [append/2, delete/2, member/2]).

   :

set_diff(As, []) ->
            As;
set_diff(As, [B|Bs]) ->
            set_diff(delete(B, As), T).
\end{verbatim}
\pause \vspace{20pt}


\end{frame}

\begin{frame}[fragile]{import or not}

\pause \vspace{20pt}
Exported functions can be used explicitly in the code:

\begin{verbatim}
insert_all([Key|Rest], Store) ->
        inser_all(Rest, foo:insert(Key, ok, Store)).
\end{verbatim}

\pause \vspace{20pt}
or made available by an {\tt import} attribute:

\begin{verbatim}
-import(foo, [new/1, insert/2, lookup/2]).

insert_all([Key|Rest], Store) ->
        inser_all(Rest, insert(Key, ok, Store)).
\end{verbatim}
\pause \vspace{20pt}
Pros and cons?

\end{frame}

\begin{frame}[fragile]{more attributes}

\begin{verbatim}
-compile(debug_info).  % have you found the debugger?
-compile(export_all).  % very handy 
\end{verbatim}
\pause \vspace{20pt}

\begin{verbatim}
-include("my_macros.hrl").   % macros - what macros?
\end{verbatim}
\pause \vspace{20pt}

{\em many more attributes, even user defined}

\pause \vspace{20pt}

\begin{verbatim}
 > lists:module_info().
\end{verbatim}

\end{frame}

\begin{frame}[fragile]{user defined attributes}
\begin{verbatim}
-module(test).

-foo("this is it").

-bar([gurka, 42]).
\end{verbatim}
\pause
 \vspace{20pt}
\begin{verbatim}
 > test:module_info(attributes).
\end{verbatim}


\end{frame}

\begin{frame}[fragile]{an overview}

The stages of a source code to running program:
\pause
\begin{itemize}
\item preprocess: source to source transformation
\pause
\item compilation: source to machine code transformation
\pause
\item link: link inter-module procedure calls
\pause
\item load: load to (virtual) machine 
\pause
\item run: ... start executing
\end{itemize}
\pause \vspace{20pt}

{\em These stages are more or less visible, depending on language environment.}

\end{frame}

\begin{frame}[fragile]{macros}

\begin{verbatim}
-module(test).

-define(pi, 3.14).
\end{verbatim}

\pause
\begin{verbatim}
area(R) -> math:pow(R,2)*?pi.
\end{verbatim}
\pause

\pause
\begin{verbatim}
> compile:file(test, ['P']).
\end{verbatim}

\pause\vspace{20pt}

{\em \ldots check test.P}

\end{frame}


\begin{frame}[fragile]{macros}

\begin{verbatim}
-module(test).

-define(TRACE(Msg), io:format("trace: ~s~n", [Msg])).
%-define(TRACE(Msg), true).
\end{verbatim}

\pause
\begin{verbatim}
area(R) -> 
   ?TRACE("the area function was called"),
   math:pow(R,2)*?pi.
\end{verbatim}
\pause \vspace{20pt}

{\em make sure the result has correct syntax}

\end{frame}

\begin{frame}[fragile]{conditional macros}

\begin{verbatim}
-module(test).

-ifdef(debug).
-define(TRACE(Msg), io:format("trace: ~s~n", [Msg])).
-else.
-define(TRACE(Msg), true).
-endif.
\end{verbatim}

\pause
\begin{verbatim}
> c(test, [{d, debug}]).
\end{verbatim}
\pause \vspace{20pt}
\pause
\begin{verbatim}
$ erlc -Ddebug test.erl
\end{verbatim}
\pause

\end{frame}

\begin{frame}{predefined macros}

Some useful predefined macros, useful when debugging and tracing.

\begin{itemize}
\pause
\item {\tt ?MODULE}: the name of the current module.
\pause
\item {\tt ?LINE}: the current line number.
\pause
\item \ldots some more
\end{itemize}

\end{frame}

\begin{frame}{macros or functions}

Anything that can be done with macros can be done with functions.

\pause \vspace{20pt}

Pros and cons?

\end{frame}

\begin{frame}{Erlang modules}

{\em Erlang uses a flat name space for modules - not good.}

\pause\vspace{20pt}

Useful library modules:
\pause\vspace{20pt}

\begin{itemize}
\item {\bf erlang}: automatically imported, (mostly) builtins that can not be implmented in Erlang itself
\pause
\item {\bf lists}: efficient implementation, some in C, of the most comon functions over lists
\pause
\item {\bf math}: mathematical expressions
\pause
\item {\bf file}: handling files, open, close, byte oriented operations
\pause
\item {\bf io}: read and write of terms etc
\end{itemize}

\pause\vspace{20pt}

{\em \ldots library modules grouped in a hierachy in the manual}

\end{frame}






\end{document}


