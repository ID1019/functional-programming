\documentclass[a4paper,11pt]{article}

\usepackage{ifthen}
\usepackage[latin1]{inputenc}
\usepackage{verbatim}

\usepackage[breaklinks=true,
            bookmarksnumbered=true,
            pdftitle={Distributed Computing},
            pdfauthor={Johan Montelius},
            pdfsubject={Distributed Mandelbrot set},
            pdfkeywords={distributed computing erlang}
            ]{}

\newcommand{\nnsection}[1]{
\section*{#1}
\addcontentsline{toc}{section}{#1}
}

\begin{document}

\begin{center}
\vspace{20pt}
\textbf{\large Generating a Mandelbrot image }\\
\vspace{10pt}
\textbf{Johan Montelius}\\
\vspace{10pt}
\today{}
\end{center}

\nnsection{Introduction}

In this exercise you will implement a Mandelbrot set generator, or
rather an image generator. You should do some reading so that you
understand the basics of what the Mandelbrot set is and why it can
generate some beautiful images; this text only contains a minimal
explanation.

\section{Mandelbrot}

The Mandelbrot set is defined as the set of complex numbers $c$ for
which the sequence $z_n$ does not approach infinity. The value $z_n$
is defined as follows:

\begin{eqnarray*}
    z_0 &= &0 \\ 
    z_{n+1} & = &z_n� + c
\end{eqnarray*}

If you remember how to do the square of a complex numbers you know
everything there is to know to start:

$$ (x + yi)� = x� - y� + 2xyi$$

How do we know if a complex number $(x + yi)$ belongs to the
Mandelbrot set? We could of course start to compute $z_n$ for higher
values and see if we approach infinity but that would of course take
a very long time (to say the least). 

An observation saves us from spending the rest of our lives computing
$z_n$: {\em if $|z_n| >= 2$ then there is no turning back, $z_n$ will
  only increase in size}. The absolute value of a complex number is of
course:

$$|(x + yi)| = \sqrt{x� + y�}  $$

We still do not want to compute forever; if the number actually does
belong to the Mandelbrot set we will of course never hit the
threashold. Therefore, we set an upper limit $n$ that will be the
maximum {\em depth} of our computation.

So given a maximum value of $n$, we can for any complex number $c$ say
if it {\em definitely does not belong to} or if it {\em could possibly
  belong to} the Mandelbrot set. In the case where we know for sure
that the number does not belong to the set we also have a value $i$
which was the point where $|z_i| >= 2$. This value $i$ is the color we
need to generate a beautiful Mandelbrot image.

\subsection{complex numbers}

Since we are going to work with complex numbers we might as well start
by implementing a module to handle these. Let's make it simple and
represent a complex number as a tuple with its real and imaginary
values. Create a module {\tt cmplx} that exports the following
functions:

\begin{itemize}
 \item {\tt new(X, Y)} : returns the complex number with real value X and imaginary Y
 \item {\tt add(A, B)} : adds two complex numbers
 \item {\tt sqr(A)} : squares a complex number
 \item {\tt abs(A)} : the absolute value of A
\end{itemize}

You might want to use the {\tt sqrt/1} function exported from the {\tt
  math} module when calculating the absolute value.

\subsection{the brot module}

For no reason at all we will call our first module {\tt brot}, it will
implement the computation of the $i$ value given a complex value
$c$. We must of course give it a maximum iteration {\em depth} or we
risk getting stuck in an infinite computation.

Implement a function {\tt mandelbrot/2} that, given the complex number
$c$ and the maximum number of iterations $m$, return the value $i$ at
which $|z_i| >= 2$ or $0$ if it does not for any $i < m$ i.e. it
should always return a value in the range $0..(m-1)$.

\begin{verbatim}
mandelbrot(C, M) -> 
    Z0 = cmplx:new(...,...),
    I = 0,
    test(I, Z0, C, M).
\end{verbatim}

The {\tt test/4} function should of course test if we have reached the
maximum iteration, in which case it returns zero, or if the absolute
value of {\tt Z} is greater than $2$, in which case it returns {\tt
  I}. Make sure that you use the functions exported from the {\tt
  cmplx} module.

Do some test to see that the function works (here I'm writing the
complex numbers directly knowing that we represent them as tuples,
this is of course cheating but very convenient).

\begin{itemize}
 \item {\tt brot:mandelbrot(\{0.8, 0\}, 30)}
 \item {\tt brot:mandelbrot(\{0.5, 0\}, 30)}
 \item {\tt brot:mandelbrot(\{0.3, 0\}, 30)}
 \item {\tt brot:mandelbrot(\{0.27, 0\}, 30)} 
 \item {\tt brot:mandelbrot(\{0.26, 0\}, 30)}
 \item {\tt brot:mandelbrot(\{0.255, 0\}, 30)}
\end{itemize}

What is happening? Which values could possibly belong to the
Mandelbrot set - how sure are you? Do some more testing, why stop at
thirty iterations? Try fifty!

\subsection{the printer}

Before carrying on we should make sure that we can generate an
image. You should have the file {\tt ppm} module that will write the
final image to a file. Make sure that you can use this module and that
you know where files are located when created.

The api to the module is:

\begin{itemize}
 \item {\tt write(Name, Image)}: where the name is the name (possibly
   full path name) of the file and the image is a list of rows where
   each row is a list of tuples {\tt \{R,G,B\}} (each value being in
   the range 0..255).
\end{itemize}

So once we know that it is working we can carry on to produce some images. 

\subsection{colors}

We create one module (that in the end will be the one that you want to
play with the most), the {\tt color} module. This module should export
a function {\tt convert(Depth, Max)} that given a depth on a scale
from zero to max gives us a color,

The conversion of depth information to {\em RGB values} can of course
be done in many different ways and the one presented here is only for
inspiration.

Let's assume that we have a depth of a point $D$, with the maximum
possible depth being $M$. We could create five sections that divides
the range $0$ to $M$. Divide $D$ by $M$ and so that you have a
fraction $F$. Then multiply this fraction by four to generate a
floating point $A$ from $0$ to $4$. Now truncate the value to give you
an integer $X$ from $0$ to $4$ (this is the section) and generate an
offset $Y$ that is the truncated value of $255*(A-X)$.

The two values $X$ and $Y$ will now be used to give you an RGB
value. You can use the following transformation:

\begin {itemize}
\item 0 : $\lbrace Y, 0, 0 \rbrace$
\item 1 : $\lbrace 255, Y, 0 \rbrace$
\item 2 : $\lbrace 255-Y, 255, 0 \rbrace$
\item 3 : $\lbrace 0, 255, Y \rbrace$
\item 4 : $\lbrace 0, 255-Y, 255 \rbrace$
\end{itemize}

What colors does this correspond to? Does it look anything like a
rainbow? Close to a rainbow? The mapping from depth to colors is one
thing that one can play with, its not at all given that the colors
should be chosen base only on the depth, one might even want to know
the distribution of depths in the whole image or reuse colors at
different depths.


\subsection{computing the set}

So we know how to find the depth of a complex number so why not try to
compute the depth at all points in a rectangular plane. We create a
module {\tt mandel} that should calculate an image. The function that
will be our interface to the module looks like this:

\begin{verbatim}
mandelbrot(Width, Height, X, Y, K, Depth) ->
    Trans = fun(W, H) -> 
              cmplx:new(X + K*(W-1), Y-K*(H-1))
            end,
    rows(Width, Height, Trans, Depth, []).
\end{verbatim}

What is happening here? We want to generate an image of the size {\em
  Width x Height}. The upper left corner of this image is the point $X
+ Yi$ and the offset between two points is $K$. This means that the
first pixels of the upper row should correspond to the ``depth'' of
$X+Yi$, $(X+K) + Yi$, $(X+2K) + Yi$ etc and that the second row starts
with $X + (Y-K)i$. 

To help the mandelbrot generator from keeping track of this we simply
provide a function that does the work. The {\em Trans} function will
take a pixel position ($W$,$H$) and return a complex number that is
the one we should compute the depth of. It is better to do this here
and then we could more easily change the function rather than passing
all the necessary information in arguments.

Now the {\em rows} function should return a list of rows, where each
row is a list of colors. Each item in a row corresponds to a pixel at
($W$, $H$) and the color is computed by:

\begin{itemize}
 \item generating the complex number that corresponds to the pixel
 \item calculate the depth of this value
 \item convert the depth to a color
\end{itemize}

The only tricky issue is to generate the rows in ``correct'' order, it
is easy to generate the image up side down or mirrored. In the end it
does not mean very much but try to get it right.

When you can generate an image, write it to a file using the {\tt ppm}
module. This code would hopefully give you a first look at the Mandelbrot set.

\begin{verbatim}
demo() ->
    small(-2.6,1.2,1.6).

small(X,Y,X1) ->
    Width = 960,
    Height = 540,
    K = (X1 - X)/Width,
    Depth = 64,
    T0 = now(),
    Image = mandelbrot(Width, Height, X, Y, K, Depth),
    T = timer:now_diff(now(), T0),
    io:format("picture generated in ~w ms~n", [T div 1000]),
    ppm:write("small.ppm", Image).
\end{verbatim}

\section{Carrying on}

Generate a nice looking image, you will find the most interesting
things close to the edge of the black set. This is where the fractals
start to spin out of control and the beauty of the Mandelbrot set is
found. It's amazing that so much information could be hidden in a
function this simple.

\begin{eqnarray*}
    z_0 &= &0 \\ 
    z_{n+1} & = &z_n� + c
\end{eqnarray*}

Could we speed up the calculations? Are there any operations that are
of unnecessary complexity? Can you include an image in your report
(you would probably have to convert it to png)?



\end{document}
