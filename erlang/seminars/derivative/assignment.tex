\documentclass[a4paper,11pt]{article}

\usepackage{wrapfig}
\usepackage{ifthen}
\usepackage[latin1]{inputenc}

\newcommand{\nnsection}[1]{
\section*{#1}
\addcontentsline{toc}{section}{#1}
}

\begin{document}

\begin{center}
\vspace{20pt}
\textbf{\large Taking the derivative}\\
\vspace{10pt}
\textbf{Johan Montelius}\\ 
\vspace{10pt}
\textbf{\today}
\end{center}

\nnsection{Introduction}

You of course know how to take the derivative of a function, this is
something you probably learned already in high school. There are some
simple rules that one applies and in five minutes one can find the
derivative of even complicated functions.

Our task in this tutorial is to work with functions as symbolic
expressions and compute the derivative of functions. Part of the
problem is to understand how we can represent functions using the data
structures that we have in Erlang. Once we know how to do this the
derivation will be quite simple.

\section{Representing functions}

You might wonder what the problem is; after all we represent a function in Erlang using the syntax:

\begin{verbatim}
fun(X) -> X * 2 end
\end{verbatim}

This is only partly true, what we do is express what the function
should look like, how it is actually represented is hidden to us. If we
write:

\begin{verbatim}
 F = fun(X) -> X * 2 end
\end{verbatim}

then {\tt F} is a function of one argument that we can use, but if we
only have access to {\tt F} we can not determine that it is a function
of one argument nor that the the body of the function is a
multiplication of the argument and a constant.

In order to derivate the function {\tt F}, with respect to {\tt X}, we
need to know that the body of the expression is {\tt X * 2}. We need to
be able to examine the expressions {\tt X*2}, and determine that it is
a product of the variable {\tt X} and {\tt 2}, only then can we
determine that the derivative is $2$. So, just because we have {\em
  functions} in Erlang, does not mean that we necessarily should use
them to represent our ``functions''. 

\subsection{an expression}

Let's start by finding a representation for constants and
variables. We can limit ourselves to the domain of reals and a real
number could of course be represented by a Erlang number. This is of
course trivial but remember to separate the real number $2.34$ from the
representation in Erlang by {\tt 2.34}. A variable could be represented
as an atom so the variable $x$ is represented by the Erlang atom {\tt
  x}.

This simple mapping of elements in our domain to Erlang data structures
is tempting to use but it has some disadvantages, we can not use
pattern matching to determine if an element is a number or a
variable. In our program we would have to use the built-in recognizers
to separate the two cases; we would use code as the following:

\begin{verbatim}
derivative(N) when is_number(N) -> 
          ....;
derivative(N) when is_atom(N) -> 
          ....;
\end{verbatim} 

We would also have to think twice when we want to include constants
such as $\pi$, could we represent it using the atom {\tt pi}?
Moreover, when we derivatives, is it important to separate the
constant $\pi$ from the constant $2.34$?

A better approach (all though it will turn our expressions into huge
data structures) is to be more explicit in our choice of
representation. What if we choose to represent all constants using the
tuple {\tt\{const, C\}} where {\tt C} could be either an atom ({\tt
  pi}) or a number ({\tt 2.34}). Variables could, in the same way, be
represented by tuples {\tt \{var, V\}} where {\tt V} is a atom that
identifies the variable.

If constants and variables are our {\tt literals}, we have the
following definition:

\begin{verbatim}
-type literal() :: {const, number()} | {const, atom()} | {var, atom()}.
\end{verbatim}

\subsection{expressions}

Assume that we, for the time being, limit ourselves to the arithmetic
operations multiplication and addition, we have a very natural
representation. We will of course represent them using tuples where
the first element is an atom that identifies the operation. We have
{\tt\{mul, A, B\}} and {\tt \{add, A, B\}}. Expressions are thus:

\begin{verbatim}
-type expr() :: literal() | {mul, expr(), expr()} | {add, expr(), expr()}
\end{verbatim}


This gives us everything we need to represent a limited sets of
expressions. The expression $2x + 3$ could for example be represented
by the Erlang structure:

\begin{verbatim}
   {add, {mul, {const, 2}, {var, x}}, {const, 3}}
\end{verbatim}

As you see it it is not a syntax we would like to use when we write
expressions by had but it has its advantages when it comes to handle
the expressions using Erlang clauses.

\section{the derivative of}


What are the rules of derivation? You of course remember that derivative
of $2x + 3$ with respect to $x$ is $2$, and that the derivative of
$x^2$ is $2x$ but now we should define a program that does this
automatically so we need to have very clear understanding if the
rules. If you have not done so yet, this is the point where you should
brush up on derivative rules so that you can follow the reasoning.

These are two rules that we will use:

\begin{itemize}
\item $d/dx \: x \equiv 1$
\item $d/dx \: c \equiv 0$  for any literal different from $x$
\item $d/dx \: f(x) + g(x) \equiv  f'(x) + g'(x)$
\item $d/dx \: f(x) \times  g(x) \equiv  f'(x) \times  g(x) + f(x) \times  g'(x)$
\end{itemize}

The third rule is quiet straight forward; the derivative of an sum is
the sum of the derivatives of the terms. The derivative of $4x� + 2x +
5$ is $8x + 2$ since since the derivative of $4x�$ is $8x$ etc. 

The last rule, you might not even have learned as a rule but simply
it's consequences in the most common cases, is the rule that says that
$4x�$ is $8x$ and $2x$ is $2$. When learning using examples like these
the general rule is quite simple: multiply the constant with the power
and reduce the power by one. The general rule gives us the definition
for any product, be it $2x$ or $2x^2$.


\section{the rules of the game}

So if we know how expressions are represented and how to take the
derivative of sums and products, we are ready to implement the
rules. This is a skeleton on what a function {\tt deriv/2} would look like:

\begin{verbatim}
deriv({const, _}, _) ->
    ...;
deriv({var, V}, V) ->
    ...;
deriv({var, Y}, _) ->
    ...;
deriv({add, E1, E2}, V) ->
   ...;
deriv({mul, E1, E2}, V) ->
   ...;
\end{verbatim}

What is the derivative of $2X� + 3x + 5$ with respect to $x$? How do
we represent the expression in our system? Can you calculate the
derivative using the {\tt deriv/2} function?

The answer that you get might not look like the answer you would have
hoped for but it might be that what you see is an expression that can
be simplified. The derivative of $2x$ is of course $2$ but our function
will return something that looks like $0\times x + 2\times 1$, which of course is
equal to $2$.


\section{carrying on}

Add more rules to the {\tt deriv/2} function. We should of course be
able to take the derivative of the following expressions:

\begin{itemize}
\item $x^n$
\item ${\rm ln}(x)$
\item $1/x$
\item $\sqrt{x}$
\item $sin(x)$
\end{itemize}

To be able to handle these expressions you of course need to find a
suitable representation. You then have to find the general rule for
finding the derivative.

\section{simplification}

The results of our derivation might be correct but they are very hard
to read. They contain multiplications with zero, addition with
constant values etc. All of those could be removed if we simplified the results.

Simplification could be tricky, You could start by transforming an
expression so that all functions with constant arguments were actually
evaluated. You could the remove expressions that are multiplied with
zero etc. The problem is to know if there are any more things that
could be done; how do we know that will not be able to do more.

There will also be a discussion of what the simplest form would look like. Should we write $$ x \times  (y + 2)$$ or should we write $$xy + 2x$$



\end{document}
