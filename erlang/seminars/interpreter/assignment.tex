\documentclass[a4paper,11pt]{article}


\usepackage[latin1]{inputenc}

\newcommand{\nnsection}[1]{
\section*{#1}
\addcontentsline{toc}{section}{#1}
}
\usepackage{syntax}

\begin{document}

\begin{center}\vspace{20pt}
\textbf{\large A meta-interpreter}\\
\vspace{10pt}
\textbf{Johan Montelius}\\ 
\vspace{10pt}
\textbf{\today}
\end{center}

\nnsection{Introduction}

In this assignment you will implement an interpreter for a small
functional language. The language could be the functional subset of
Erlang but it does not match exactly how Erlang is defined (but maybe
it should). The exercise will hopefully give you a better
understanding of how a functional programming language is defined and
how powerful a functional programming language can be as tool; the
size of our interpreter is surprisingly small.

\section{The overall picture}

Before staring the implementation you need to get an overall picture
of the goal and what is actually needed. Never start coding one piece
in a jigsaw puzzle if you don't know what the final picture should look
like or how the piece should fit in.

\subsection{a meta-interpreter}

An interpreter, in computer engineering terminology, is a program that
takes a program as input and executes the statements of the program to
produce a result. Some programming systems use only a interpreter to
execute programs but most systems today use a compiler to produce
machine code that is then executed either directly or in a
virtual machine. The advantage of an interpreter is that we do not
have to wait for a compilation phase, the disadvantage is that the
interpreter is much slower (typically a factor ten to one hundred).
Do some reading and find out if the programming languages that you
know uses a compiler or an interpreter.

A meta-interpreter, also called self-interpreter, is an interpreter
that is implemented in the language that it interprets. The reason for
this not obvious but if you have a compiled system and need an
interpreter of the same language, why not implement it in the language
itself - this is probably the language you know best. We will develop
an interpreter for a subset of the Erlang language. Since it is not the
complete language it is not a true meta-interpreter but it's close.


\subsection{expressions}

Our target programming language is a limited functional language and
in its first incarnation it will be limited to simple expressions; We
will not have any function calls nor any program with function
definitions (you might not even call this much of language but it will
be a good starting point). This is an example of a {\em sequence} in
the language:

\begin{figure}[h]
\begin{verbatim}
    X = foo, Y = nil, {Z,_} = {bar,grk}, {X,{Z,Y}}.
\end{verbatim} 
\caption{a simple sequence}
\label{fig:sequence}
\end{figure}
 
A {\em sequence} consist of a, possibly empty, sequence of pattern
matching expression followed by a single expression. This is a bit
more restrictive compared to Erlang sequences but it does not limit
the expressiveness of our language. Simple expressions will for now be
limited to {\em terms}. If we use a BNF grammar we can describe a
sequence in the following way:

\begin{figure}[h]
\begin{grammar}
<sequence> ::= <expression> \alt <match> ',' <sequence>

<match> ::= <pattern> '=' <expression> 

<expression> ::= <atom> \alt <variable> \alt <cons> 

<cons> ::= '\{' <expression> ',' <expression> '\}'
\end{grammar}
\end{figure}

\subsection{terms, data structures and patterns}

Expressions of the form that we have just described are also referred
to as {\em terms}. We will later see examples of expression that are
more complicated and we will talk about expressions in general and
{\em term expressions} but for now we will simply call them {\em terms}.

It is important that we are clear on the terminology when we discuss
our language. There is a difference between {\em terms}, {\em
  patterns} and {\em data structures}. In the sequence previously
shown, the following constructs are terms: \verb+foo+, \verb+nil+,
\verb+{bar,nil}+ and, \verb+{X,{Y,Z}}+. Terms are syntactical
constructs that, {\em when evaluated}, will result in {\em data
  structures}; data structures are thus something that we handle
during execution, the terms are only recipes for how these structures
should be constructed.

If we {\em evaluate} the sequence above we will obtain the data
structure \textit{ \{foo , \{bar , nil\}\}}. We will in this
text use {\it italics} when we refer to the data structures and
\verb+teletype+ when we talk about the terms.

For now, terms are restricted to atoms, variables and the binary
compound {\em cons} structure. Once we understand how to handle this
subset it's fairly easy to extend the language.

The patterns in the sequence are: \verb+X+, \verb+Y+ and,
\verb+{Z,_}+. The syntax we use for patterns, is the same as the one
that we use for terms but we are also allowed to use an underscore
(\verb+_+) to represent a {\em don't-care} variable. This variable
acts as a placeholder for a data structure that we have no interest
in.

\begin{figure}[h]
\begin{grammar}

<pattern> ::= <atom>
\alt <variable>
\alt '_'  
\alt <cons-pattern> 

<cons-pattern> ::= '\{' <pattern> ',' <pattern> '\}'
\end{grammar}
\end{figure}

The reason why we can use the same syntax for terms and patterns
without confusion is that it is clear from the grammatical rules of the
language if we refer to a pattern or a term. 

\subsection{evaluation}

Our interpreter will take a sequence and evaluate the pattern matching expressions one
after the other; the result of the last expression is the result of
the whole sequence. When the evaluation starts the interpreter will
have en empty {\em environment} i.e. it knows of no variable
bindings. Each pattern matching expression will add variable binding
to the environment and the following expressions are then evaluated in
the new environment.
 
If we evaluate the sequence in figure \ref{fig:sequence} we gradually
build an environment, first we add the binding {\em X/foo}, then {\em
  Y/nil} and then {\em Z/bar}. The final expression is thus evaluated
in the following environment: {\em \{X/foo, Y/nil, Z/bar\}}. In this
environment the term \verb+{X,{Z,Y}}+ is evaluated to the data
structure {\em \{foo,\{bar,nil\}\}}.

\subsection{the architecture}

In order to implement our interpreter we need to solve the following
problems: we need to represent expressions and we need to implement an
environment. Once we have these pieces in place we can start to define
the rules for the interpreter.

Before you proceed, you should think this problem through. What do
sequences look like; assume that we will only handle sequences as the
one shown above? What are the elements of a pattern
matching expression, what is on the right side and what is on the left
side? How should we represent a term and does it have to be different
from the representation of a pattern? How should data structures be
represented?

Read this section through one more time, then start to sketch on your
representation. Write it down and then later compare it to the
representation proposed in this exercise. 

\section{The implementation}

We will build the interpreter starting with the environment, that will be implemented in a separate module. 
Then we will handle evaluation of expressions, pattern matching and finally a sequence of expressions. 

\subsection{the environment}

Implementing an environment will be the simplest task that we have; an
environment is simply a mapping from variables to data structures. If
we assume that environments will be small, we can represent an
environment as a list of key-value tuples. The environment {\em
  \{X/foo, Y/bar\}} could be represented as:
\verb+[{'X', foo}, {'Y', bar}]+ . The variables are represented by
atoms, and we have here chosen to name them {\tt 'X'} and {\tt 'Y'}
but we could as well have chosen other atoms ({\tt x}, {\tt
  variable\_x}) or integers ({\tt 1} and {\tt 2}), the important thing
is that they all have unique names.

In a module {\tt env}, you should now define the following functions:

\begin{itemize}
\item {\tt new()} : return an empty environment

\item {\tt add(Id, Str, Env)} : return an environment where the a
  binding of the variable {\tt Id} to the structure {\tt Str} has been
  added to the environment {\tt Env}.

\item {\tt lookup(Id, Env)} : return either {\tt \{Id, Str\}}, if the
  variable {\tt Id} was bound, or {\tt false}

\end{itemize}

These are all operations we need from the environment module. Test the
environment by calling the functions from the Erlang shell to see that
it produces the expected results. The following
call should return {\tt \{foo,  42\}}:

\begin{verbatim}
   env:lookup(foo, env:add(foo, 42, env:new())).
\end{verbatim}

\subsection{terms and patterns}

If we only needed to represent terms consisting of atoms and
cons-cells, things would be trivial. The term {\tt \{a, b\}} could
simply be represented by the Erlang term {\tt \{a, b\}}.  The problem
is that we also need to represent variables. We will of
course not be able to represent a variable in our target language with
a variable in Erlang; we have to find a way to represent variables and
make sure that we can separate them from atoms.

One solution is to represent atoms with the tuple {\tt \{atm, A\}} and
variables with the tuple {\tt \{var, V\}}. The good things is then
that we only have to make sure that the identifiers of variables are
all different and that the identifiers of atoms are all different. An
atom could of course have the same identifier as a variable with out
causing any problems; the atom {\tt \{atm, 123\}} is different from
the variable {\tt \{var, 123\}}.

A cons cell could be represented by a tuple {\tt \{cons, Head,
  Tail\}}. We could of course have chosen to represent cons cells as
Erlang cons cells but we want to make a distinction between the
representation of terms in our target language and terms in Erlang (our
target language now happens to be Erlang look-alike but that is of
course not always the situation).

As an exercise you can write down the representation of the term:

\begin{verbatim}
     {a, {X, b}}
\end{verbatim}

You will have to choose identifiers for the atoms {\tt a}, {\tt b}
(why not {\tt a}, {\tt b}) and the variable {\tt X} (why
not {\tt x}).

The representation of patterns will be exactly the same as for terms
with the only difference that we need to represent the special {\em
  don't-care} pattern. We choose the atom {\tt ignore} which will be
exactly what we will do when we encounter the symbol.

\subsection{expressions}

In our simple language an expression is simply a term
expression. Expressions should be evaluated to data structures and the
question is of course how these data structures should be represented.

The nice thing with a meta-interpreter is that the data structures of
the interpreted language could be mapped directly to the data
structures of the implementation language. An atom will thus be
represented by an Erlang atom, and a cons structure of the Erlang tuple
 i.e. {\tt \{a, b\}}. 

Create a module called {\tt eager} (for reasons that will be given
later) and implement a function {\tt eval\_expr/2} that takes an
expression and an environment and returns either {\tt \{ok, S\}},
where {\tt S} is a data structure, or {\tt error}. An error is
returned if the expression can not be evaluated. This should be a
quite simple task. The following skeleton code will get you started:

\begin{verbatim}
eval_expr({atm, Id}, _) ->
    {ok, Id};
eval_expr({var, Id}, Env) ->
    case ... of
        false ->
            error;
        ... ->
            {ok, ..}
    end;
eval_expr({cons, ..., ...}, Env) ->
    case ... of
        error ->
            ...;
        {ok, ...} ->
            case ... of
                error ->
                    error;
                {ok, ...} -> 
                    {ok, {...,...}}
            end
    end.
\end{verbatim}

Here are some examples that you should be able to handle.

\begin{itemize}
 \item {\tt eval\_expr(\{atm, a\}, [])} : returns {\tt \{ok, a\}}
 \item {\tt eval\_expr(\{var, x\},  [\{x, a\}])} : returns {\tt \{ok,  a\}}
 \item {\tt eval\_expr(\{var, x\},  [])} : returns {\tt error}
 \item {\tt eval\_expr(\{cons, \{var, x\}, \{atm, b\}\},  [\{x, a\}])} : returns {\tt \{a, b\}} 
\end{itemize}


Note that {\tt eval\_expr/2} returns a data structure if successful
i.e. {\em a}, {\em foo}, or {\em \{a, b\}}. When evaluating a
pattern matching expressions we will first evaluate the right hand
side, resulting in a data structure, and then perform the matching
procedure. The last test is an effect of representing the binary
tuples in our source program {\tt \{a, b\}}, with the internal
representation {\tt \{cons, \{atom, a\}, \{atom, b\}\}} that when
evaluated will return the datastructure {\it \{a,b\}}.

\subsection{pattern matching}

A pattern matching will take a pattern, a data structure and an
environment and return either {\tt \{ok, Env\}}, where {\tt Env} is an
extended environment, or the atom {\tt fail}. 

Implement a function {\tt eval\_match/3} that implements
the pattern matching. Some examples will give you an idea of what
we're looking for.

\begin{itemize}
 \item {\tt eval\_match(\{atm, a\}, a, [])} : returns {\tt \{ok, []\}}
 \item {\tt eval\_match(\{var, x\}, a, [])} : returns {\tt \{ok, [\{x, a\}]\}}
 \item {\tt eval\_match(\{var, x\}, a, [\{x, a\}])} : returns {\tt \{ok, [\{x, a\}]\}}
 \item {\tt eval\_match(\{var, x\}, a, [\{x, b\}])} : returns {\tt fail}
\end{itemize}

Solving the cases where the pattern is an atom or variable is quite
straight forward, especially since we already have the environment
module. The slightly more problematic case is when the pattern is a
cons structure. The following skeleton code should lead you in the
right direction.

\begin{verbatim}
eval_match(ignore, ..., ...) ->
    {ok, ...};
eval_match({atm, Id}, .., ...) ->    
    {ok, ...};
eval_match({var, Id}, ..., ...) -> 
    case ... of
        false ->
            {ok, ...};
        {Id, Str} ->
            {ok, ...};
        {Id, _} ->
            fail
    end;
\end{verbatim}
\begin{verbatim}
eval_match({cons, ..., ...}, ..., Env) -> 
    case eval_match(..., ..., ...) of
        fail ->
            fail;
        {ok, ...} ->
            eval_match(..., ..., ...)
    end;
eval_match(_, _, _) -> 
    fail.
\end{verbatim}

Complete the implementation and try the following calls:

\begin{itemize}
 \item {\tt eval\_match(\{var, x\}, a, [])}
 \item {\tt eval\_match(\{cons, \{var, x\}, \{var, x\}\}, \{a,a\}, [])}
 \item {\tt eval\_match(\{cons, \{var, x\}, \{var, x\}\}, \{a,b\}, [])}
\end{itemize}

\subsection{sequences}

We now have all the pieces of the puzzle to implement the evaluation
of a sequence. We represent a sequence as list, the first elements
will of course be pattern matching expressions but the last element is
of course a regular expression. The evaluation starts with an empty
environment that is extended as we proceed down the list.

Each pattern matching expressions is evaluated in two steps, first the
expression on the right hand side is evaluated returning a data
structure. The pattern on the left hand side is then match to the data
structure resulting in an extended environment. Here is some skeleton
code that Will get you started.

\begin{verbatim}
eval_seq([Exp], Env) ->
    ...;
eval_seq([{match, Ptr, Exp}|Seq], Env) ->
    case ... of
        error -> 
            ...;
        {ok, Str}  ->
            case ... of
                fail ->
                    error;
                {ok, ...} ->
                    ...
            end
    end.
\end{verbatim}

When you have everything in place you should define a function {\tt
  eval/1}, that takes a sequence and returns either {\tt \{ok, Str\}}
or {\tt error}. You should then be able to run the following query:

\begin{verbatim}
 Seq = [{match, {var, x}, {atm,a}},
        {match, {var, y}, {cons, {var, x}, {atm, b}}},
        {match, {cons, ignore, {var, z}}, {var, y}},
        {var, z}],
 eval(Seq).
\end{verbatim}

The query is the representation of the following expression:

\begin{verbatim}
   X = a, Y = {X,b}, {_,Z} = Y, Z.
\end{verbatim}


\section{The assignment}

We now have an interpreter that can handle sequences of expressions
but the expressions are rather simple. You should now extend the
language and the interpreter to handle {\em case expressions}. You
should describe the complete language using BNF syntax as has been
done in the introduction above. You should then explain how case
expressions are represented ad how the interpreter should be extended to
handle them.

\subsection{structure of report}

The report should be self contained i.e. it should include the whole
language, how all expressions and data structures are represented and
how the interpreter is implemented. It should not be a description, not a tutorial like
this text, and the target should be a fellow student that knows as
much Erlang as you do but does not know what a meta-interpreter is
nor how a limited functional programming language could be defined.

\subsection{the interpreter}

Think things through before starting to program. The important steps
are first of all what the language should look like and then how the
case expressions should be represented. Identify the elements of a
case expression; an expression to be evaluated and a sequence of {\em
  clauses} - how would you represent clauses? Once you have the
representation the implementation will be quite simple, only adding
21 lines of code. The only problem you might face is that {\tt case}
is a reserved word and you will have to use another atom ({\tt
  switch}) in the representation.

Give an example of a sequence that you now can interpret that includes
a case expression.

\subsection{\LaTeX{} support}

When describing the BNF grammar it is very convenient to use the \LaTeX
package {\tt syntax}. Include \verb+\usepackage{syntax}+ in the
beginning of your document and then use the following code to describe
the grammar:

\begin{verbatim}
\begin{grammar}
<expression> ::= <atom>
\alt <variable>
\alt <cons> 

<cons> ::= '\{' <expression> ',' <expression> '\}'
\end{grammar}
\end{verbatim}

Code is included using the {\tt verbatim} environment, but make sure
that you do not have any tab characters in the code (these will only
be interpreted as a single space character and the indentation will be
wrong). There are packages, for example the {\tt listings} package,
that does a better job in type setting code but {\tt verbatim} will do
fine.

\end{document}
