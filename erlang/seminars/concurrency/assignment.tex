\documentclass[a4paper,11pt]{article}


\usepackage[latin1]{inputenc}

\usepackage{amsmath}


\newcommand{\nnsection}[1]{
\section*{#1}
\addcontentsline{toc}{section}{#1}
}

\begin{document}

\begin{center}
\vspace{20pt}
\textbf{\large Erlang Concurrency}\\
\vspace{10pt}
\textbf{Johan Montelius}\\
\vspace{10pt}
\today{}
\end{center}


\nnsection{Concurrency}

Erlang was designed for concurrent programming. You will quickly learn
how to divide your program into communicating processes and thereby
give it far better structure. Try the following:

\begin{verbatim}
-module(wait).
-export([hello/0]).

hello() ->
    receive
      X -> io:format("aaa! surprise, a message: ~s~n", [X])
    end.
\end{verbatim}

\noindent The {\tt io:format} procedure will output the string to the stdout and
replace the control characters (characters preceded by a tilde) with
the elements in the list. The {\tt ~s} means that the next element in
the list should be a string, {\tt ~n} simply outputs a newline. Load
the above module and execute the command:

\begin{verbatim}
   P = spawn(wait, hello, []).
\end{verbatim}

\noindent The variable {\tt P} is now bound to the {\em process identifier} of
spawned process. The process was created and called the procedure {\tt
  hello/0} (this is how we name a function with zero arguments). It is
now suspended waiting for incoming messages. In the same Erlang shell
execute the command:

\begin{verbatim}
   P ! "hello".
\end{verbatim}

\noindent This will send a message, in this case a string ``hello'', to the
process that now wakes up and continues the execution.

To make life easier one often register the process identifiers under
names that can be access by all processes. If two processes should
communicate they must know the process identifier. Either a process is
given the identifier to the other process when it is created, in a
message or, through the registered name of the process. Try this in a
shell (first type {\tt f()} to make the shell forget the previous
binding to P):

\begin{verbatim}
   P = spawn(wait, hello, []).
\end{verbatim}

\noindent Now register the process identifier under the name ``foo''.

\begin{verbatim}
   register(foo, P).
\end{verbatim}

\noindent And then send the process a message.

\begin{verbatim}
   foo ! "hello".
\end{verbatim}


\nnsection{Tic-Tac-Toe}

\noindent In the example above the only thing we sent was a string but
we can send arbitrary complex data structures. The {\tt receive}
statement can have several clauses that try to match incoming
messages. Only if a match is found will a clause be used. Try this:

\begin{verbatim}
-module(tic).
-export([first/0]).

first() ->
    receive
      {tic, X} -> 
          io:format("tic: ~w~n", [X]),
          second()
    end.

second() ->
    receive
      {tac, X} -> 
          io:format("tac: ~w~n", [X]),
          last();
      {toe, X} -> 
          io:format("toe: ~w~n", [X]),
          last()
    end.

last() ->
    receive 
      X ->
          io:format("end: ~w~n", [X])
    end.
\end{verbatim}


\noindent Then in a shell execute the following commands:

\begin{verbatim}

P = spawn(tic, first, []).

P ! {toe, bar}.

P ! {tac, gurka}.

P ! {tic, foo}.

\end{verbatim}

\noindent In what order where they received by the process. Note how messages
are queued and how the process selects in what order to process them.

\end{document}
