\documentclass[a4paper,11pt]{article}

\usepackage[utf8]{inputenc}
\usepackage[T1]{fontenc}

\newcommand{\nnsection}[1]{
\section*{#1}
\addcontentsline{toc}{section}{#1}
}

\begin{document}

\begin{center}
\vspace{20pt}
\textbf{\large Mutual exclusion - locks, semaphores and monitors}\\
\vspace{10pt}
\textbf{Johan Montelius}\\
\vspace{10pt}
\today{}
\end{center}


\nnsection{Getting started}

In this assignment you will learn about the concept of {\em mutual
  exclusion} and how locks, semaphores and monitors can give us what
we want. These concepts are not frequently used in Erlang programming
but you should know about them and understand why they are not often
explicity used in Erlang.

The main idea with mutual exclusion is that we wan to limit the
concurrency to at most one executing process in a {\em critical
  section}. A critical section could be a section in the program where
we modify some data structure and we do not want any other process to
see what we have done until we are completely done. Since we do not
have any updatable data structures in Erlang the need for locks is
limited but think about updating a set of files where you want to do
all the changes before you let another process see what you have done
or do their modifications.

\section{Let's implement a lock}

We will start by implement a {\em lock process} (or rather try to
implement it). A lock is something that can only be held by one
process, the process that {\em takes the lock} knows that it is the
sole owner of the lock and can proceed to a critical section.



Our first attempt to implement a lock process is quite straight forward
- we will implement the lock as a process that only holds one value
and accepts two messages: set and get.

\begin{verbatim}
-module(cell).
-export([new/0]).

new() -> spawn_link(fun() -> cell(open) end).

cell(State) ->
  receive
     {get, From} -> 
         From ! {ok, State}, 
         cell(State);
     {set, Value, From} -> 
         From ! ok, 
         cell(Value)
  end.
\end{verbatim}

To make it easier to use the lock we also provide two functions that
hide the fact that we do asynchronous communication.

\begin{verbatim}
get(Cell) ->
   Cell ! {get, self()},
   receive 
       {ok, Value} -> Value
   end.   

set(Cell, Value) ->
   Cell ! {set, Value, self()},
   receive 
       ok -> ok
   end.
\end{verbatim}

If we have created a cell we could use it to protect a critical
operation using the codes that follows (not true so don's stop reading
here).

\begin{verbatim}
do_it(Thing, Lock) ->
   case cell:get(Lock) of
     taken ->
        do_it(Thing, Lock);
     open -> 
        cell:set(Lock, taken),
        do_ya_critical_ting(Thning),
        cell:set(Lock, open)
   end.
\end{verbatim}

Perfect, case closed ... ehhh, there is something wrong - what happens
if...? Before you continue think about what would happen if two
processes called the {\tt do\_it/2} procedure with the same lock; do we
guarantee that the two processes will never ever execute the critical
section at the same time?

\section{the atomic swap}

There are two solutions to the problem of the lock in the first
section: {\em atomic swap} and {\em Peterson's algorithm}. Using atomic
swap we implement a new message that will read and write to the cell
in the same operation. This feature is often found in hardware and
programs written in C or C++ can often make direct use of this. In our
implementation we will implement it ourselves with a small extension to the cell. 

\begin{verbatim}
cell(State) ->
  receive
     {swap, Value, From} -> 
         From ! {ok, State},
         cell(Value);
     {set, Value, From} -> 
         From ! ok, 
         cell(Value)
  end.
\end{verbatim}

Assuming we also provide a functional interface we could now use the lock as follows.

\begin{verbatim}
do_it(Thing, Lock) ->
   case cell:swap(Lock, taken) of
     taken ->
        do_it(Thing, Lock);
     open -> 
        do_ya_critical_ting(Thing),
        cell:set(Lock, open)
   end.
\end{verbatim}

In this version it does not matter if two processes calls the
procedure at the same time; both of them will swap in the value {\tt
  taken} but only one of them would receive the {\tt open} value in
return. The process that looses the race will have to retry to take
the lock and will succeed once the holding process sets the lock to
{\tt open}.

\section{Peterson's algorithm}

You might wonder if there is a way to implement a lock without an
atomic swap operations. If not, we are sure lucky that the hardware
people have implemented it. It turns out that there is and the
algorithm is fairly simple once you understand why it works.

\subsection{the algorithm}

Assume we have our original cell with only the {\tt get} and {\tt set}
operations. We also assume that there are only two processes that will
compete for the lock. We will now use three cells: {\tt P1}, {\tt P2}
and {\tt Q}. In the cell {\tt P1} the first process will declare its
interest in moving into the critical section. The second process will
declare its interest using {\tt P2}. The {\tt Q} cell will be used to
determine the winner if we have a draw.


The two processes will execute slightly different code when trying to
enter the critical section; or rather, the code is the same but the
parameters are shifted. The first process will call the procedure {\tt
  lock(0,P1,P2,Q)} where as the second process will call {\tt lock(1, P2, P1,
  Q)}.

\begin{verbatim}
lock(Id, M, P, Q) ->
   cell:set(M, true),
   Other = (Id+1) rem 2,
   cell:set(Q, Other), 
   case cell:get(P) of
      false ->
         locked;
      true ->
         case cell:get(Q) of
            Id -> 
               locked;
            Other ->
                lock(Id, M, P, Q)
          end
   end
end.

unlock(_Id, M, _P, _Q) ->
   cell:set(M, false).
\end{verbatim}

The intuition is that each process begins to declare that they are
interested in taking the lock. They then set the common cell {\tt
  Q} to the second process's identifier as a signal to the second
process to go ahead if they are both interested of the lock. If a
process is however sees that the second process is not interested it
holds the lock and proceed into the critical section.

\subsection{prove it}

To understand how Peterson's algorithm works is not easy; to prove
that it ensures that the two processes do not believed to hold the
lock at the same time is more difficult. 
 
If you want to prove that Peterson's algorithm works, you can draw a
finite state machine diagram with the state of the variables: {\em
  p1}, {\em p2}, {\em q} and two variable {\em l1} and {\em l2} that
describes if a process is in the critical sector. One alternative
method is to use so-called {\em temporal logic} where the rules can
prove that it is never so that {\em l1} and {\ em l2} are both true.

An interesting question is whether one can make use of Peterson's
algorithm in a computer that has as much as possible in the cache, or
in a distributed system where clients have local copies. A
prerequisite for the algorithm to work is that if a process sets {\em
  p1} to {\em true} and then reads that {\em p2} is {\em false} then
it can not be that the second process manages to set {\em p2} to {\em
  true} and then read that {\em p1} is {\em false}. Describe a
scenario using cached local copies that will violate this prerequisite. 

\subsection{the bakery algorithm}

When I lived in Barcelona I learned a wonderful algorithm for keeping
track of who is next to be served in a bakery. When you entered a
bakery (or butchery) you simply greeted every one with the phrase
``¿Buenas diaz, ultimo?''. The person who was the last person in line
would reply ``Si'' and then you would know who was the person just in
front of you. If someone else entered the store you would be the one
to reply ``Si'' and that was it. The system works perfectly and you
avoid the hassle of finding a machine to give you a ticket. If Leslie
Lamport had lived in Barcelona he would probably never have named his
algorithm {\em the bakery algorithm} since it is based on a numbering
system where entering processes picks a number that is higher than any
other number in the queue.

The algorithms uses a shared array with one index per process. If the
index of a process is set to $0$
it means that the process is not interested in entering the
critical section. When a process wishes to enter the critical section it
will scan the array and find the highest ticket number and then set
its own index to the number plus one. The intuition is that all
processes that entered the store before it should have precedence. It
could of course happen that two processes enters at the same time and
chooses an identical ticket but this is solved by giving precedence to
the process with the lowest id.

When a process has selected a ticket number it will again scan the
array from the beginning and wait until all indexes before its own, are
either set to $0$
or have a ticket number that is higher than its own and, all indexes
after its own are set to $0$
or have ticket numbers higher or equal to its own. When the process
has scanned the array it is allowed into the critical section and will
when it is done set its own index to $0$.

The scanning of the array can proceed one step at a time, if a index
has the value $0$
it could of course be set by the owner of the index but then it will be
set to a value equal or higher to the ticket of the scanning
process. An index that holds a value that is lower than then ticket of
the scanner will eventually be set to $0$
once the processed has completed its critical section. 

\section{why not in Erlang}

Atomic swap, Peterson's or the bakery algorithm are hardly ever used
in a Erlang program, nor in any higher level program. The reason is
that the mutual exclusion problem is solved by language constructs that
hides the details of how things are implemented. The first higher level construct that we will take a look at is {\em the semaphore}.

\section{The semaphore}

In the previous section we implemented the locks using a {\em busy
  waiting} or {\em spin-lock} strategy. A process that would not
immediately require the lock would try and try again until the lock
was acquired. This is a very aggressive strategy that in the worst case
means that a process will spend a lot of resources just reading from a
memory location, or as in our case send thousands of messages to a
cell process and request its state.

A better strategy (not always better) could be to suspend the
execution if the lock is not taken and only continue to execute once
the lock has been acquired. The semaphore concept is also often
described as being more general compared to a binary lock. We will
describe a semaphore that will allow at most {\em n} processes to
enter the critical section. If {\em n} is $1$
then it called a {\em binary semaphore}.


\subsection{was this it}

In Erlang this is expressed so easily that you hardly realize that you
have implemented a semaphore. Look at the code below:

\begin{verbatim}
semaphore(0) ->
   receive 
     release ->
        semaphore(1)
   end;
semaphore(N) ->
   receive 
      {request, From} ->
         From ! granted,
         semaphore(N-1);
      release ->
         semaphore(N+1)
   end.
\end{verbatim}
      
If we create a semaphore with the initial value $4$
then at most four processes will be granted access to the critical
section. The code for requesting entrance would of course look like follows:

\begin{verbatim}
request(Semaphore) ->
   Semaphore ! {request, self()},
   receive 
      granted -> 
        ok;
   end.
\end{verbatim}

The difference from the locks we implemented before is that the
requesting process will now be suspended waiting in a receive
statement until the semaphore responds with a {\tt granted} message. 

\subsection{I've heard it was tricky}

In the classical description of a semaphore one must explain what
happens when a process request access and there are no resources
left. One will then describe how this process is added to a queue of
waiting processes and how it then yields the execution. When a process
leaves the critical section, the first process in the queue of
suspended processes will be selected and added to the set of runnable
processes (often by sending it a signal to wake up).

In our Erlang implementation all this is hidden in the message
queue. A process that sends a {\tt request} message will of course
have its message inserted as the last message in the message queue. If
there are no resources left (the first clause), then request messages
will simply not be handled. Only when resources are available will the
semaphore handle requests and then it will of course handle them in the
order they have arrived in the message queue.

As an exercise you can re-write the semaphore so that it only has one
clause and always accepts requests. If there are no resources
available the requesting process must be held on hold in a list of
waiting processes. When a release message is received and the resource
is incremented from zero the first process, if any, in the lists of
waiting processes should be granted access. 

\section{The monitor}

The semaphore gave us a solution to the mutual exclusion problem but
is does of course require that the processes do respect the rules. No
process is allowed to enter the critical section if it has not being
granted access by the semaphore. It must also release its access when
it leaves the critical section. A process that not play by the rules
could of course ruin the whole system.

A better strategy is to encapsulate the critical section inside a
semaphore so that no process can enter the critical section without
using the semaphore. This concept is called {\em a monitor} and is how
things are done in Erlang as well as Java. In Java one would declare a
method of an object to be {\em synchronized} thereby preventing more
than one thread at a time to execute the method. In Erlang the same
thing is of course handled by messages. 

\begin{verbatim}
monitor(State) ->
   receive 
     {request,  From} ->
         Updated = critical(State), 
         From ! ok,
         monitor(Updated)
   end.
\end{verbatim}

The Actors model, that Erlang is built on, automatically gives us the
properties of a monitor. From the implementation of {\tt monitor/1}
we easily see that request to execute the critical section are of
course handled one by one and will even be done in a fair order.

You could easily implement more advanced constructs where a request
could contain a lambda expression that should be applied to the
monitor state. In this way the requesting process has more freedom to
control what should be done.

\begin{verbatim}
monitor(State) ->
   receive 
     {request, Fun, From} ->
         Updated = Fun(State), 
         From ! ok,
         monitor(Updated)
   end.
\end{verbatim}

In one way, you can view every Erlang process as a monitor that
protects its state. It will only handle one message at a time and it
is the only process that has access to the state.

\section{Deadlocks}

The locks, semaphores and monitors solves the problem of data
corruption i.e. by only allowing one transformation at a time. As you
have seen the Actors model gives us this almost for free but this is
only half of the problem introduced by concurrency. A equally important
problem is the problem of {\em deadlock} i.e. a situation were no
process can proceed since they are all waiting for someone else to
take the first step.

\subsection{shades of hell}

A complete deadlock is of course the worst thing that could happen but
there are other related problems that are almost as bad:

\begin{itemize}
 \item deadlock: Nothing moves.
 \item livelock: Things move but we're not making progress.
 \item starvation: We make progress but at least one process is prevented from progressing.
 \item non-fair scheduling: All processes make progress but some processes do not get a fair share of the resources. 
\end{itemize}

It is not necessarily so that every system you implement must
implement a fair scheduling algorithm that guarantees that all
processes should have a equal chance in acquiring the resources of the
system. It could be enough that it is starvation free or proved to
never go into a livelock. Its important to understand what is required
and then choose algorithms to meet these requirements. If you always
go for an implementation that guarantees fair scheduling then you
might pay more than anyone asked for.

\subsection{locks, semaphores and monitors}

Go back to the locks, semaphores and monitors in the previous section
and ask yourself what the implemented solutions actually provided. They
were hopefully dead lock free but were they starvation free? Did they
provide a fair scheduling i.e. if a process has requested a lock or
access to a critical section, will the process be granted access
before any process that issues a request at a later moment?


\subsection{detecting a deadlock}

Even if the locks, semaphores and monitors that we have discussed have
algorithms that will not deadlock, we can easily create a deadlock if
we have two or more locks. If process {\em P1} is granted a lock {\em
  A}, process {\em P2} is granted a lock {\em B} and requests lock
{\em A} it will of course have to wait. If process {\em P1} now
requests lock {\em B} we have a circular dependency that has caused a
deadlock.

We can create a similar circular dependency with monitors if one
monitor will, as part of its critical section, request the service of
another monitor. This monitor might in turn request a third resource
that request the service of the first monitor.

One way to break the deadlock is to give up waiting for a lock,
release some of the locks held, do something else for a while and then
retry to take the locks. We can implement this using a timeout in the
receive statement. The implementation will however be more complicated
than you might first think.

\subsection{master should be resting}

Assume that we have implemented a semaphore and use the following
function to acquire access.

\begin{verbatim}
request(Semaphore) ->
  Semaphore ! {request, self()},
  receive 
    granted ->
       ok
  after 1000 ->
       abort
  end.    
\end{verbatim}

A process that calls the procedure {\tt request/1} will then be given
the result {\tt ok} or {\tt aborted}. If everything went fin it will
continue to execute but if it receives {\tt aborted} it knows that we
could in the worst case be in a deadlock. If it is not holding any
other lock it is not much it can do but if it holds another request a
good strategy could be to release this resource and then ponder $\pi$
for a while.

\begin{quotation}
This all sounds fine but the above explanation could have been given
by Gollum.
\end{quotation}

If you implement it like this your in for a surprise. The message that
you sent to the semaphore is not lost but simply waiting in the
message queue of the semaphore. Sooner or later the semaphore will
handle this request and send a {\tt granted} message to you. Now you
have the resource even if you don't realize it. If you at a later point
in time return to the semaphore and again request the resource you
will find this granted message that has been there all the time. 

This means that you can not just walk away from a request. The
interesting thing is that you can return a resource before you got it
- look at this:

\begin{verbatim}
request(Semaphore) ->
   Ref = make_ref(),
   Semaphore ! {request, Ref, self()},
   wait(Semaphore, Ref).

wait(Semaphore, Ref) ->
  receive 
    {granted, Ref} ->
       ok;
    {granted, _} ->
       wait(Semaphore, Ref)
  after 1000 ->
       Semaphore ! release
       abort
  end.
\end{verbatim}

We should of course also change the implementation of the semaphore so
that it accepts requests on the form {\tt \{request, Ref, From\}} and
reply with a {\tt \{granted, Ref\}} but then we are on the safe side.

The reason we can send a {\tt release} before we actually have been
granted the request is that we know that our request is in the message
queue. We also make sure, by using the unique references, that we will
not mistake an old granted message as a reply to a new request. 

Better than trying to resolve a deadlock situation is of course to
avoid it all together and there is a simple strategy that will always work. 


\subsection{avoiding deadlock}

The problem with a deadlock is of course that we have a circular
structure where everyone is waiting for someone else. If we can avoid
building a circular structure we could avoid deadlocks all together.

Assume all resources (locks, semaphores or monitors) are ordered and
you're never allowed to take higher resource before a lower
resource. If you find that you're waiting for a resource the resource
is of course held by someone. This someone is either working, in which
case everything is fine and you will be given access sooner or later,
or suspended waiting for a higher resource. It can not be suspended
waiting for a lower resource since it is not allowed to request a
lower resource if it is holding the resource that you're waiting for.

You might ask, what happens if the process is suspended waiting for a
higher resource but then it is in the same situation as you are. We
have a chain of processes, all waiting. Since we only have a finite
set of resources the process in the end of the chain is not suspended
but working. Sooner or later it will let go of its resources and allow
the next process in line to continue its execution.

The only problem with this strategy is that it sometimes is hard to
order the resources so that everyone knows the order. Nor is it always
easy to determine beforehand which resources that will be needed - if
you start by grabbing a resource that you know you need you're not
allowed to grab a resource with a lower rank.


\section{Summary}

The Actors model of handling concurrency saves us from most of the
problems of protecting critical sections. The process is in a sense a
monitor that protects a critical section and the message handling will
give all processes fair access to the resource.

Deadlock is a problem but can often be avoided by ordering the
resources and always request the resources in this order. Detecting a
deadlock situation and resolve the situation might be trickier than
think. 


\end{document}

