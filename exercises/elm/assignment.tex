\documentclass[a4paper,11pt]{article}

\input ../include/preamble.tex


\begin{document}


\title{Elm - an event simulator}

\author{Johan Montelius}
\date{Spring Term 2024}

\maketitle

\defaultpagestyle

\section*{Introduction}

This is a small tutorial in how to use Elm and how to build a small
client side web application. Since we have to build someting we might
as well build something fun so why not a particle simulator.

Why Elm you might wonder and the reason is that Elm is functional
programming language and it's fun to see how much you understand now
that you know some Elixir. You have learnt not only a programming
language but a programming paradigm. It's also fun to see how a
functional programming language can be used to build a fron-end web
application. As you hopefully will see, the functional paradigm makes
it easy to divide you system into: model, view and control. 

\section*{Hello world}

To start working you need to install the Elm frame work. Everything
you need is found at {\tt elm-lang.org}. When you're done you should
have program called {\tt elm} installed; this is the program that will
set up your project and compile your Elm code. Elm is compiled into
javascript so in the end we will have a html file and a javascript
file that we can upload to a web server.


Start by creating a directory called {\tt hello}, open a terminal
navigate to the directory. In the director type {\tt elm init}, this
should ask you a question (to wich you answer {\tt Y}) and you can
then read more at a URL proposed by Elm.

\begin{verbatim}
> elm init 
Hello! Elm projects always start with an elm.json file. I can create them!

Now you may be wondering, what will be in this file? How do I add Elm files to
my project? How do I see it in the browser? How will my code grow? Do I need
more directories? What about tests? Etc.

Check out <https://elm-lang.org/0.19.1/init> for all the answers!

Knowing all that, would you like me to create an elm.json file now? [Y/n]: Y
Okay, I created it. Now read that link!
\end{verbatim}

You now open a web browser and navigate to {\tt
  elm-lang.org/examples/buttons} and find a program that you copy and
save as the file {\tt Main.elm} in the {\tt src} directory. The next
step is where magic takes place, in the terminal you now execute the
command {\tt elm reactor}.

\begin{verbatim}
> elm reactor
Go to http://localhost:8000 to see your project dashboard.
\end{verbatim}

If you open a browser and enter the URL you will see your first Elm
project running. Navigate to {\tt src/Main.elm} and the program
starts.

Running your program using {\tt elm reactor} is very convinient during
development but once you have something that works you of course want
to deploy it on a regular web server. To turn everything into a stand alone web page you use the command {\tt elm make src/Main.elm}.

\begin{verbatim}
> elm make src/Main.elm
Success!

    Main ---> index.html
\end{verbatim}

You can open the file {\tt index.html} on your local browser using the
URL {\tt file::///..../index.html} (you have to locate where it is) or
rename it and copy it to your KTH {\tt public_html} directory and then
access over the web. 


\section*{The simulator}

Let's start a new project, a simulator. 





\end{document}
