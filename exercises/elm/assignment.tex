\documentclass[a4paper,11pt]{article}

\input ../include/preamble.tex


\begin{document}


\title{Elm - an event simulator}
}
\author{Johan Montelius}
\date{Spring Term 2024}

\maketitle

\defaultpagestyle

\section*{Introduction}

This is a small tutorial in how to use Elm and how to build a small
client side web application. Since we have to build someting we might
as well build something fun so why not a particle simulator.

Why Elm you might wonder and the reason is that Elm is functional
programming language and it's fun to see how much you understand now
that you know some Elixir. You have learnt not only a programming
language but a programming paradigm. It's also fun to see how a
functional programming language can be used to build a fron-end web
application. As you hopefully will see, the functional paradigm makes
it easy to divide you system into: model, view and control. 

\section*{Hello world}

To start working you need to install the Elm frame work. Everything
you need is found at {\tt elm-lang.org}. When you're done you should
have program called {\tt elm} installed; this is the program that will
set up your project and compile your Elm code. Elm is compiled into
javascript so in the end we will have a html file and a javascript
file that we can upload to a web server.


Start by creating a directory called {\tt hello}, open a terminal
navigate to the directory. In the director type {\tt elm init}, this
should ask you a question (to wich you answer {\tt Y}) and you can then read
more at a url proposed by Elm (for axample {\tt
  https://elm-lang.org/0.19.1/init}). In short file called {\tt
  elm.js} and a director {\tt src} has been created.

You now open a web browser and navigate to {\tt
  elm-lang.org/examples/buttons} and find a program that you copy and
save as the file {\tt Main.elm} in the {\tt src} directory. When this
is done you go to the terminal and write: {\tt elm make src/Main.j}
(and of course if you use Windows it's something different).

You should now have a new file in you {\tt hello} directory; {\tt
  index.html}, try to open it in you regular browser. If you can
increment and decrement the counter you have your first Elm program up
and running. Let's look at the file {\tt Main.elm} to see what i
contains.

\section*{The simulator}



\end{document}
