\documentclass[a4paper,11pt]{article}

\input ../include/preamble.tex

\usepackage{tikz}
\usetikzlibrary{automata,arrows,topaths,calc,positioning}

\usepackage{subcaption}

\usepackage{changepage}

\usepackage{listings}

\lstdefinelanguage{elixir}{
	morekeywords={case,catch,def,do,else,false,%
		use,alias,receive,timeout,defmacro,defp,%
		for,if,import,defmodule,defprotocol,%
		nil,defmacrop,defoverridable,defimpl,%
		super,fn,raise,true,try,end,with,%
		unless},
	otherkeywords={<-,->},
	sensitive=true,
	morecomment=[l]{\#},
	morecomment=[n]{/*}{*/},
	morestring=[b]",
	morestring=[b]',
	morestring=[b]"""
}

\lstset{language=elixir}

\begin{document}

\title{
  \textbf{An emulator}\\
  \large{Programming II}
}
\author{Johan Montelius}
\date{Spring Term 2021}
\maketitle
\defaultpagestyle

\section*{Introduction}

In this exercise we will combine your knowledge of assembly
programming and functional programming to implement an emulator for a
subset of MIPS assembler. You should have a basic understanding of
assembly programming but we will not do very advanced programming so
no need to know all instructions by heart. The idéa here is not to
imlpement a complete emulator but to see how we can implement it in
Elixir.


\section{MIPS assembler}

We will only implement a small subset of the MIPS instruction set. If
we can emulate one arithmetic operation I think we could easily extend
the emulator to handle all. The subset is however choosen to cover the
various forms of instructions: arithmetic, load and store, branching
etc. We will also inlcude some non-mips insructions that might be fun
to have when we run our programs.


These are the instructions that we will implement:

\begin{itemize}
\item {\tt add \$d \$s \$t} : add the values of register {\tt \$s} and {\tt \$t} and place result in {\tt \$d}.
\item {\tt  sub \$d \$s \$t} : subtract the values of register {\tt \$s} and {\tt \$t} and place result in {\tt \$d}.
\item {\tt addi \$d \$t imm} : add the values of register {\tt \$t} and the immediat value {\tt imm} and place result in {\tt \$d}.
\item {\tt lw \$d offset(\$t)} : load the value found at adress {\tt offset + \$t} and place it in {\tt \$d}.
\item {\tt sw \$s offset(\$t)} : store the value in register {\tt \$s} at adress {\tt offset + \$t}. 
\item {\tt beq \$s \$t offset} : branch to {\tt pc + offset} if values at {\tt \$s} and  {\tt \$t} are equal.
\end{itemize}


We will also implement the two following instrucktions that do not
have any corresponding machine instructions but are very handy for our
implementation.

\begin{itemize}
\item {\tt halt} : halt the execution (normally implemented as an enless loop but we will actually stop)
\item {\tt out \$s} : output value at register location {\tt \$s}.
\end{itemize}

When you normally write assembly programs you of course use an editor
nd write you programs in a text file. A program could for example
look like this:


\begin{verbatim}
	.text 	
main:
	li $t2, 42		# load 42 into reg $t2 
	lw $t3, value		# load the word stored at address value into reg $t3
	add $t4, $t2, $t3	# add $t2 and $t3, place result in $t4
	sw $t4, result		# Store the answer at address result

loop:   j loop

	.data

value:	.word 37

result 	.word 0
\end{verbatim}

We could write a parser that would read a file and represents the
program in a suitable data structure for our emulator but we will skip
this step and start from a resonable data strucure that represents the
program. We will represent our program in two structures, one that
holds the data and one that holds the program. This is not really how
things work but it's fine for our needs and it make things easier.

The program will be a list of instructions where each instruction is a
tuple hodling the name of the instruction and its arguments. The
program above could be represented by the following list:

\begin{verbatim}
 [{:li, :t2, 42},
  {:lw, :t3, 0},            # value will be at data address 0
  {:add, :t4, :t2, :t3},
  {:sw, :t4, 4},            # result will be at data address 4
  {:j, -4}]                 # pc is allready pc+4 when we instruction is executed
\end{verbatim}

The data segment will in the sam way be represented as a sequence of
value but we make it a bit flexible and present it as a sequence of
segments, each with an initial address. We will make life a lot easier
for us and limit memory operations to full word operations i.e. 4
bytes. So the following would be the representation of a data area
with two segment, one starting at address 0 and one starting at
address 4.


\begin{verbatim}
 [ {0, [37]}, {4, [0]}]
\end{verbatim}

The data structures now presented are how the program and data area
are presented to us. These structures might not be the best structures
for our puproses but that is one of our first tasks of our
implementation, find a suitable representation of this data.

\section{The implementation}

We start by doing an overall design of the system. This will give us
insight into which moduels we will need an how data best is
represented.

\subsection{the state of the computation}

The fisrt thning we should think through is what the state of the
computation is. The program itself is of course part of the state but
since we have separated the code from the data the code part is
static. We will only read from the code using a {\em program
  counter}. The program counter is thus somethning that is part of the
state and it will of course change during the execution. The most
normal operation is that the program counter is incremented by 4 to
index the next instruction but it could also be set by a branch or
jump instruction.

The memmory is of course also part of the state and it will of course
change with each store operation. We should be able to index it using
addresses and let's assume that we only use addresses aligned by 4
bytes (you could change this later and allow for byte addressing). So
give that we should both be able to read from and write to this data
structure we might choose somthing different from the code area.

The final state of the computation are the registers of the CPU. In a
MIPS architecture we have 32 general purpose registers (well, register
0 always holds the value 0) but the MIPS assembler language has a
convention of usage. Register 28 is pointing to the data area, 29 is
used as a stackpointer, 30 as the frame base pointer etc.

In the code example above we saw registers called {\tt \$t4} and these
are temporary registers that are not preserved across a function
call. Registers that are preserved are called {\tt \$s0}, {\tt \$s1}
etc. How regsiters should be used is of course very important for the
programmer or for the developer of the assembler but it is nothing
that we need to consider. We should simply execute the instructions
give to us so for all that we care we could view them as 32 general
purpose registers (aprt from register 0).


\subsection{the execution}

When we start our emulation we will have a code area where we can read
the next instruction referred to by the program counter. We will
retreive the instruction, interpret it, possibly modify registers
and/or memory and then determine how to update the program
counter. Let's give it a try:

\begin{minted}{elixir}

  defmodule Emulator do
  
  def run(pc, code, reg, mem) do
    next = Program.read(code, pc)
    case next do

      :halt ->  
          :ok

      {:add, rd, rs, rt} ->
         pc = pc + 4
         s = Register.read(reg, rs)
         t = Register.read(reg, rt)
         reg = Register.write(reg, rd, s + t) # well, almost
         run(pc, code, reg, mem)
      
       :

     end
  end
\end{minted}
   
Let's totally ignore the problem of overflow and that negative numbers
should be represented as two's complement etc. This is not the course
of data architecture, we're only doing this for fun.

It should be rather simple to complete this piece of the
emulator. Load and store instructions will simply us a Memory module
that does the right thing and branch and jump instructions are simple.


We add one thing that will collect the output generated from the {\tt
  out} instruction. Let's keep keep the details hidden in a module. We
also add a more convient way of calling the emulator, the registers
are after all cleared when we start and let's always start with the
program counter set to zero.


\begin{minted}{elixir}
  def run(code, mem, out) do
     reg = Register.new() 
     run(0, code, reg, mem, out)
  end

  def run(pc, code, reg, mem, out) do
    next = Program.read(code, pc)
    case next do

       :halt ->
          Out.close(out)
    
       {:out, rs} ->
         pc = pc + 4
         s = Register.read(reg, rs)
         out = Out.put(out, s)
         run(pc, code, reg, mem, out)

       :
 \end{minted}

 That's it, all we have to do now is implement the supporting modules.

 







\end{document}

