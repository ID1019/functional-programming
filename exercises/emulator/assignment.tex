\documentclass[a4paper,11pt]{article}

\input ../include/preamble.tex

\usepackage{tikz}
\usetikzlibrary{automata,arrows,topaths,calc,positioning}

\usepackage{subcaption}

\usepackage{changepage}

\usepackage{listings}

\lstdefinelanguage{elixir}{
	morekeywords={case,catch,def,do,else,false,%
		use,alias,receive,timeout,defmacro,defp,%
		for,if,import,defmodule,defprotocol,%
		nil,defmacrop,defoverridable,defimpl,%
		super,fn,raise,true,try,end,with,%
		unless},
	otherkeywords={<-,->},
	sensitive=true,
	morecomment=[l]{\#},
	morecomment=[n]{/*}{*/},
	morestring=[b]",
	morestring=[b]',
	morestring=[b]"""
}

\lstset{language=elixir}

\begin{document}

\title{
  \textbf{An emulator}\\
  \large{Programming II}
}
\author{Johan Montelius}
\date{Spring Term 2021}
\maketitle
\defaultpagestyle

\section*{Introduction}

In this exercise we will combine your knowledge of assembly
programming and functional programming to implement an emulator for a
subset of MIPS assembler. You should have a basic understanding of
assembly programming but we will not do very advanced programming so
no need to know all instructions by heart. The idea here is not to
implement a complete emulator but to see how we can implement it in
Elixir.


\section{MIPS assembler}

We will only implement a small subset of the MIPS instruction set. If
we can emulate one arithmetic operation I think we could easily extend
the emulator to handle all. The subset is however chosen to cover the
various forms of instructions: arithmetic, load and store, branching
etc. We will also include some non-MIPS instructions that might be fun
to have when we run our programs.


These are the instructions that we will implement:

\begin{itemize}
\item {\tt add \$d \$s \$t} : add the values of register {\tt \$s} and {\tt \$t} and place result in {\tt \$d}.
\item {\tt sub \$d \$s \$t} : subtract the values of register {\tt \$s} and {\tt \$t} and place result in {\tt \$d}.
\item {\tt addi \$d \$t imm} : add the values of register {\tt \$t} and the immediate value {\tt imm} and place result in {\tt \$d}.
\item {\tt lw \$d offset(\$t)} : load the value found at address {\tt offset + \$t} and place it in {\tt \$d}.
\item {\tt sw \$s offset(\$t)} : store the value in register {\tt \$s} at address {\tt offset + \$t}. 
\item {\tt beq \$s \$t offset} : branch to {\tt pc + offset} if values at {\tt \$s} and  {\tt \$t} are equal.
\end{itemize}


We will also implement the two following instructions that do not
have any corresponding machine instructions but are very handy for our
implementation.

\begin{itemize}
\item {\tt halt} : halt the execution (normally implemented as an endless loop but we will actually stop)
\item {\tt out \$s} : output value at register location {\tt \$s}.
\end{itemize}

When you normally write assembly programs you of course use an editor
and write you programs in a text file. A program could for example
look like this:

\begin{verbatim}
        .text   
main:
         addi $1  $0  5       # $1 <- 5 
         lw   $2  $0  arg     # $2 <- data[arg]
         add  $4  $2  $1      # $4 <- $2 + $1
         addi $5  $0  $1      # $5 <- 1
loop:    sub  $4  $4  $5      # $4 <- $4 - $5
         bne  $4  $0  loop    # branch if not equal

        .data

arg:    .word 12
\end{verbatim}

We could write a parser that would read a file and represents the
program in a suitable data structure for our emulator but we will skip
this step and start from a reasonable data structure that represents
the program. We will represent our program in two structures, one that
holds the code and one that holds the data. This is not really how
things work but it's fine for our needs and it makes things easier.

The program will be a list of instructions where each instruction is a
tuple holding the name of the instruction and its arguments. The
program above could be represented by the following list:

\begin{verbatim}
  [{:addi, 1, 0, 5},     # $1 <- 5 
   {:lw, 2, 0, :arg},    # $2 <- data[:arg]
   {:add, 4, 2, 1},      # $4 <- $2 + $1
   {:addi, 5, 0, 1},     # $5 <- 1
   {:label, :loop},
   {:sub, 4, 4, 5},      # $4 <- $4 - $5
   {:out, 4},            # out $4
   {:bne, 4, 0, :loop},  # branch if not equal
   :halt]
\end{verbatim}

We have added two extra instructions: {\tt :out} that will be used for
output and {\tt :halt} that will terminate the execution.

The data segment will in the same way be represented as a sequence of
labels and values.

\begin{verbatim}
  [ {:label, :arg}, {:word, 12} ]
\end{verbatim}

We combine the code segment and data segments into one tuple and this
is what we will call a program in this tutorial.

\begin{verbatim}
 {:prgm,  code(), data()} 
\end{verbatim}

This data structure is how the program is presented to us. The
structure might not be the best structures for our purposes but that
is one of our first tasks of our implementation, find a suitable
representation that we can work with. 

\section{The implementation}

We start by doing an overall design of the system. This will give us
insight into which modules we will need an how data best is
represented.

\subsection{the state of the computation}

The first thing we should think through is what the state of the
computation is. The program itself is of course part of the state but
since we have separated the code from the data the code part is
static. We will only read from the code using a {\em program
  counter}. The program counter is thus something that is part of the
state and it will of course change during the execution. The most
normal operation is that the program counter is incremented by 4 to
index the next instruction but it could also be set by a branch or
jump instruction.

The memory is of course also part of the state and it will of course
change with each store operation. We should be able to index it using
addresses and let's assume that we only use addresses aligned by 4
bytes (you could change this later and allow for byte addressing). So
given that we should both be able to read from and write to this data
structure we might choose something different from the code area.

The final state of the computation are the registers of the CPU. In a
MIPS architecture we have 32 general purpose registers (well, register
0 always holds the value 0) but the MIPS assembler language has a
convention of usage. Register 28 is pointing to the data area, 29 is
used as a stack pointer, 30 as the frame base pointer etc. For our
purposes the registers are all the same (apart from register zero).


\subsection{the execution}

When we start our emulation we will have a code area where we can read
the next instruction referred to by the program counter. We will
retrieve the instruction, interpret it, possibly modify registers
and/or memory and then determine how to update the program
counter. Let's give it a try:

\begin{minted}{elixir}
  defmodule Emulator do

  def run(prgm) do
     {code, data} = Program.load(prgm)
     reg = Registers.new() 
     run(0, code, reg, data)
  end

  def run(pc, code, reg, mem) do
    next = Program.read_instruction(code, pc)
    case next do

      :halt ->  
          :ok

      {:add, rd, rs, rt} ->
         pc = pc + 4
         s = Register.read(reg, rs)
         t = Register.read(reg, rt)
         reg = Register.write(reg, rd, s + t)  # well, almost
         run(pc, code, reg, mem)
      
       :

     end
  end
\end{minted}
   
Let's totally ignore the problem of overflow and that negative numbers
should be represented as two's complement etc. This is not the course
of data architecture, we're only doing this for fun.

It should be rather simple to complete this piece of the
emulator. Load and store instructions will simply us a module
that does the right thing and branch and jump instructions are simple.

\subsection{the output}

The {\tt :out} instruction could of course echo the value of the
register in the terminal but let's add a feature that collects the
output in a list that is returned when the program terminates. We can
hide the details of how things are done in a separate module that
takes care of everything.


\begin{minted}{elixir}
  defmodule Emulator do

  def run(prgm) do
     {code, data} = Program.load(prgm)
     out = Out.new()
     reg = Registers.new() 
     run(0, code, reg, data, out)
  end


  def run(pc, code, reg, mem, out) do
    next = Program.read_instruction(code, pc)
    case next do
 
     :halt ->
        Out.close(out)
    

      {:out, rs} ->
         pc = pc + 4
         s = Register.read(reg, rs)
         out = Out.put(out, s)
         run(pc, code, reg, mem, out)          

       :

     end
  end
\end{minted}

 That's it, all you have to do now is implement the supporting modules and you have a MIPS emulator up and running.

 \section{Your implementation}

 There are four modules that you need to implement, three supporting
 modules and then the emulator itself. Start with the supporting
 modules and test them before you implement the final solution. These
 are the modules:

 \begin{itemize}
  \item{\tt Program:} The module should be able to create a code
   segment and a data segment given a program description. It should
   provide functions to read from the code segment and both read and
   write to a data segment.

  \item{\tt Register:} The register module should handle all operations for
   the registers: create a new register structure and, read and write to
   individual registers.

 \item{\tt Out:} The module should collect the output from the
   execution and be able to return it as a list of integers. 

 \item{\tt Emulator:} This module is the heart of the system. It
   should be able to take a program, transform it to a code and data
   segment and then execute the program and return the output.
 \end{itemize}

 The different modules all handle a state but the requirements on
 these states are different. The code segments only needs to provide a
 quick lookup operation given a program counter but does not have to
 care about changing the code itself. The data segment should of
 course be easy both to read and write to and the same goes for the
 registers. The difference between the data segment and the register is
 that the register is of a fixed and fairly small size. The output
 structure should support incremental writing but no reading, apart
 from when the list is returned. These differences should be taken
 into account in you implementation, choose a data structure that is
 suitable in each case.









\end{document}

