\documentclass[a4paper,11pt]{article}

\input ../include/preamble.tex


\begin{document}


\title{
    \textbf{An environment}\\
    \large{Programming II - Elixir Version}
}
\author{Johan Montelius}
\date{Spring Term 2023}
\maketitle
\defaultpagestyle


\section*{Introduction}

The name of this assignment might be confusing and it will only be
clear later in the course why we call this assignment "An
environment". What you should implement is a key-value database, also
called a "map", that can be used to look up the value associated with
a key. We don't make any assumptions about the keys and will simply
compare them using the regular operators.

You should implement the map using two different techniques, both
implementations should provide the following interface:

\begin{itemize}
\item {\tt new()} : return an empty map.

\item {\tt add(map, key, value)} : return a map where an association of
  the key {\tt key} and the data structure {\tt value} has been added to
  the given {\tt map}. If there already is an association of the key
  the value is changed.

\item {\tt lookup(map, key)} : return either {\tt \{key, value\}}, if
  the key {\tt key} is associated with the data structure {\tt value},
  or {\tt nil} if no association is found.

\item {\tt remove(key, map)} : returns a map where the
  association of the key {\tt key} has been removed
\end{itemize}


We will in this assignment use keys of atoms and the values will be
integers but the implementation should not take this for granted.

\subsection*{a map as a list}

If we assume that the map will be small, we can represent it as a list
of key-value tuples. The list \verb+[{:a, 12}, {:b, 13}]+ would then
represent a map where the atom \verb+ :a+ is associated with 12 and the
atom \verb+ :b+ is associated with 13.

{\em Note - Elixir has a way of printing lists of binary tuples where
  the first element is an atoms. The above list is printed as
  \verb+[a: 12, b: 13]+. You can also use this syntax which makes it
easier to write long sequences of key value pairs.}

In a module {\tt EnvList}, implement the functions above and test
that you can perform the different operations. Would it make sense to
keep the list sorted?

\subsection*{a map as a tree}

If the map grows larger the list implementation might not be our best
option. A tree structure would probably be a better approach. You
could implement you tree any way you want but why not choose a simple
representation where an empty tree is represented by the atom {\tt
  nil} and a node by the structure {\tt \{:node, key, value, left,
  right\}}.

In a module {\tt EnvTree}, implement the functions above and test that
you can perform the different operations.

The tree should of course be sorted to make the lookup operation
efficient. You don't have to implement a balanced tree but this would
of course be something that one would need to consider.

The {\tt add/3} and {\tt lookup/2} functions are fairly straight
forward to implement. Identify the base cases and then how to recurse
down the right or left branch. Use the following skeleton code to get
you started:

\begin{minted}{elixir}
  def add(nil, key, value) do
    ... adding a key-value pair to an empty tree ..
  end

  def add({:node, key, _, left, right}, key, value) do 
    ... if the key is found we replace it .. 
  end

  def add({:node, k, v, left, right}, key, value) when key < k do
    ... return a tree that looks like the one we have
    but where the left branch has been updated ...
  end
  def add({:node, k, v, left, right}, key, value) do
    ... same thing but instead update the right banch
  end
\end{minted}

Remember that we are in fact not "updating" the tree that we have but
rather constructing a copy of the tree where we have added the new
key-value pair.

Implementing {\tt lookup/2} is very similar to the {\tt add/3}
function. The difference is of course that we are not building a new
tree but returning the found key-value pair or {\tt nil} if not found.

Implementing {\tt remove/2} is slightly more tricky and you have to
remember the algorithm how to do this. The idea is to first locate the
key to remove and then replace it with the leftmost key-value pair in
the right branch (or the rightmost in the left branch). Removing and
returning the leftmost key-value pair should be simple so if you only
can handle all special cases you should be up an running in a few minutes.

\begin{minted}{elixir}
  def remove(nil, _) do  ...  end
  def remove({:node, key, _, nil, right}, key) do  ...  end
  def remove({:node, key, _, left, nil}, key) do ... end    
  def remove({:node, key, _, left, right}, key) do 
    ...   = leftmost(right)
    {:node, ..., ..., ..., ...}
  end
  def remove({:node, k, v, left, right}, key) when key < k do
     {:node, k, v, ..., right}
  end
  def remove({:node, k, v, left, right}, key) do 
     {:node, k, v, left, ...} 
  end
  def leftmost({:node, key, value, nil, rest}) do ...  end
  def leftmost({:node, k, v, left, right}) do
    ... = leftmost(left)
    ...
  end
\end{minted}


\subsection*{benchmark}

Now let's do some benchmarks to see how the implementations perform
and compare to each other. We want to see how the different
implementations work with a growing number of key-value pair so let's
set up a benchmark where we first construct a map of a number of
elements and then measure the time it takes to perform an
operation. We will set up the benchmark so that we use the same
key-value pairs for each of the implementations.

Measuring time can be done using the function {\tt :timer.tc(fun)}
(this is the Erlang module {\tt timer} and we thus write it using the
regular atom syntax). This function will take a function, call this
function and return a tuple with the number of microseconds it took
and the result. The argument to {\tt tc/1} is a function and as you
will see this is quite efficient.

Using functions as arguments we can for example use the function {\tt
  Enum.each/2} to generate a list of random numbers. Try the following
with different values of $n$ and $i$:

\begin{minted}{elixir}
   Enum.map(1..n, fn(_) -> :rand.uniform(i) end)
\end{minted}

To build a key-value store using our {\tt EnvList} module we can do as follows:

\begin{minted}{elixir}
    seq = Enum.map(1..i, fn(_) -> :rand.uniform(i) end)

    list = Enum.reduce(seq,  EnvList.new(), fn(e, list) -> 
                  EnvList.add(list, e, :foo) 
                  end)
\end{minted}

We first build a list of random numbers (from 0 to $i$) and then use
this list as keys when we add dummy key-value pairs to an empty store.

Once we have a store of $i$ pairs we can construct another sequence of $n$
random numbers and use them when we benchmark for example adding a new pair:

\begin{minted}{elixir}
    seq = Enum.map(1..n, fn(_) -> :rand.uniform(i) end)
    
    {tla, _} = :timer.tc(fn() -> Enum.each(seq, fn(e) ->
                      EnvList.add(list, e, :foo)
                      end)
                    end)
\end{minted}

Note that we here ignore the constructed store with the new pair, we
simply add the item and then drop the result. In this way we can first
build a store containing for example 64 entries and then measure the
time it takes to add a thousand new pairs without growing the store.

We can now do the same with the other operations and combine everything
in one function:

\begin{minted}{elixir}
  def bench(i, n) do 				    
    seq = Enum.map(1..i, fn(_) -> :rand.uniform(i) end)

    list = Enum.reduce(seq,  EnvList.new(), fn(e, list) -> 
                    EnvList.add(list, e, :foo) 
                    end)

    seq = Enum.map(1..n, fn(_) -> :rand.uniform(i) end)
    
    {add, _} = :timer.tc(fn() ->
                    Enum.each(seq, fn(e) -> 
                      EnvList.add(list, e, :foo) 
                      end) 
                    end)

    {lookup, _} = :timer.tc(fn() -> 
                    Enum.each(seq, fn(e) ->
                      EnvList.lookup(list, e) 
                      end) 
                    end) 
    
    {remove, _} = :timer.tc(fn() -> 
                    Enum.each(seq, fn(e) -> 
                      EnvList.remove(list, e) 
                    end) 
                   end) 

    {i, add, lookup, remove}
  end
\end{minted}

The {\tt bench/2} function is then used repeatedly with growing values
of $i$. If we print out the result we will have a nice sequence of
numbers to plot or present in a table.

{\em The specification {\tt ~12.2f} means print a float with 2
  decimals right aligned using a width of 12 characters. Change it to
  anything that suits your needs.}

\begin{minted}{elixir}
  def bench(n) do

    ls = [16,32,64,128,256,512,1024,2*1024,4*1024,8*1024]        

    :io.format("# benchmark with ~w operations, time per operation in us\n", [n])
    :io.format("~6.s~12.s~12.s~12.s\n", ["n", "add", "lookup", "remove"])

    Enum.each(ls, fn (i) ->
      {i, tla, tll, tlr} = bench(i, n)
      :io.format("~6.w~12.2f~12.2f~12.2f\n", [i, tla/n, tll/n, tlr/n])
    end)
  end
\end{minted}

Once you have your benchmark up an running you can compare your two
implementations of the key-value store. You might also wan to see how
well you did compared to the key-value store that comes with the
Elixir system form start. There is a module called {\tt Map} that does
exactly what you want and more (the functions are called {\tt put/3},
{\tt get/2} and {\tt delete/2}). It uses a trier data structure (a
tree of hash tables) and is implemented in C++ so it is not strange if
it performs better than your solution but my guess is that you're not
that far off.

Take a look at the documentation of the {\tt Map} module. From now on
you can use it if you want. 

\end{document}
