\documentclass[a4paper,11pt]{article}

\input ../include/preamble.tex

\begin{document}

\title{
    \textbf{Enumerable primes}\\
    \large{Programming II}
}
\author{Johan Montelius}
\date{Spring Term 2022}
\maketitle
\defaultpagestyle

\section*{Introduction}

Your task is to implement a data structure that holds the infinite
sequence of primes and can be used by the {\tt Enum} and {\tt Stream}
modules. In the end we should be able to do like this:

\begin{minted}{elixir}
  iex(10)> Enum.take( Stream.map(Primes.primes(), fn(x) -> 2*x end), 5)
  [4, 6, 10, 14, 22]
\end{minted}

or, using the pipe notation:

\begin{minted}{elixir}
  iex(11)> Primes.primes() |>
           Stream.map(fn(x) -> 2*x end) |>
           Enum.take(5)
\end{minted}


The primes should implement the {\tt Enumarable} protocol and so you
should have some understanding of {\tt Structs} and {\tt Protocols} but
don't be afraid, things will work out.

\section*{Infinite sequence of primes}

Before implementing the enumerable protocol you should implement a
infinite sequence of primes using anonymous functions. We divide this
into three tasks and we start by implementing the infinite sequence of
natural numbers (starting on 3).

\subsection*{z}

The function {\tt z/1} should take one argument, an integer, and
return a function that when applied returns a tuple {\tt \{n, fun\}}
where {\tt n} is the integer given and {\tt fun} is a function that
takes no argument and when applied returns a tuple {\tt\{n+1,
  fun\}}. The function now returned should of course return a tuple
with the firste element {\tt n+2} etc.

This is what it should look like:

\begin{minted}{elixir}
iex(29)> z = Primes.z(3)
z = Primes.z(3)
#Function<4.106663018/0 in Primes.z/1>
iex(30)> {_, z} = z.()
{_, z} = z.()
{4, #Function<4.106663018/0 in Primes.z/1>}
iex(31)> {_, z} = z.()
{_, z} = z.()
{5, #Function<4.106663018/0 in Primes.z/1>}
iex(32)> {_, z} = z.()
{_, z} = z.()
\end{minted}

\subsection*{filter}

The next thing you should implement is a {\tt filter/2} function. This
function should take a function that can be used to generate a
sequence as above, and a number, {\tt f}, that it should use as a
filter. It should return a tuple {\tt \{p, fun\}} where {\tt p} is
the first number in the sequence that is not divisable by {\tt f}. The
function, the second element in the tuple, is a function that
will deliver the next number in the sequnce not divisable by {\tt
  f}. This is what it should look like:

\begin{minted}{elixir}
iex(55)> {_n, f} = Primes.filter(Primes.z(3), 2)
{_n, f} = Primes.filter(fn() -> Primes.z(3) end, 2)
{3, #Function<1.106663018/0 in Primes.filter/2>}
iex(56)> {_n, f} = f.()
{_n, f} = f.()
{5, #Function<1.106663018/0 in Primes.filter/2>}
iex(57)> {_n, f} = f.()
{_n, f} = f.()
{7, #Function<1.106663018/0 in Primes.filter/2>}
\end{minted}

If this works we have way of implementing a sequence of all numbers
not divisable by two; why not continue on this path and implement
the sieve of Eratosthenes.

\subsection*{sieve}

Implement a function {\tt sieve/2} that takes a function and a (prime)
number as arguments. The function should be a sequence generator like
{\tt z} of {\tt f} in the above examples. We will assume that the
generator returns all possible primes greater than the prime given as
the second argument.

\begin{minted}{elixir}
  def sieve(n, p) do
    :
    :
  end
\end{minted}

When we say ``possible'' we mean that the numbers generated have been
filtered by all primes less than {\tt p}. How do you generate the next
prime greater than {\tt p}? Can you also create a function that
returns all possible primes greater than this prime although they
might be divisable by the new prime that you generated?

If you can do this, how do you let {\tt sieve/2} return a tuple
{\tt \{next, fun\}} where {\tt next} is the next prime greater than
{\tt p} and {\tt fun} is function that returns all possible primes
  greater than, and filtered by, the prime that you have found?

\subsection*{primes}

This final thing is to boot strap everything and provide a function
{\tt primes} that returns a function that when applied generates a
tuple {\tt \{2, fun\}} where {\tt 2} of course is the first prime and
{\tt fun} is a function that will return the remaining primes.

\begin{minted}{elixir}
  def primes() do
    fn() -> {2, fn() -> .......  end} end
  end
\end{minted}

Hmm, what do we write on the dotted line? We could try with {\tt
  z(3)}; that would give us all numbers greater than 3 but we don't
want to have all the even numbers. Should we filter the number from
{\tt z(3)} with 2? Close, but that would give us the sequence 2,3,5,6
and we do not want to have 6. Ahh,....

\begin{minted}{elixir}
iex(4)>p = Primes.primes()
#Function<2.128507103/0 in Primes.primef/0>
iex(5)> {_, p} = p.()
{_, p} = p.()
{2, #Function<7.128507103/0 in Primes.primef/0>}
iex(6)> {_, p} = p.()
{_, p} = p.()
{3, #Function<4.128507103/0 in Primes.sieve/2>}
iex(7)> {_, p} = p.()
{_, p} = p.()
{5, #Function<1.128507103/0 in Primes.filter/2>}
iex(8)> {_, p} = p.()
{_, p} = p.()
{7, #Function<1.128507103/0 in Primes.filter/2>}
iex(9)> {_, p} = p.()
{_, p} = p.()
{11, #Function<1.128507103/0 in Primes.filter/2>}
iex(10)>
\end{minted}


\section*{Enumerable}

So now you can generate an infinite sequence of primes, time to make
it work with the {\tt Enum} and {\tt Strem} modules. The thing you now
nned to do is implement the {\tt Enumerable} protocol.

A protocol is defined for a particular struct so we have to create a
struct associated with the {\tt Primes} module. It's a simple data
structure with only one element, {\tt next}. that will be your primes
function. The function {\tt primes/0} will now return a {\tt Primes}
struct with the initial fucntion. 

\begin{minted}{elixir}
defmodule Primes do

  defstruct [:next]

  def primes() do
    %Primes{next: ... }
  end
\end{minted}

The protocol can now be defined and we nned to implement four
functions. The first three are not applicable since we have an
infinite sequence so we simply return {\tt\{:error, \_\_MODULE\_\_\}}. It
is in the function {\tt reduce/3} where the magic happens.

\begin{minted}{elixir}
defimpl Enumerable do

    def count(_) do  {:error, __MODULE__}  end
    def member?(_, _) do {:error, __MODULE__}  end
    def slice(_) do {:error, __MODULE__} end

    def reduce(_,       {:halt, acc}, _fun) do
       {:halted, acc}
    end
    def reduce(primes,  {:suspend, acc}, fun) do
       {:suspended, acc, fn(cmd) -> reduce(primes, cmd, fun) end}
    end
    def reduce(primes,  {:cont, acc}, fun) do
      {p, next} = Primes.next(primes)
      reduce(next, fun.(p,acc), fun)
    end      
  end
\end{minted}

Now for you to implement the function {\tt next/1}, it should return a
tuple, the next prime and a Primes struct that represents the
continuation. If everything works you should be able to do something like this.

\begin{minted}{elixir}
  iex(59)> Enum.take( Stream.map(Primes.primes(), fn(x) -> 2*x end), 5)
  [4, 6, 10, 14, 22]
\end{minted}






\end{document}
