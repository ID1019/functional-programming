\documentclass[a4paper,11pt]{article}

\input ../include/preamble.tex


\begin{document}


\title{Reduce and friends}

\author{Johan Montelius}
\date{Spring Term 2024}

\maketitle

\defaultpagestyle

\section*{Introduction}

In this exercise you're going to explore to power of higher order
functions and how you can use them. Your first task is to implement a
set of functions using regular recursive programming and then, as a
second task, implement them using your higher order functions.


\section*{Recursive functions}

Let's keep it simple in the beginning and work with lists of
integers. Your first task is to implement these functions using
regular recursive programming:

\begin{itemize}
  \item {\tt length([integer()]) :: integer() } : return the length of the list. 

  \item {\tt even([integer()]) :: [integer()]} : return a list of all even number.

  \item {\tt inc([integer()], integer()) :: [integer()] } : return a list where each
    element of the given list has been incremented by a value.

  \item {\tt sum([integer()]) :: integer()} : return the sum of all values of the given list.

  \item {\tt dec([integer()], integer()) :: [integer()] } : return a list where each
    element of the given list has been decremented by a value.

  \item {\tt mul([integer()], integer()) [integer()] } : return a list where each
    element of the given list has been multiplied by a value.

  \item {\tt odd([integer()]) :: [integer()]} : return a list of all odd number.

  \item {\tt rem([integer()], integer()) :: [integer()] } : return a list with the
    result of taking the reminder of dividing the original by some integer.

  \item {\tt prod([integer()]) :: integer() } : return the product of
    all values of the given list (what is the product of all values of
    a an empty list?)

  \item {\tt div([integer()], integer()): [integer()] } : return a list of all
    numbers that are evenly divisible by some number.
    
  \end{itemize}
  
  These functions should not be too hard to implement even though you
  might feel like you're doing the same thing over and over again. If
  you group them into three categories you will find three different
  patterns; identify the patterns before you proceed. 

  \section*{map, reduce and filter}

  The first pattern is where we traverse through the list and returns
  a new list where we some how calculated a new value given the
  original value. This pattern is called {\em map} and will be you
  first function to implement.

  \begin{itemize}
  \item {\tt map([a()], ( a() -> b())) :: [b()]} : take a list of values
    and a function that when applied to a value returns another
    value. Create a new list with the resulting values when you apply
    the function on each value in the original list.
  \end{itemize}


  The second pattern is slightly more complex in that it will reduce
  the original list of integers to one single value. Your
  implementation of {\tt sum/1} added all the numbers together while
  the function {\tt mul/1} multiplied them. If you look at your
  functions you probably came up with a solution where the sum of all
  integers in an empty list i zero and ( which of course could be
  discussed) the product of all values in an empty list is one. We
  thus need not only provide the function but also an initial value to
  know what to return if we reduce an empty list.

  \begin{itemize}

  \item {\tt reduce([a()], b(), (a(), b() -> b())) :: b()} : apply the function on the value of the list
    and the accumulated value. Return the final value.

  \end{itemize}

  You may now realize that we have two different ways of implementing
  this function and the way you choose probably depends on how you
  chose to implement the functions {\tt sum/1} and {\tt prod/1}. Is
  your implementation {\em tail recursive}?  If we should provide a
  function that could be used with very large input we could go
  through the trouble to write the function as a tail recursive
  function, or rather why not implement two versions of the function
  one tail recursive ({\tt reducel}) and one that is not ({\tt
    reducer}).

  The third function is one that will select a few values from the
  given list and we call this {\tt filter/2}. The function is given a
  list of values and a function that will return either {\tt true} or
  {\tt false} and determine if we should add the value to the list
  returned.
  
  \begin{itemize}

  \item {\tt filter([a()], (a() -> boolean())) :: [a()]}
    : apply the function to all values of the given list and return a
    list with those values for which the function returned true.

  \end{itemize}

  Also this function can be implemented i two different ways; which?
  There is of course a third way if we do not have to maintain the
  order of the values - would that be more efficient?

  Now that you have your higher order functions, re-implement the
  functions of the first task. Also implement a function that takes a
  list of integers and returns the sum of the square of all values
  less than $n$.

  The final twist is to use the {\em pipe operator} ({\tt |>}). The
  operator can be used to pass the returned value of one function call
  to the first argument of the next function call. Take for example the following definition:

  \begin{minted}{elixir}
    def test(lst, x, y) do
       lst_one = one(lst, x)
       lst_two = two(lst_one, y)
       three(lst_two)
    end
  \end{minted} 

  This could of core be written as follows:

  \begin{minted}{elixir}
    def test(lst, x, y) do
       three(two(one(lst, x), y))
    end
  \end{minted}   

  This is more compact but the question if often if it's easy to
  quickly see what is going on.  Using the pipe operator we can write
  the function as follows:

  \begin{minted}{elixir}
    def test(lst, x, y) do
       lst |>
         one(x) |>
         two(y) |>
         three()
    end
  \end{minted}

  This syntax saves us from inventing variables names, its rather
  compact while not as complicated to understand what is
  happening. The pipe operator should be used with care and only when
  it makes a difference.

  
\end{document}
