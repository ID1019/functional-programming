\documentclass[a4paper,11pt]{article}

\input ../include/preamble.tex

% Finite state machines
\usepackage{tikz}
\usetikzlibrary{automata,arrows,topaths}


\begin{document}

\title{
    \textbf{Streams }\\
    \large{Programming II - Elixir Version}
}
\author{Johan Montelius}
\date{Spring Term 2021}
\maketitle
\thispagestyle{fancy}


\section*{Introduction}

The goal of this assignment is that your should practice working with
functions as arguments and return values. It also gives insight into
how streams are represented and handled in Elixir.

To begin with we will implement a range in and see how we can represent
it in a generic way. We then extend this representation to lazy
streams. The interface that we use will be very similar to the {\tt
  Enumerables} protocol.


\section{A range}

Let's begin by representing a range of integers. We can of course do
this by a list of integers but that wold not be so practical if we are
talking about ranges of thousands of integers. Why not simply
represent a range as a tuple {\tt \{:range, from, to\}} where {\em
  from} and {\em to} set the limits of, but are included in, the
range. Choosing this representation we continue and implement some
functions over ranges.

\subsection{sum of all integers}


We could then implement a function {\tt sum/1} that could sum all the
integers. We don't do a simple calculation but recursively reduce the
range to a value. 

\begin{minted}{elixir}
def sum({:range, to, to}) do n end  
def sum({:range, from,  to}) do  from + sum({:range, from+1, to}) end  
\end{minted}

We could of course now write arbitrary functions that iterate over the
integers in the range but why not write a generic function that works
in the same way as {\tt foldl/3}. Let's call it {\tt reduce/3}:

\begin{minted}{elixir}
  def reduce({:range, from, to}, acc, fun) do
     if from <= to do
       reduce({:range, from+1, to}, fun.(from, acc), fun)
     else 
       acc
     end
  end
\end{minted}

\noindent Summing up all integers is now a one-liner:

\begin{minted}{elixir}
  def sum(range) do reduce(range, 0, fn(x,a) x + a end) end
\end{minted}

\noindent We can even implement a {\tt map/2} function in one line of
code. The list of mapped elements might be in the reversed order but we
can always reverse the result when we have it.

\noindent Implement {\tt map/2}. 

\subsection{stop half way}

The implementation we have gives us the ability to implement arbitrary
fold operations over ranges. But what if we want to implement a
function that returns the first $n$ elements of a range. We could of
course implement is explicitly:

\begin{minted}{elixir}
  def take({:range, _  , _}, 0) do [] end
  def take({:range, from, to}, n) do
     [ from | take({:range, from+1, to}, n-1) ]
  end  
\end{minted}

This works fine but we might want to implement a more generic function
that could do arbitrary things with the first $n$ elements. We can
change how {\tt reduce/2} works and use it to stop the reduction when
we have found $n$ elements. First we let the second argument signal
what should be done. For the regular case we want the reducer to
continue the execution but if we tell it to halt it should stop.

\begin{minted}{elixir}
  def reduce({:range, from, to}, {:cont, acc}, fun) do
     if from <= to do
       reduce({:range, from+1, to}, fun.(from, acc), fun)
     else 
       {:done, acc}
     end
  end
  def reduce({:range, from, to}, {:halt, acc}, fun) do
    {:halted, acc}
  end
\end{minted}

\noindent We now rewrite the {\tt sum/1} function to use the new protocol.

\begin{minted}{elixir}
  def sum(range) do
     reduce(range, 0, fn (x, a) -> {:cont, x + a} end)
  end
\end{minted}

\noindent That did not make things easier but now take a look at this:

\begin{minted}{elixir}
  def take(range, n) do
   reduce(range, {:cont, {:sofar, 0, []}},
      fn (x, {:sofar, s, acc}) ->
	if s == n do
	  {:halt, acc}
	else
	  {:cont, {:sofar, s+1, [x|acc]}}
	end
      end)
  end
\end{minted}

\noindent As long as $s$ is not equal to $n$ we allow the reducer to continue
but when we have all the integers that we need we tell it to stop.

Given this implementation the {\tt sum/1} and {\tt take/2} will of
course return {\tt \{:done, 21\}} or {\tt \{:halted, [4,3,2,1]\}} but
this is for you to fix. Now use the extended version of the reducer to
implement the following functions.

\begin{itemize}
\item A function that returns all integers in the range less than $n$.
  
\item A function that returns the sum of all integers less than $n$

\item A function that returns the last integer that makes the sum exceed $n$

\end{itemize}

\noindent It's tricky in the beginning but once you get around thinking about
what the function does it becomes easier. Note how much less boiler
plate code you have to write. The whole mechanism of how to do the
recursion over the range is done by the reducer.

\subsection{suspend and continue}

Now for the last step to complete the reducer. What if we want to run
through the beginning of a range to a certain point as we do in {\tt
  take/2} but then return not only the result so far but {\em a
  continuation}. A continuation would be something that allows us to
continue the execution and retrieve for example the following five
elements.

We extend the reducer with yet another command {\tt :suspend} that
should do the trick. Here we see the power of being able to create a
closure. The tuple that we return {\tt \{:suspend, acc, closure\}} now
contains a closure that includes the current range and the function
that we have been using.

\begin{minted}{elixir}
  def reduce(range, {:suspend, acc}, fun) do
    {:suspended, acc, fn (cmd) -> reduce(range, cmd, fun) end}
  end
\end{minted}

\noindent Let's see how we can pick one element at the time from the range.

\begin{minted}{elixir}
  def head(range) do
     reduce(range, {:cont, :na}, fn (x,_) -> {:suspend, x} end)
  end
\end{minted}

\noindent This looks very strange at first site but it is of course simple once
we go through the steps of the evaluation.

\noindent Assume we have a range {\tt \{:range, 1, 10\}} and call our function
{\tt head/1}. This will result in a call to:

\begin{minted}{elixir}
   reduce({:range, 1, 10}, {:cont, :na}, fn (x,_) -> {:suspend, x} end)
\end{minted}

\noindent The reducer is asked to make one step and prepared the call to:

\begin{minted}{elixir}
   reduce({:range, 2, 10}, fun.(1,:na), fun)
\end{minted}

\noindent Since we have provided the function this evaluates to:

\begin{minted}{elixir}
   reduce({:range, 2, 10}, {:suspend, 1}, fun)
\end{minted}
 
\noindent The function reduce will now return what we were looking for:

\begin{minted}{elixir}
   {:suspended, 1,  fn(cmd) -> reduce({:range, 2, 10}, cmd, fun) end}
\end{minted}

\noindent Let's see if this works in practice:

\begin{minted}{elixir}
> {:suspended, n, cont} = High.head({:range, 1, 10})
{:suspended, 1, #Function<7.2894770/1 in High.reduce/3>}

> {:suspended, n, cont} = cont.({:cont, :na})
{:suspended, 2, #Function<7.2894770/1 in High.reduce/3>}

> {:suspended, n, cont} = cont.({:cont, :na})
{:suspended, 3, #Function<7.2894770/1 in High.reduce/3>}
\end{minted}

\noindent As an exercise you can rewrite {\tt take/2} to return a tuple {\tt
  \{:suspended, elements, continuation\}} to be able to take more
elements from the range.

\subsection{Reduce a list}

If you take a look at your functions {\tt sum/1}, {\tt map/2}, {\tt
  take/2} etc, you will notice that they do not know anything about
what a range looks like. The only function that knows what a range
look like is {\tt reduce/3}. What if we extend this function so that
it could also operate over a lists. If we do this we could use the
same code to take element of a list as we use to take elements of a
range. It might not be the most efficient solution but we do not have
to keep track of if we have a list or a range in our hand.

If you do this right the only thing you need to add are two clauses:

\begin{minted}{elixir}
  def reduce([from|rest], {:cont, acc}, fun) do
      ...
  end
  def reduce([], {:cont, acc}, _) do
      ...
  end
\end{minted}

\noindent Given this we can compose functions that operate over ranges
but when we do we need to be careful. Take a look at the two examples
below, is there a difference in execution?


\begin{minted}{elixir}
  def take_n_map(range, n, f) do map( take(range, n), f) end

  def map_n_take(range, n, f) do taken( map(range, f), n) end  
\end{minted}

\noindent The end result should be the same. Taking five elements from
a range and apply a function to each of the elements can not be
different from applying the function to the elements of the range and
then take five elements from the resulting list. The only difference is
execution time but this will be quite significant of the range is
huge.

We would rather be lazy in our evaluation and not map the function to
all elements of the range unless it not required. How do we do this?

The answer is streams.

\section{Streams}

So now we now what pour task is, we should implement lazy ranges. If
you map a function to a range, nothing should happen unless someone
requests the first element. Only then should we select the first
element from the range and apply the function.

\subsection{give me the next element}

Take a look at {\tt reduce/3}, where does it actually know what the
range look like? It is only in the first clause where i does the
pattern matching, checks if {\tt from} is less than {\tt to} and then
increments {\tt from}. What the clause is actually doing is checking if
there is another element in the range.

What if we provided another representation of ranges, {\tt \{:stream,
  fun\}}, where {\tt fun} is a function that will, when applied to no
arguments, return either {\tt \{:ok, from, cont\}} or {\tt :nil}. We
could then extend {\tt reduce/3} to also handle this type of range.

\begin{minted}{elixir}
  def reduce(...   , {:cont, acc}, fun) do
    case ...   do
      {:ok, from, cont} ->
	reduce(... , fun.(from,acc), fun)
      :nil ->
	{:done, acc}
    end
  end
\end{minted}

\noindent To see the magic we should now define a stream of fibonacci
numbers. Since we define what the next element should look like we are
not constrained to sequential ranges. The stream is potentially
infinite so we have to care full when we work with it.


\begin{minted}{elixir}
  def fib() do {:steam, fn () -> fib(1,1) end} end

  def fib(f1,f2) do
     {:ok, f1, fn () -> fib(f2, f1+f2) end}
  end
\end{minted}

\noindent Let's try this an see if it works:

\begin{minted}{elixir}
 >  High.take(High.fib(), 5)
\end{minted}

\noindent Magic?

\subsection{we're not done yet}

Let's implement a range as a lazy stream. We can now use it do do some experiments. 

\begin{minted}{elixir}
  def range(from, to) do
    {:stream, fn() -> next(from, to) end}
  end

  def next(from, to) do
    if from <= to do
      {:ok, from, fn() -> next(from+1, to) end}
    else
      :nil
    end
  end
\end{minted}

\noindent This call below works but it might not do exactly what we
want it to do. The reason is that our {\tt map/2} function will
happily generate a list of ten integers that is then passes to {\tt
  take/2} that only picks the first five. We want {\tt map/2} to
return a lazy stream.

\begin{minted}{elixir}
take( map( range(1,10), fn(x) -> x + 1 end), 5)
\end{minted}

\noindent Let's change out map function so it becomes aware of the difference
between ranges and lazy streams. We do not have to do a lot to make it work. Add this clause as the first clause of {\tt map/2}:

\begin{minted}{elixir}
  def map({:stream, next}, fun) do
    {:stream,
     fn () -> case ...   do
  		{:ok, from, cont} ->
  		  {:ok, ... , cont}
  		:nil ->
  		  :nil
  	      end
       end
    }
  end
\end{minted}

There is of course a lot more to be said about the {\em Enum} and {\em
  Stream} libraries in Elixir and of course the implementation of
ranges and streams are more complex than in this tutorial but I hope
that you now have a better understanding of how they work.

I also hope that you see how higher order programming can be used to
implement some rather complex data structures in a not so complicated
way.


\end{document}
