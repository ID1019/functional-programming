\documentclass[a4paper,11pt]{article}

\input ../include/preamble.tex


\begin{document}


\title{
    \textbf{Evaluating an expression}\\
    \large{Programming II - Elixir Version}
}
\author{Johan Montelius}
\date{Spring Term 2023}
\maketitle
\defaultpagestyle


\section*{Introduction}

In this assignment you will evaluate a mathematical expression
containing variables.


\section*{Expressions}

Arithmetic expressions are represented as tuples {\tt \{op, arg1,
  arg2\}} and we can for the time being limit ourselves to the
operators {\tt :add}, {\tt :sub}, {\tt :mul} and integer division {\tt
  :div}.  Expressions are thus:

\begin{minted}{elixir}
@type expr() :: {:add, expr(), expr()} 
            | {:sub, expr(), expr()} 
            | {:mul, expr(), expr()}
            | {:div, expr(), expr()}             
            | literal()
\end{minted}

The literals that we will use are either integers, variables or rational numbers. To
make it explicit we choose to represent integers as {\tt \{:num, n\}}
and variables as {\tt \{:var, a\}}. Rational numbers are represented as {\tt \{:q, n, m\}}.

This gives us everything we need to represent a limited sets of
expressions. The expression $2x + 3 + 1/2$ could for example be represented
by the Elixir structure:

\begin{minted}{elixir}
{:add, {:add, {:mul, {:num, 2}, {:var, :x}}, {:num, 3}}, {:q, 1,2}}
\end{minted}

\noindent As you see it it is not a syntax we would like to use when we write
expressions by hand but it has its advantages when it comes to handle
the expressions using Elixir clauses. 

\section*{Evaluation}

You task is to implement a function {\tt eval/2}, that takes an
expression and an environment and evaluate the expression to a
literal. The environment is a mapping from variable names to values
and we expect to have values for all variables in the expression.

The environment should provide two functions, one to create a new
environment with a given set of bindings and one function that finds
a binding given a variable name. 

Once you have an environment working then the function {\tt eval/2}
should be done in fifteen minutes, this skeleton might give you a
flying start:

\begin{minted}{elixir}

  def eval({:num, ..}, ...) do ... end
  def eval({:var, ..}, ...) do ... end
  def eval({:add, ..., ...}, ...) do
    add(..., ...)
  end
    :
    : 
\end{minted}

If you follow this skeleton you only have to implement the function
{\tt add/2}, {\tt sub/2} etc. It seams like a trivial task and this is
why we threw in rational numbers and integer division. Dividing $5$ by
$5$ is thus not $2.5$ but $2/5$ or as we would represent it {\tt
  \{:q, 2, 5\}}. So when you implement the arithmetic operations you
have to take into account that one of the operands might be a rational
number.

In order to make tings more readable we of course want to reduce the
rational numbers as much as possible. If you evaluate $2 \times 3/4$
the answer should not be $6/4$ but $3/2$.

Implement the function {\tt eval/2} and show by some examples that it
works.


\end{document}
