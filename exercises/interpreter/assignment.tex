\documentclass[a4paper,11pt]{article}

\input ../include/preamble.tex

% Syntax-free grammar
\usepackage{syntax}


% SECTIONS
%
% * Introduction
% * The overall picture
% * The implementation
% * Extensions


\begin{document}

% ================================================== %
% == Title == %
% ================================================== %

\title{
    \textbf{A Meta-Interpreter}\\
    \large{Programming II - Elixir Version}
}
\author{Johan Montelius}
\date{Spring Term 2018}
\maketitle
\defaultpagestyle


% ================================================== %
% == Introduction == %
% ================================================== %

\section*{Introduction}

In this assignment you will implement an interpreter for a small
functional language. The language could be the functional subset of
Elixir but it does not match exactly how Elixir is defined (but maybe
it should). The exercise will hopefully give you a better
understanding of how a functional programming language is defined and
how powerful a functional programming language can be as tool; the
size of our interpreter is surprisingly small. 


% ================================================== %
% == The overall picture == %
% ================================================== %

\section{The overall picture}

Before staring the implementation you need to get an overall picture
of the goal and what is actually needed. Never start coding one piece
in a jigsaw puzzle if you don't know what the final picture should look
like or how the piece should fit in.

\subsection{A meta-interpreter}
An interpreter, in computer engineering terminology, is a program that
takes a program as input and executes the statements of the program to
produce a result. Some programming systems use only a interpreter to
execute programs but most systems today use a compiler to produce
machine code that is then executed either directly or in a
virtual machine. The advantage of an interpreter is that we do not
have to wait for a compilation phase, the disadvantage is that the
interpreter is much slower (typically a factor ten to one hundred).
Do some reading and find out if the programming languages that you
know uses a compiler or an interpreter.

A meta-interpreter, also called self-interpreter, is an interpreter
that is implemented in the language that it interprets. The reason for
this not obvious but if you have a compiled system and need an
interpreter of the same language, why not implement it in the language
itself - this is probably the language you know best. We will develop
an interpreter for a subset of the Elixir language. Since it is not the
complete language it is not a true meta-interpreter but it's close.

\subsection{Expressions}
Our target programming language is a limited functional language and
in its first incarnation it will be limited to simple expressions; We
will not have any function calls nor any program with function
definitions (you might not even call this much of language but it will
be a good starting point). This is an example of a {\em sequence} in
the language:

\begin{figure}[h]
\begin{minted}{elixir}
x = :foo; y = :nil; {z, _} = {:bar, :grk}; {x, {z, y}}
\end{minted}
\caption{a simple sequence}
\label{fig:sequence}
\end{figure}

A {\em sequence} consist of a, possibly empty, sequence of pattern
matching expression followed by a single expression. This is a bit
more restrictive compared to Elixir sequences but it does not limit
the expressiveness of our language. Simple expressions will for now be
limited to {\em terms}. If we use a BNF grammar we can describe a
sequence in the following way:

\begin{figure}[h]
\begin{grammar}
<sequence> ::= <expression> \alt <match> ';' <sequence>

<match> ::= <pattern> '=' <expression> 

<expression> ::= <atom> \alt <variable> \alt '\{' <expression> ',' <expression> '\}'
\end{grammar}
\end{figure}

\subsection{Terms, data structures and patterns}
Expressions of the form that we have just described are also referred
to as {\em terms}. We will later see examples of expression that are
more complicated and we will talk about expressions in general and
{\em term expressions} but for now we will simply call them {\em terms}.

It is important that we are clear on the terminology when we discuss
our language. There is a difference between {\em terms}, {\em
  patterns} and {\em data structures}. In the sequence previously
shown, the following constructs are terms: \verb+:foo+, \verb+:nil+,
\verb+{:bar, :grk}+ and, \verb+{x, {y, z}}+. Terms are syntactical
constructs that, {\em when evaluated}, will result in {\em data
  structures}; data structures are thus something that we handle
during execution, the terms are only recipes for how these structures
should be constructed.

If we {\em evaluate} the sequence above we will obtain the data
structure \textit{ \{foo, \{bar, nil\}\}}. We will in this
text use {\it italics} when we refer to the data structures and
\verb+teletype+ when we talk about the terms.

For now, terms are restricted to atoms, variables and the binary
compound {\em cons} structure. Once we understand how to handle this
subset it's fairly easy to extend the language.

The patterns in the sequence are: \verb+x+, \verb+y+ and,
\verb+{z, _}+. The syntax we use for patterns, is the same as the one
that we use for terms but we are also allowed to use an underscore
(\verb+_+) to represent a {\em don't-care} variable. This variable
acts as a placeholder for a data structure that we have no interest
in.

\begin{figure}[h]
\begin{grammar}
<pattern> ::= <atom>
\alt <variable>
\alt '_'  
\alt '\{' <pattern> ',' <pattern> '\}'
\end{grammar}
\end{figure}

The reason why we can use the same syntax for terms and patterns
without confusion is that it is clear from the grammatical rules of the
language if we refer to a pattern or a term. 

\subsection{Evaluation}
Our interpreter will take a sequence and evaluate the pattern matching
expressions one after the other; the result of the last expression is
the result of the whole sequence. When the evaluation starts the
interpreter will have en empty {\em environment} i.e. it knows of no
variable bindings. Each pattern matching expression will add variable
binding to the environment and the following expressions are then
evaluated in the new environment.
 
If we evaluate the sequence in figure \ref{fig:sequence} we gradually
build an environment, first we add the binding {\em x/foo}, then {\em
  y/nil} and then {\em z/bar}. The final expression is thus evaluated
in the following environment: {\em \{x/foo, y/nil, z/bar\}}. In this
environment the term \verb+{x, {z, y}}+ is evaluated to the data
structure {\em \{foo, \{bar, nil\}\}}.

\subsection{The architecture}
In order to implement our interpreter we need to solve the following
problems: we need to represent expressions and we need to implement an
environment. Once we have these pieces in place we can start to define
the rules for the interpreter.

Before you proceed, you should think this problem through. What do
sequences look like; assume that we will only handle sequences as the
one shown above? What are the elements of a pattern
matching expression, what is on the right side and what is on the left
side? How should we represent a term and does it have to be different
from the representation of a pattern? How should data structures be
represented?

Read this section through one more time, then start to sketch on your
representation. Write it down and then later compare it to the
representation proposed in this exercise. 

% ================================================== %
% == The implementation == %
% ================================================== %

\section{The implementation}

We will build the interpreter starting with the environment, that will
be implemented in a separate module. Then we will handle evaluation
of expressions, pattern matching and finally a sequence of
expressions.

\subsection{The environment}

Implementing an environment will be the simplest task that we have; an
environment is simply a mapping from variables to data structures. If
we assume that environments will be small, we can represent an
environment as a list of key-value tuples. The environment {\em
  \{x/foo, y/bar\}} could be represented as:
\verb+[{:x, :foo}, {:y, :bar}]+ . The variables are represented by
atoms, and we have here chosen to name them {\tt :x} and {\tt :y} but
we could as well have chosen other atoms ({\tt :x12}, {\tt
  :variable\_x}) or integers ({\tt 1} and {\tt 2}), the important
thing is that they all have unique names.

In a module {\tt Env}, you should now define the following functions:

\begin{itemize}
\item {\tt new()} : return an empty environment

\item {\tt add(id, str, env)} : return an environment where the
  binding of the variable {\tt id} to the structure {\tt str} has been
  added to the environment {\tt env}.

\item {\tt lookup(id, env)} : return either {\tt \{id, str\}}, if the
  variable {\tt id} was bound, or {\tt nil}

\item {\tt remove(ids, env)} : returns an environment where all
  bindings for variables in the list {\tt ids} have been removed
\end{itemize}

These are all operations we need from the environment module. Test the
environment by calling the functions from the Elixir shell to see that
it produces the expected results. The following
call should return {\tt \{:foo,  42\}}:

\begin{minted}{elixir}
Env.lookup(:foo, Env.add(:foo, 42, Env.new()))
\end{minted}

Note that Elixir will write a list of binary tuples in a special
way. It is so common that we work with key-value where the ``key'' is
an atom that Elixir uses a special syntax. The list {\tt [\{:x, 12\},
  \{:y, 13\}]} will be written {\tt [x: 12, y: 13]}. You also use this
for when you enter key-value list, it will save you some typing.

\subsection{Terms and patterns}

If we only needed to represent terms consisting of atoms and
cons-cells, things would be trivial. The term {\tt \{:a, :b\}} could
simply be represented by the Elixir term {\tt \{:a, :b\}}. The problem
is that we also need to represent variables. We will of
course not be able to represent a variable in our target language with
a variable in Elixir; we have to find a way to represent variables and
make sure that we can separate them from atoms.

One solution is to represent atoms with the tuple {\tt \{:atm, a\}} and
variables with the tuple {\tt \{:var, v\}}. The good things is then
that we only have to make sure that the identifiers of variables are
all different and that the identifiers of atoms are all different. An
atom could of course have the same identifier as a variable with out
causing any problems; the atom {\tt \{:atm, 123\}} is different from
the variable {\tt \{:var, 123\}}.

A cons cell could be represented by a tuple {\tt \{:cons, head,
  tail\}}. We could of course have chosen to represent cons cells as
Elixir cons cells but we want to make a distinction between the
representation of terms in our target language and terms in Elixir (our
target language now happens to be Elixir look-alike but that is of
course not always the situation).

As an exercise you can write down the representation of the term:

\begin{minted}{elixir}
{:a, {x, :b}}
\end{minted}

You will have to choose identifiers for the atoms {\tt :a}, {\tt :b}
(why not {\tt :a}, {\tt :b}) and the variable {\tt x} (why
not {\tt :x}).

The representation of patterns will be exactly the same as for terms
with the only difference that we need to represent the special {\em
  don't-care} pattern. We choose the atom {\tt :ignore} which will be
exactly what we will do when we encounter the symbol.

\subsection{Expressions}
In our simple language an expression is simply a term
expression. Expressions should be evaluated to data structures and the
question is of course how these data structures should be represented.

The nice thing with a meta-interpreter is that the data structures of
the interpreted language could be mapped directly to the data
structures of the implementation language. An atom will thus be
represented by an Elixir atom, and a cons structure of the Elixir tuple
 i.e. {\tt \{:a, :b\}}. 

Create a module called {\tt Eager} (for reasons that will be given
later) and implement a function {\tt eval\_expr/2} that takes an
expression and an environment and returns either {\tt \{:ok, str\}},
where {\tt str} is a data structure, or {\tt :error}. An error is
returned if the expression can not be evaluated. This should be a
quite simple task. The following skeleton code will get you started:

\begin{minted}{elixir}
def eval_expr({:atm, id}, ...) do ... end

def eval_expr({:var, id}, env) do
  case ... do
    nil ->
      ...
    {_, str} ->
      ...
  end
end

def eval_expr({:cons, ..., ...}, ...) do
  case eval_expr(..., ...) do
    :error ->
      ...
    {:ok, ...} ->
      case eval_expr(..., ...) do
        :error ->
          ...
        {:ok, ts} ->
          ...
      end
  end
end
\end{minted}

Here are some examples that you should be able to handle.

\begin{itemize}
\item {\tt eval\_expr(\{:atm, :a\}, [])} : returns {\tt \{:ok, :a\}}
 \item {\tt eval\_expr(\{:var, :x\},  [\{:x, :a\}])} : returns {\tt \{:ok, :a\}}
 \item {\tt eval\_expr(\{:var, :x\},  [])} : returns {\tt :error}
 \item {\tt eval\_expr(\{:cons, \{:var, :x\}, \{:atm, :b\}\},  [\{:x, :a\}])} : returns {\tt \{:ok, \{:a, :b\}\}} 
\end{itemize}

Note that {\tt eval\_expr/2} returns a data structure if successful
i.e. {\em a}, {\em foo}, or {\em \{a, b\}}.  The last test is an
effect of representing the binary tuples in our source program {\tt
  \{:a, :b\}}, with the internal representation {\tt \{:cons, \{:atm,
  :a\}, \{:atm, :b\}\}} that when evaluated will return the
data structure {\em \{a, b\}}.

\subsection{Pattern matching}
A pattern matching will take a pattern, a data structure and an
environment and return either {\tt \{:ok, env\}}, where {\tt env} is an
extended environment, or the atom {\tt :fail}. 

Implement a function {\tt eval\_match/3} that implements
the pattern matching. Some examples will give you an idea of what
we're looking for.

\begin{itemize}
 \item {\tt eval\_match(\{:atm, :a\}, :a, [])} : returns {\tt \{:ok, []\}}
 \item {\tt eval\_match(\{:var, :x\}, :a, [])} : returns {\tt \{:ok, [\{:x, :a\}]\}}
 \item {\tt eval\_match(\{:var, :x\}, :a, [\{:x, :a\}])} : returns {\tt \{:ok, [\{:x, :a\}]\}}
 \item {\tt eval\_match(\{:var, :x\}, :a, [\{:x, :b\}])} : returns {\tt :fail}
 \item {\tt eval\_match(\{:cons, \{:var, :x\}, \{:var, :x\}\}, \{:a, :b\}, [])} : returns {\tt :fail}
\end{itemize}

Solving the cases where the pattern is an atom or variable is quite
straight forward, especially since we already have the environment
module. The slightly more problematic case is when the pattern is a
cons structure. Note that we first would add a binding for {\tt :x} to
{\tt :a} and then try to match {\tt :x} with {\tt :b}. This will of
course fail, the variable {\em x} can not have two values.

The following skeleton code should lead you in the right direction. We
start with the simple cases, we can ignore the case where the atoms
do not match for now.

\begin{minted}{elixir}
def eval_match(:ignore, ..., ...) do
  {:ok, ...}
end

def eval_match({:atm, id}, ..., ...) do
  {:ok, ...}
end
\end{minted}

Matching a variable is only slightly more complicated, we check if it
has a value and if not we add it to the environment. This skeleton
code uses a special construct that you might not have seen before, the
{\tt \^{}str} variable. This is to indicate that we do not want a new
variable {\tt str} but rather use the existing variable.

\pagebreak

\begin{minted}{elixir}
def eval_match({:var, id}, str, env) do
  case ... do
    nil ->
      {:ok, ...}
    {_, ^str} ->
      {:ok, ...}
    {_, _} ->
      :fail
  end
end
\end{minted}

Now the complicated (not so complicated) case where we match a cons
pattern with a cons structure. This is where we must make sure that a
variable binding in one branch is transferred to the pattern matching
of the other branch.

\begin{minted}{elixir}
def eval_match({:cons, hp, tp}, ..., env) do
  case eval_match(..., ..., ...) do
    :fail ->
      ...
    ... ->
      eval_match(..., ..., ...)
  end
end
\end{minted}

And last but not least, if we can not match the pattern to the data
structure we fail.

\begin{minted}{elixir}
def eval_match(_, _, _) do
  :fail
end
\end{minted}

Complete the implementation and try the examples give before.

\subsection{Sequences}

We now have all the pieces of the puzzle to implement the evaluation
of a sequence. We represent a sequence as list, the first elements
will of course be pattern matching expressions but the last element is
of course a regular expression. The evaluation starts with an empty
environment that is extended as we proceed down the list.

Each pattern matching expressions is evaluated in two steps, first the
expression on the right hand side is evaluated returning a data
structure. The pattern on the left hand side is then match to the data
structure resulting in an extended environment.

Before we perform the match operation we have to create a new
environment that looks like the existing environment, with the
exeption that we have removed all bindings for variables that we find
in the pattern.  It is important to understand how the environment is
extended. The evaluation of the following sequence should result in
{\em \{c, b\}}.

\begin{minted}{elixir}
x = :a; y = :b; x = :c; {x, y}
\end{minted}

Here is some skeleton code that Will get you started. You need to
implement the function {\tt extract_vars/1} that returns a list of all
variables in the pattern.

\begin{minted}{elixir}
def eval_scope(..., ...) do
  Env.remove(extract_vars(...), ...)
end
\end{minted}

\begin{minted}{elixir}
def eval_seq([exp], env) do
  eval_expr(..., ...)
end

def eval_seq([{:match, ..., ...} | ...], ...) do
  case eval_expr(..., ...) do
    ... ->
      ...
    ... ->
      ... = eval_scope(..., ...)

      case eval_match(..., ..., ...) do
        :fail ->
          :error
        {:ok, env} ->
          eval_seq(..., ...)
      end
  end
end
\end{minted}

When you have everything in place you should define a function {\tt
  eval/1}, that takes a sequence and returns either {\tt \{:ok, str\}}
or {\tt :error}. You should then be able to run the following query:

\begin{minted}{elixir}
seq = [{:match, {:var, :x}, {:atm,:a}},
        {:match, {:var, :y}, {:cons, {:var, :x}, {:atm, :b}}},
        {:match, {:cons, :ignore, {:var, :z}}, {:var, :y}},
        {:var, z}]

Eager.eval(seq)
\end{minted}

The query is the representation of the following expression:

\begin{minted}{elixir}
x = :a; y = {x, :b}; {_, z} = y; z
\end{minted}


% ================================================== %
% == Extensions  == %
% ================================================== %

\section{Extensions}

We now have an interpreter that can handle sequences of expressions
but the expressions are rather simple. You should now extend the
language and the interpreter to handle: {\em case expressions}, {\em
  lambda expressions} and named functions. In each case you need to
think about how expressions are represented before thinking about how
the {\tt eval_expr/2} function is extended.

\subsection{Case expressions}
A case expression consists of an expression and a list of clauses
where each clause is a pattern and a sequence. We of course need to be
able to tell a case expression from any other expression so why not
represent it as a tuple with a {\tt :case} key word as the first
element. A clause is simply a tuple with the key word {\tt :clause}.

Now we extend the evaluation with a clause that can handle the case
expressions.

\begin{minted}{elixir}
def eval_expr({:case, expr, cls}, ...) do
  case eval_expr(..., ...) do
    ... ->
      ...
    ... ->
      eval_cls(..., ..., ...)
  end
end
\end{minted}

The function {\tt eval_cls/1} will take a list of
clauses, a data structure and an environment. It will select the right
clause and continue the execution.

\begin{minted}{elixir}
def eval_cls([], _, _, _) do
  :error
end

def eval_cls([{:clause, ptr, seq} | cls], ..., ...) do
  ...
  ...
  case ... do
    :fail ->
      eval_cls(..., ..., ...)

    {:ok, env} ->
      eval_seq(..., ...)
  end
end
\end{minted}

That's it, you should now be able to evaluate something like this:

\begin{minted}{elixir}
seq = [{:match, {:var, :x}, {:atm, :a}},
       {:case, {:var, :x},
          [{:clause, {:atm, :b}, [{:atm, :ops}]},
           {:clause, {:atm, :a}, [{:atm, :yes}]}
          ]}
       ]

Eager.eval_seq(seq, Env.new())
\end{minted}

\subsection{Lambda expressions}

Adding lambda expressions might look very complicated but it turns out
to be quite simple. We first have to find a representation but this is
by now not a problem. We know that a lambda expression (unnamed
function) consists of a sequence of parameter variables, a sequence of
free variables and a sequence expression. If we agree to represent the
parameters as well as the free variables as lists of identifiers we
are done.

\begin{minted}{elixir}
{:lambda, parameters, free, sequence}
\end{minted}

Now when we evaluate a lambda expression we need to represent a
closure but this is equally simple. A closure simply consists of a
sequence of parameter variables, a sequence expression together with an
environment.

\begin{minted}{elixir}
{:closure, parameters, sequence, environment}
\end{minted}

To evaluate a lambda expression we need to add a function to the {\tt
  Env} module that creates a new environment from a list of variable
identifiers and an existing environment. If we can do this the rest is
simple.

\begin{minted}{elixir}
def eval_expr({:lambda, par, free, seq}, ...) do
  case Env.closure(free, ...) do
    :error ->
      :error
    closure ->
      {:ok, {:closure, ..., ..., ...}}
  end
end
\end{minted}

The only thing we now have left is the function application i.e. when we
apply a closure to a sequence of argument expressions. We need a way
to represent this and the most natural way is the best. Note that we
have an expression in the structure. The closure is something we will
hopefully have as a result of evaluating the expression.

\begin{minted}{elixir}
{:apply, expression, arguments}
\end{minted}

The evaluation should first evaluate the expression. If this is indeed
a closure we can evaluate the arguments and apply the closure to the
resulting list of data structures. The function {\tt eval_args/2}
should return either {\tt :error} or {\tt \{:ok, strs\}} where {\tt
  strs} is a list of structures.

\begin{minted}{elixir}
def eval_expr({:apply, expr, args}, ...) do
  case ... do
    :error ->
      :error
    {:ok, {:closure, par, seq, closure}} ->
      case eval_args(..., ...)  do
        :error ->
          :error
        {:ok, strs} ->
          env = Env.args(par, strs, closure)
          eval_seq(seq, env)
      end
  end
end
\end{minted}

You have to extend the {\tt Env} module to include a function {\tt
  args/3} that is given a list of variable identifiers ({\tt par}) , a
list of data structures ({\tt strs}) and an environment (that includes
the values of all free variables). The environment that is returned
should now include bindings for all the variables in the sequence of
the closure. If you get it right we should be able to evaluate the
following.

\begin{minted}{elixir}
seq = [{:match, {:var, :x}, {:atm, :a}},
       {:match, {:var, :f}, 
          {:lambda, [:y], [:x], [{:cons, {:var, :x}, {:var, :y}}]}},
       {:apply, {:var, :f}, [{:atm, :b}]}
      ]
      
Eager.eval_seq(seq, Env.new())
\end{minted}

\subsection{Named functions}

You're now a small step from being able to handle named functions
i.e. a program. We will add these in a very simple way. For each named
function we will have a Elixir function that returns our
representation of the parameters and its sequence. We add one new
term, {\tt \{:fun, id\}}, to our languge that indicate that the
identifier is a name of a function. We can then lookup the definition
by making a fucntion call to the {\tt Prgm} module as shown in the code below:

\begin{minted}{elixir}
def eval_expr({:fun id}, env)  do
   {par, seq} = apply(Prgm, id, [])
   {:ok,  {:closure, par, [], seq}}
end
\end{minted}

Now we can define our wersion of {\tt append/2} in a module {\tt Prgm}:

\begin{minted}{elixir}
  defmodule Prgm do

  def append() do
    {[:x, :y],
      [{:case, {:var, :x}, 
        [{:clause, {:atm, []}, [{:var, :y}]},
         {:clause, {:cons, {:var, :hd}, {:var, :tl}}, 
          [{:cons, 
            {:var, :hd}, 
            {:call, :append, [{:var, :tl}, {:var, :y}]}}]
        }]
      }]
    }
  end
\end{minted}

If everything works you should now be able to evaluate the following:

\begin{minted}{elixir}
seq = [{:match, {:var, :x}, 
         {:cons, {:atm, :a}, {:cons, {:atm, :b}, {:atm, []}}}},
       {:match, {:var, :y}, 
          {:cons, {:atm, :c}, {:cons, {:atm, :d}, {:atm, []}}}},
       {:apply, {:fun, :append}, [{:var, :x}, {:var, :y}]}
      ]
    
Eager.eval_seq(seq, Env.new())
\end{minted}


\end{document}
