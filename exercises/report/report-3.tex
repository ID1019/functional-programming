\documentclass[a4paper,11pt]{article}

\usepackage[utf8]{inputenc}

\usepackage{graphicx}

\usepackage{minted}

\begin{document}

\title{
    \textbf{Your third report.}
}
\author{Your Name}
\date{Spring Term 2022}

\maketitle

\section*{Introduction}

Here are some comments on the reports that you uploaded and some
guide lines of what you should think of in the next report (and all
reports in the future of you life). All these tips and tricks are how
to use \LaTeX, it has nothing to do with Elixir.

\section*{Explaining your code}

The reports that you handed in last week look much better than the
first reports. Some still had a page of code in the middle of the
report and this is not the way to do it. When you describe you code,
you describe it in sections. This s for example the append function:

\begin{verbatim}
  def append([], y) do y end
  def append([h|t], y) do 
    [h | append(t, y)]
   end
\end{verbatim}

This is now when we explain why this piece of code is here and it is
because we want to talk about tail recursion. How do we implement
this in a tail recursive way? Hmm, tricky, can we do like this:

\begin{verbatim}
  def tailr([], y) do y end
  def tailr([h|t], y) do 
    tailr(t, [h|y])
  end
\end{verbatim}

No, this will of course not work since the elements will occur in the
wrong order. If we should use this then we would have to start with a
reversed list .... hmm, why not do this?

\begin{verbatim}
  def append(x, y) do 
    rev = reverse(x)
    tailr(rev, y) 
  end
\end{verbatim}

Ahh, so now the problem is to write reverse in a tail-recursive way
but you know how to do this so - problem solved. Explaining you code
like this is better than listing the code in the end. 

\section*{Tables}

You will now start to present numbers and one way is to present them
in a table. You will have to do some reading on how to format table bu
the general structures is quite easy. This is for example a table with
some run-time figures.

\begin{table}[h]
\begin{center}
\begin{tabular}{l|c|c}
\textbf{prgm} & \textbf{runtime} & \textbf{ratio}\\
\hline
  dummy      &  115 &     1.0\\
  union      &  535 &     4.6\\
  tailr      &  420 &     3.6\\
\end{tabular}
\caption{Union and friends, list of 50000 elements, runtime in microseconds}
\label{tab:table1}
\end{center}
\end{table}

As I said in the previous note you should be careful with how many
significant figures you use. If you say that something took
$56.123456$ microseconds I can guarantee that the next time you try
the measurement will be very different. It might be $56.2....$ or
$55.9....$ so why write $56.123456$?  Why not write $56$ microseconds?

The other thing, as you see in the table above, the runtime per se
might not be interesting. The interesting thing is how it relates to
something else. Look at the ratio above, it gives you the information
that we are looking for.

So when you include numbers, ask your self why you have these numbers
in the report. What is the purpose, can you describe this in a better
way?

\section*{Graphs}

Since you will present numbers you of course want to include a graph
in your report. There are many ways to generate graphs but you want to
use a way that minimize manual work. My tool over the years has been
{\em Gnuplot} and if you do not have a favorite tool you could give it
a try.

Gnuplot is not a statistical program or a program that is very good at
manipulating numbers but it is very good at taking a sequence of
numbers and generate a nice graph.

To show you how to do, let's use an example - why not runtime of
Fibonacci. The first thing wee need is a benchmark program that writes
a sequence runtimes. We will use two tab separated columns, on that
gives us $n$ and one that gives us the runtime of $fib(n)$. 

\begin{minted}{elixir}
  def bench(l) do
    {:ok, file} = File.open("fib.dat", [:write, :list])
    seq = [12,14,16,18,20,22,24,26,28]
    :io.format(file, "# This is a comment since it starts with #\n", [])
    :io.format(file, "# Fibonacchi of n, time in us\n", [])
    :io.format(file, "# n\ttime\n", [])
    Enum.each(seq, fn n -> bench(l, n, file) end)
    File.close(file)
  end
\end{minted}

The {\tt bench/1} function will open a file, print some comments and
then call {\tt bench(l, n, file)} for all $n$ in the sequence {\tt
  [12,14 ..]}.  The {\tt bench/3} function will call {\tt loop(l,n)}
and print out the $n$ and the time it took to calculate {\tt
  fib(n)}. 

\begin{minted}{elixir}
  def bench(l, n, file) do
    {t, _} = :timer.tc(fn -> loop(l, n) end)
    :io.format(file, "~w\t~.2f\n", [n, t/l])
  end
\end{minted}

The function {\tt loop(l,n)} is responsible for calling {\tt fib(n)}
$l$ times and this could of course be done like this.

\begin{minted}{elixir}
  def loop(0,_) do :ok end
  def loop(l,n) do 
    fib(n)
    loop(l-1, n)
  end
  
  def fib(1) do 1 end
  def fib(2) do 1 end
  def fib(n) do fib(n-1) + fib(n-2) end
\end{minted}

If everything works, then the file {\tt fib.dat} will look something like this:

\begin{verbatim}
# This is a comment since it starts with #
# Fibonacchi of n, time in us
# n     time
12      3.07
14      6.92
16      17.82
 :
\end{verbatim}

Now comes the interesting part, how do we turn this into a nice plot?
You should first install Gnuplot and I'm sure you can do this with
out my help. Once you have Gnuplot installed we write a small script in
{\tt fib.p} that looks like this:

\begin{verbatim}
set terminal pdf
set output "fib.pdf"

set title ""

set xlabel "n"
set ylabel "time in us"

plot "fib.dat" u 1:2 with lines title "runtime fib(n)"
\end{verbatim}

There are of course tons of things you can do in this script but here
we basically say that we want to generate a pdf of the graph and that
the plot should use the first and second column in the file {\tt
  fib.dat}.  If you now have Gnuplot installed you can run it from a terminal:

\begin{verbatim}
 $> gnuplot fib.p
\end{verbatim}

If you now look in your folder you will have a file called {\tt
  fib.pdf} that is the graph. This is now included in the report like
this (you need to have {\tt \\usepackage\{graphicx\}} in your report):

\begin{figure}[H]
  \center
  \includegraphics[]{fib.pdf}
  \caption{This is a fib graph.}
  \label{fig:fib}
\end{figure}

We can probably do better than this nut now we have a very easy way to
generate a new graph in our report. If you run the benchmark, run
gnuplot and then generate the report you will have new graph in less
than 20 seconds.



\end{document}
