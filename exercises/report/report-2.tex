\documentclass[a4paper,11pt]{article}

\usepackage[utf8]{inputenc}

\usepackage{minted}

\begin{document}

\title{
    \textbf{Your second report.}
}
\author{Your Name}
\date{Spring Term 2022}

\maketitle

\section*{Introduction}

Here are some comments on the reports that you uploaded and some
guide lines of what you should think of in the next report (and all
reports in the future of you life). All these tips and tricks are how
to use \LaTeX, it has nothing to do with Elixir.

First tip, the name of the report should not be ``My first report''
and the name should not be ``My Name''.

\section*{Read the instructions}

Almost all of you handed in a report written in LaTeX but of course
there were two that handed in something written i Word or
Notepad. Only two out of a hundred and fifty is still very good, well done.

The instructions said ``not four pages of code'' most of you got the
meaning. My idea is that you will learn a lot more if you have to
explain you code to someone else. It one thing to get the code to work
but explaining what you did is more complicated. When explaining
something one often realize why things work as they do, or in the best
case realize that it does not work.

Explaining something also has the advantage that the reader gets
another perspective on how the assignment can be solved and the
arguments why it should be solved like this.

\section*{Sections}

I suggested that since these are all small reports, numbering of
sections could be omitted. In LaTeX you create a section header
either using

\begin{verbatim}
  \section\{something}
\end{verbatim}
or

\begin{verbatim}
  \section*{something}
\end{verbatim}

The difference is that when you use the {\tt *} version, no number is
added to the header. You also use this for sub-sections i.e.

\begin{verbatim}
  \subsection{something}
\end{verbatim}

or

\begin{verbatim}
  \subsection*{something}
\end{verbatim}

If you don't use a number on your section header it looks strange to
number the sub-sections.

\section*{Inserting code}

Code snippets are best included using the package {\tt minted} as I
said. This packade will take care of most formating issues and also
color the code. You still need to indent the code so in the example
below I've done the indentation explicitly.

\begin{minted}{elixir}
def append([], y) do y end
def append([h|t], y) do
  [h | append(t, y)]
end
\end{minted}

You could use other packages to format code and one polular is {\tt
  lstlisting}. I don't think there is a default elixir mode for this
and I thinsk minted does a very good job so why not use minted. 

You could also use the simple {\tt verbatim} environment but thenn you
have to be careful. LaTeX ignores {\tt tab} characters, it will treat
them as null characters and this could lead to strange things. The two
pieces of code below looks identical in the source code. The first one
is formatted using space character but the second is using a tab
character. As you can see the generated output is quite different.

\begin{verbatim}
   def append([], y) do y end
   def append([h|t], y) do
        [h | append(t, y)]
   end
\end{verbatim}

\begin{verbatim}
   def append([], y) do y end
   def append([h|t], y) do
	[h | append(t, y)]
   end
\end{verbatim}

\section*{less than}

If you in your LaTeX code write 5 \textless\ 7 it will look like 5 <
7 and 9 \textgreater\ 7 will look like 9 > 7. Using the characters
\textless\ and \textgreater\ directly does not work ... so, how did I
do it?  I used the commands {\tt  \textbackslash textless} and {\tt
  \textbackslash textgreater} to generate the symbols \textless\ and
\textgreater.

You could also use {\tt \{\textbackslash tt 5 < 7\}} but then it
will use the teletype font and look like this: {\tt 5 < 7}.

Still another way is to write it using so called {\tt math mode}. This
is a mode used for writing mathematical formulas in a nice way. You
enclose your expression in {\tt \$} signes like this {\tt \$5 < 7\$}
and then it will look like this $5 < 7$.

If you have a larger mathematical expression you enclose it in double
\$ and the result is that it is written centered with some space
around it like this:  $$ 5 < (3 * 8 / 3 ) $$


\section*{no f*ing screen shots}

I know that you are all very happy that things actually work and
eagerly want to show what things look like on you screen but please,
don't use screen shots. It looks ugly and it's impossible to mark or
copy the things that you want to show. It also, most often, show a lot
of irrelevant things so instead of using an image, copy the text and
format it so it's easy to read.


\section*{standard layout}

The rows in regular article mode are short - because it makes it
easier to read. Do not set the column width or margins explicitly, let
LaTeX decide what it should look like.


\section*{numbers}

You will soon include some run-time measurements in your reports. You
should the think about the number of significant figures that you
use. Just because a benchmark took $1.2345678 s$ does not mean that
you should report it in this way. If you write this in your report
you're implicitly saying - if I do this again the number will be the
same. This could be true but I doubt that anything you do on a
computer can be determined with an 8 figure accuracy. The next time
you try it might very well take $1.2354678 s$. What you report is
maybe $1.235 s$ or $1.2 s$?

\section*{why strange font}

If you want to wite {\em foo} in teletype font you wite
\verb+{\tt foo}+ like this {\tt foo But now if you forget the closing
  \} character everything from now on will be written in this
  font......

  I hope that these tips and tricks will help you write an even better
  report.

\end{document}
