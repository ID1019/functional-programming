\documentclass[a4paper,11pt]{article}

\input ../include/preamble.tex


\begin{document}


\title{Queues}
}
\author{Johan Montelius}
\date{Spring Term 2024}

\maketitle

\defaultpagestyle

\section*{Introduction}

This is an exercise where you implement different versions of
queues. The first implementation is the most natural but the
performance is not very good as the number of elements increase. The
second version is much better and as you will see the performance is
rather good.


\section*{The first queue}


To implement a queue we need a data structure where we can easily
implement the following functions:

\begin{itemize}
  \item {\tt @enqueue( any(),  queue()) :: queue()} : add a new item to the queue, place it "last".
  \item {\tt @dequeue( queue()) :: \{ any(), queue()\}} : remove the "first" element, return a tuple with the element and the remaining queue
  \item [\tt @empty( queue()) :: boolean()} : is the queue empty?
  \item {\tt @new() :: queue()} : return an empty queue.
\end{itemize}

Since we don't know how many elements will have in the queue we might
take the easy way out and represent the queue simply as a tuple with a
list of items.

\begin{minted}{elixir}
  @type queue() :: {:queue, [any()]}
\end{minted}


Given this representation implement the three functions above in a
module called {\tt Queue1}. What should we do if we try to dequeue an
element from an empty queue? Should we take care of this and return
something that the caller can recognize or should we simply crash?


Let's implement a small benchmark program to see how our queue
behaves. The code below is prepared to work both with the current {\tt
  Queue1} module and a future {\tt Queue2} module.

\begin{minted}{elixir}
  defmodule Bench do

  def time(module, n) do
    :timer.tc( fn() -> test(module, n) end)
  end

  def test(module, n) do
    q = module.new()
    q = enqueue(module, q, n)
    dequeue(module, q)
  end

   :
\end{minted}

As you see we have the module name as a parameter to the {\tt test/2}
function and use it to call the right module in for example {\tt
  module.new()}.  Implement the recursive functions {\tt enqueue/3}
and {\tt dequeue/2} that enqueues {\tt n} dummy elements and then
dequeues elements as long as the queue is not empty.

What is the run-time complexity of the functions?


\section*{A second try}

If you were given the task to implement a better queue data structure
you would either use an array or a linked list with a reference to the
last node. Both of those solutions are however not possible to
implement in Elixir since we are not allowed to modify data structures.

A solution to our problem could be to represent a queue using two
lists. The first list is used when we remove elements from the queue
and we then of course remove the first item in the list. The second
list is used when we add elements to the queue and now, instead of
adding an item to the end we add it as the first element.

\begin{minted}{elixir}
  @type queue() :: {:queue, [any()], [any()]}
\end{minted}

This does of course sound very strange but it will soon be clear why
this works. Let's first implement a solution that almost works.

\begin{minted}{elixir}
  defmodule Queue2 do

  def enqueue(itm, {:queue, front, back}) do
      {:queue, front, [itm|back]}
  end
  
  def dequeue({:queue, [itm|rest], back}) do
      (itm, {:queue, rest, back})
  end
  
   :

 \end{minted}

 This works as long as we have items in the first list that we can
 dequeue. If the list is empty will will now crash even if we have
 items in the second list. How should we get hold of these items,
 remember that they are in the "wrong" order i.e. that element that we
 would like to remove is in the back of the list.

 Removing a single item from the end of the second list would of
 course work but there is a better solution. Why no reverse the whole
 list and treat it as it was the first list?

 
 \begin{minted}{elixir}
  def dequeue({:queue, [], back}) do
     case reverse(back) do
        [itm|rest] ->
               :
     end
  end
  def dequeue({:queue, [itm|rest], back}) do
      (itm, {:queue, rest, back})
  end
 \end{minted}
   
 If you get it to work do some benchmarks using the same benchmark
 program as before. How does the new implementation of our queue
 compare? What is the run-time complexity of the functions now?


 












\end{document}
