\documentclass[a4paper,11pt]{article}

\input ../include/preamble.tex

\usepackage{syntax}


\begin{document}

\title{
    \textbf{ Semantics }\\
}
\author{Johan Montelius}
\date{Spring Term 2020}
\maketitle
\defaultpagestyle

\section*{Introduction}

This is an attempt to clarify how our miniature programming language is
given an interpretation by its relation to the lambda calculus and
being defined by its operational semantics. You should have read both
about the lambda calculus and the operational semantics, this text is
meant to tie the knots together.


\section{why the lambda calculus}

The lambda calculus is the foundation of any functional programming
language. By translating programming constructs to lambda expressions
we give a precise meaning to the constructs i our language. A pure
functional programming language does not have any constructs that can
not be described as lambda expressions.  Many languages do however
introduce extensions that fall outside the scope of lambda
calculus. Extending a functional programming language beyond the
lambda calculus is fine but we should know which constructs that
violates the rules and have a good reason to include them.

If we have the expression below, we could argue what the result should
be.

\begin{figure}[h]
\centering
\verb|  y = 3; f = fn x -> x + y; y = 2; f.(y) |
\end{figure}

If we can give it an interpretation as a lambda expression we will
remove any ambiguities (or be confused on a higher level).

$$(\lambda y \rightarrow (f \rightarrow  (\lambda y \rightarrow f \quad y) \quad 2) (\lambda x \rightarrow x + y)) \quad 3 $$

In this course our focus is not on the lambda calculus and we will not
dig deeper into how we can define arithmetic nor data structures such
as atoms or tuples. We do not explain how pattern matching can be
described but you should have seen the lambda calculus and understand
its relation to our functional programming language.


\section{the operational semantics}

Even if we could give a complete description of all our language
constructs in terms of lambda expressions we would not have a precise
definition of our language. The problem is that even though the lambda
calculus clearly defines what the result should be if we had infinite
amount of time, it does not define in what order things are done.

Since we are interested in not only the final result of a computation
but also the time and memory needed to perform the execution, we need
something more. This is where our operational semantics comes in.

The rules of our operational semantics leaves less doubt about what
will actually happen during execution. The lambda calculus simply
describe what is allowed but not i what order things should be
done. The beauty of the lambda calculus is that if one evaluation
strategy results in an answer so will any evaluation strategy \ldots
if it terminates.

The operational semantics should be precise enough for a programmer to
control the execution in order to avoid inifinte computations and
estimate time and memory complexity. We could still allow alternative
execution orders but it should be clear to the programmer when these
alternatives exist.





\end{document}
