\documentclass[a4paper,11pt]{article}

\input ../include/preamble.tex

\begin{document}

\title{
    \textbf{The last assignment}\\
    \large{ID1019 Programming II}
}
\author{Johan Montelius}
\date{Spring Term 2022}
\maketitle
\defaultpagestyle


\section*{Introduction}

Morse codes were used in the days of telegraphs and manual radio
communication. It is similar in the idea to Huffman coding in that it
uses fewer symbols for frequent occurring characters and more symbols
for the infrequent ones. You task is to write an encoder and a decoder
for Morse signals, encode your name and decode two secret messages.

{\bf The solution that you present should be your original code i.e. not
  copied from any source apart from the More code tree in appendix. }

\section{Morse codes}

There are several standards for Morse codes and we will here use a
slightly extended version since we also want to code some special
character. The Morse code uses, as you probably know, long and short
(often pronounced {\em di} and {\em da}) to encode characters. You
might therefore think that is identical to Huffman codes but there is
a difference. In Morse coding we have a special signal that tells us
when one character ends and the next start. The pause between
characters is necessary in order to decode a message.

The code for 'a' is {\em di-da}, 'i' is {\em di-di} and 'l' is {\em di-da-di-di}.  If
we just had the sequence {\em di-da-di-di} we would not know if this
was ``ai'' or ``j''; we need a third signal, the pause, to tell the
difference. A sequence {di-da-pause-di-di-pause} is then decoded as ``ai''.

How does this change the structure of our decoding tree? In a Huffman
tree we only have characters in the leafs and when we hit a leaf we
know that we have a complete character and can start from the root
again. In a Morse tree we can finish anywhere along the path to a
leaf. We thus have characters in the nodes of the tree (not all).

\section{Encoding}

Your first task is to implement an encoder. The Morse codes are given
in the appendix so the only thing that you need to to is to use this
information to construct an encoding table that can then be used to
encode a message.

If you look in the appendix the Mores codes are given represented as a
tree where one branch represents a short signal '.' and one branch the
long signal '-'. You should use this table since it includes some
special characters not normally found in the Morse alphabet. The
character of the node is the ASCII value of a character or \verb+ :na+
if no character has the corresponding code. Empty branches are
represented by \verb+ nil+.

\begin{minted}{elixir}
{:node, character, long, short}
\end{minted}

The More trees starts like follows; a single long signal is the code
for ASCII 116 which is a 't' and a single short is the code for 101
i.e. an 'e'.

\begin{minted}{elixir}
  {:node, :na,
    {:node, 116,  ..., ...},
    {:node, 101,  ..., ...}}
\end{minted}

The tree is not the best structure to use when you encode a message so
your first task is to transform this into a form that gives you a
$O(lg(k))$ or $O(1)$ lookup operation (where $k$ is the number of
letters in the alphabet). Describe why you have chosen the
representation that you have and its characteristics.

Once you have an encoding table you should implement an encoder that
can encode a charlist (a list of ASCII values) as a Mores code
represented as a charlist of short '.', long '-' and pause ' ' characters.

The encoder should have a complexity of $O(n * m)$ where $n$ is the
length of the message and $m$ is the length of the Morse codes. Since
messages might be very long you should provide an implementation that
does not use stack space proportional to the length of the
message. This means that you need to provide a tail recursive
solution.

Describe your encoder and why it meets the requirements and include
the code in your report.

Show that your encoder works by encoding you name (and if you have
non-ASCII characters in your name replace them with something close
i.e. 'è' becomes 'e' etc)

\section{Decoding}

Your next task is to implement a decoder that can take a charlist of a
Mores coded message and return the clear text message. You will of
course first need an decoding table and the table that you constructed
for the encoder might not be your best choice.

The table that you construct should give you a lookup operation that
has an $O(m)$ complexity i.e. it should be proportional to the length
of the Morse code. This means that you can not represent your table as
for example a list of all the codes and their corresponding characters
since this would require and $O(k)$ operation to search through all
the codes and an $O(m)$ operation to see if a code matches the code
that you're looking for. The situation would improve somewhat if you
choose to represent that table as a map structure but then the lookup
operation would be $O(lg(k))$ which is good but does not take
advantage of the fact that frequent occurring characters have short
codes (very much the way Huffman coding works).

So what is your choice of decoding table and what does your lookup
function look like?

Once you have the lookup operation you can implement the
decoder. Since the Morse encoded messages of course could be very long
your solution should not use stack space that is proportional to the
length of the messages. Describe your decoder and why it meets the
requirements, include the implementation in your report.

Decode the secret messages below. If you cut and paste the code, make
sure that you don't have carriage-return etc in the string. The string
should only contain the dash, dot and space characters. 

\begin{verbatim}
  '.- .-.. .-.. ..-- -.-- --- ..- .-. ..-- 
    -... .- ... . ..-- .- .-. . ..-- 
    -... . .-.. --- -. --. ..-- - --- ..-- ..- ... '
\end{verbatim}

\begin{verbatim}
  '.... - - .--. ... ---... .----- .----- .-- .-- 
    .-- .-.-.- -.-- --- ..- - ..- -... . .-.-.- -.-. 
    --- -- .----- .-- .- - -.-. .... ..--.. ...- .----. 
    -.. .--.-- ..... .---- .-- ....- .-- ----. .--.-- 
    ..... --... --. .--.-- ..... ---.. -.-. .--.-- 
    ..... .---- '
\end{verbatim}


\pagebreak

\section*{The Morse codes}

\begin{minted}{elixir}
def morse() do
  {:node, :na,
    {:node, 116,
      {:node, 109,
        {:node, 111,
          {:node, :na, {:node, 48, nil, nil}, {:node, 57, nil, nil}},
          {:node, :na, nil, {:node, 56, nil, {:node, 58, nil, nil}}}},
        {:node, 103,
          {:node, 113, nil, nil},
          {:node, 122,
            {:node, :na, {:node, 44, nil, nil}, nil},
            {:node, 55, nil, nil}}}},
      {:node, 110,
        {:node, 107, {:node, 121, nil, nil}, {:node, 99, nil, nil}},
        {:node, 100,
          {:node, 120, nil, nil},
          {:node, 98, nil, {:node, 54, {:node, 45, nil, nil}, nil}}}}},
    {:node, 101,
      {:node, 97,
        {:node, 119,
          {:node, 106,
            {:node, 49, {:node, 47, nil, nil}, {:node, 61, nil, nil}},
            nil},
          {:node, 112,
            {:node, :na, {:node, 37, nil, nil}, {:node, 64, nil, nil}},
            nil}},
        {:node, 114,
          {:node, :na, nil, {:node, :na, {:node, 46, nil, nil}, nil}},
          {:node, 108, nil, nil}}},
      {:node, 105,
        {:node, 117,
          {:node, 32,
            {:node, 50, nil, nil},
            {:node, :na, nil, {:node, 63, nil, nil}}},
          {:node, 102, nil, nil}},
        {:node, 115,
          {:node, 118, {:node, 51, nil, nil}, nil},
          {:node, 104, {:node, 52, nil, nil}, {:node, 53, nil, nil}}}}}}
end
\end{minted}

\pagebreak


\end{document}
