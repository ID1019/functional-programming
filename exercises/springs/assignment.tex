\documentclass[a4paper,11pt]{article}

\input ../include/preamble.tex


\begin{document}


\title{Hot springs}

\author{Johan Montelius}
\date{Spring Term 2024}

\maketitle

\defaultpagestyle

\section*{Introduction}

This is the Advent of code for day 12 in 2023. It's a puzzle involving
a field of hot springs with unknown condition. The first task is a bit
tricky but you should solve it without any problems. The second task
is impossible to solve if you don't think things through and use
dynamic programming.


\section{The record of springs}

Imagine that your standing in a large field with rows of hot
springs. Some springs are hot, some are cold and some are a biut
unknown. The springs are heated by hot lava and should be either
operational or damaged but it's not easy to tell. There is however a
statis description that will tell you the condition of the springs in
a row; your problem is that the description is incomplete of
springs. Luckily for you there is some redundancy in the description
so even from a incomplete description you should be able to work out
at least the probable status of the springs in a row.

A complete description of a row of springs is a string of {\tt \#}
(damaged) or {\tt .} (operational) followed by a sequence of numbers
stating how many consecutive damaged springs we have. A
description of a row could look as follows:

\begin{verbatim}
#.#.### 1,1,3
\end{verbatim}

So here we see that we only have two operational springs, the second
and fourth. We have three segments of damaged springs of length: one,
one and three. If all descriptions were as clear as this example we
would not have a problem but you're now faced with descriptions where
the status could be reported as unknown: {\tt ?}.

The good thing is that the sequence with numbers that describes the
consecutive numbers of damaged springs is always correct. Given an
incomplete description you could work out possible correct descriptions. 
Assume that your give a description that looks as follows:

\begin{verbatim}
???.### 1,1,3
\end{verbatim}

Even though the first three springs are recorded as unknown, there is
only one way the status of the springs would generate the sequence:
one, one, three. The only possible solution is that the first and
third spring are both damaged and that the second spring is
operational.  Note that the state "\#\#..\#\#\#" does not match
"1,1,3" but "2,3". A consecutive sequence of damaged springs i
recorded as one sequence.

If you're given a description that looks like follows:

\begin{verbatim}
.??..??...?##. 1,1,3
\end{verbatim}

Then the damaged springs could be either: second and sixth or second
and seventh or ... All in all you have four different descriptions
that match the incomplete description.

This is a small sample that you can work with to make sure that
everything works:

\begin{verbatim}
???.### 1,1,3
.??..??...?##. 1,1,3
?#?#?#?#?#?#?#? 1,3,1,6
????.#...#... 4,1,1
????.######..#####. 1,6,5
?###???????? 3,2,1
\end{verbatim}

If you get it right the first row has 1 solution, the second 4, then
1, 1, 4 and 10. The real problem much larger so you need to find a
good representation of the problem and a strategy to count the number
of possible descriptions - add them all together, row by row and
present the grand total.

You first task is to parse the description of the wells and since all
rows have the same structure it should be fairly easy. Begin by
splitting the string into rows and then split a row into two parts
(separated by " "). The description of the status is best first turned
into a {\em charlist} using {\tt String.to_charlist\textbackslash 1}
and then turned into something more practical. The sequence of damaged
springs is first divided in by splitting the string using "," as a
separated. Each number can then be parsed using {\tt
  Integer.parse\textbackslash 1}

The second task is of course to come up with an algorithm that solves
the problem. The solution might be easier than you think, if you think
recursively. First build something that checks if a {\em complete
  description} is consistent i.e. if the sequence of description of
springs matches the sequence of consecutive damaged.

When you have this figured out you start to think what happens when
you see a unknown state. The spring could be either operational or
damaged so you need to explore both of these options. When you explore
an option you should report back if it succeeded, or rather how many
possible interpretations we have. Once you get the recursive thinking
in order the solution is surprisingly simple. 

\section{Oh, no!}

The second task at first looks simple because the only thing you have
to do is extend the input and then use the same algorithm. You should
do this first and report the time it takes to solve the puzzle. Only
when you realize that things will take too long time should you
continue to solve the puzzle. 

\subsection{extending a row}

The thing is that the descriptions that you have been given are not the
full description. The sequences should be repeated five times with a
{\tt ?} in between. The sample row that we looked at before is
actually:

\begin{verbatim}
???.###????.###????.###????.###????.### 1,1,3,1,1,3,1,1,3,1,1,3,1,1,3
\end{verbatim}

You should first implement a function that multiplies the original
description $n$ times. Now try to solve the puzzle when multiplying
by: 1, 2, 3, etc. How large puzzles can you solve, what is happening
with the execution time? Did you have the patient to solve the puzzle
for a multiple of four? Give a rough estimate how long time it would
take to solve it for a multiple of five.

\subsection{dynamic programming to the rescue}

This is where dynamic programming comes to our rescue. The dynamic
programming solution is simply the same recursive function that you
have but extended with a memory to avoid doing the same computations
twice (or a hundred times). If you get it right you will not only
reduce the execution time but completely change the algorithmic
complexity.

The strategy that you should apply is to first extend your function
with an extra argument which will be your {\em memory}. You should
also introduce a new function that takes the same arguments and only
calls the original function. You then change your first function to
call this new function in its recursive call.

So far you have not done anything magical and you should be able to run
your program as before. The only difference is that we have the new
function as an extra step in each recursion. This is the function that
you now should extend so that it first checks if you have already an
answer for the query, or if you have to calculate it. If you calculate
it then store the query and the answer in the memory and return {\bf
  both the answer and the updated memory}.

Since the new function returns both the answer and the updated memory
you need to change your first function. The answer should of course be
used as before but the updated memory should be used in the recursion.

The memory that you use could of course be implemented specifically for
this task but you can in this assignment use an Elixir map. You will
still have the problem of figuring out what to use as a key etc but
the importance here is not how the key value store is implemented.

If everything works you should now be able to call your new function
with the problem and an empty map. Now try to solve the puzzle with a
multiplier of: 1, 2, 3 etc. If it works do some benchmarking and
describe the execution time in relation to the multiplier.  How large
puzzles can you solve?



\end{document}
