\documentclass[a4paper,11pt]{article}

\input ../include/preamble.tex


\begin{document}


\title{A parallel Mandelbrot}

\author{Johan Montelius}
\date{Spring Term 2024}

\maketitle

\defaultpagestyle

\section*{Introduction}

This assignment is a continuation on the Mandelbrot assignment. You
should do this first since the task in this assignment is how to
parallelize the computation.


\section*{What do we have}

You should have one module called {\tt Brot} that exports the
functions {\tt mandelbort/2}. This is the function that computes the
depth given a complex number. If you had done some statistics you
would have realized that 99\% of the computation time is spent
in this function. The remaining functions are simply there to decide
which complex numbers that should be computed and how a depth value is
turned into a color.

You also have a module called {\tt Mandel} that is responsible for
starting the calculations, gathering all depth values and turning them
into a color code. This is the part of the program that you should
parallelize.

The crucial observation is that all depth calculations are independent
of each other. We can do them in any order and we can do them in
parallel. Since we are working in a functional programming language
this is easier to determine, non of the calculation can change a
global state that hides a dependency. 


\section*{Let's spawn some work}

In order to execute things in parallel we need to start concurrent
threads. All though there is no need for a concurrent computation,
we're breaking it up into concurrent task to allow the underlying
machinery to execute them on multiple cores.

To spawn a computation is easy but we need a way to get hold of the
computed result. A general strategy is to spawn a process that
computes something and then sends the result as a message to its
creator. Take a look at this:

\begin{minted}{elixir}
  
  def add(a, b) do
     me = self()
     spawn( fn() -> add(a, b, me) end)
     receive do
       {:result, res} ->
          res
     end
  end

  def add(a, b, mother) do
    send(mother, {:result, a+b})
  end
\end{minted}

This program will spawn a computation that will add two number and
return the result. This program does of course not run any faster than
the plain version since: 1/ the overhead of creating the thread over
shadows any gains and 2/ the mother thread does not do anything
besides waiting for the result.

You task is to spawn many computations as quickly as possible and then
collect their results. Since we are generating an image you need to
keep track of which computations belong to which computations. You
also need to figure out what to spawn. In the above examples I don't
think we would ever be able to see a speed up since the work of adding
two numbers is less than spawning and collecting the result. The
computations that you should spawn need to be large enough to make it
worth wile.





\end{document}
