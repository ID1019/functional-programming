\documentclass[a4paper,11pt]{article}

\input ../include/preamble.tex


\begin{document}


\title{Strings}

\author{Johan Montelius}
\date{Spring Term 20xx}

\maketitle

\defaultpagestyle

\section*{Introduction}

There are two ways to represent strings in Elixir and you should get
use to using both and knowing when to prefere one over the other. In
order to learn how to use strings you should do some performance
measurments to uderstand the complexity of different operations.

\section*{The history}

The reason Elixir has two ways to represent strings is historical and
dates back to how Erlang represented strings. The Erlang system was
not targeting text processing so strings were simply respresented as a
linked list of integers that represnted the characters of the
string. This representation allows great flexibility and did not
require any additional data structures.

As the Erlang system matured a new data structure, a byte sequence or
{\em binary}, was introduced. The binary data structure allowed for
efficient representation of large data structures and provided
efficient access operations. Once byte sequences where introduced
these were an obviuous form to represent strings; think about an array
of bytes compared to a linked list of integers.

In Erlang a string was written using quatations marks ({\tt "ABC"})
and was represented as a linked list of integers. Binaries were
written using {\em bit syntax} ({\tt <<65,66,67>>}) and libraries
operating on strings would accept either. To make things even more
flexible srings could be encoded as: linked lists och characyters,
binaries or a linked list of a mixture.

When Elixir was introduced, an Elixir uses Ruby syntax, a slight
change was made to the syntax. Since the byte sequence representation
is more efficient (and maybe what programmers expect), the quoted
syntax ({\tt "a sequence of bytes"}) was used to define byte
sequences. If you really wanted a linked list of characters you
instead used single quotes ({\tt 'a linked list of characters'}).

To complicate things even further strings as byte sequcnes are
represented using UTF8 which means that a character is represented by
a {\em code point} that can be: one, two, three or even four bytes. If
you only use the ASCII alphabet you get away with one byte per
character but the Swedish letters å, ä and ö require two bytes. A
tring represented as a linked list is however encoded as {\em latin1}
i.e. an extended version of ASCII. If you're not confused yet you can
try the following in a Elixir terminal:

\begin{verbatim}
> IO.inspect("åäö", binaries: :as_binaries)

> IO.inspect('åää')

\end{verbatim}

The Elixir shell of course wants to make life easy for you and will
display strings in a way that is readble. This will sometimes couse a
confusion and you will start to debug you program even though it does
return the right result. Try the following in the terminal:

\begin{verbatim}
> [104,101,108,108,111]

\end{verbatim}

If you program should return a list of five integers, Elixir will see
if all integers are between $32$ and $127$, assume that is a list of
ASCII values and print them as the characters they represent. This can
be very confusing of you don't know what is going on.


\section{A charlist - a linked list of characters}

The first thing that we should make clear is the complexity of
accessing a random element in a charlist (Elixir terminology for
linked list of characters). Since a charlist is a linked list the
complexity of different operations should not be a surpise. Use the
following definitions and explore how the execution time changes. Make
a nice table that shows your results and draw the right conclusions. 


\begin{minted}{elixir}
  def access(k, n, i) do
    a = List.duplicate(0, n)
    {t, _} = :timer.tc(fn () -> Enum.each(1..k, fn(_) -> Enum.at(a, i) end) end)
    t
  end
\end{minted}


\begin{minted}{elixir}
  def append(k, n, m) do
    a = List.duplicate(0, n)
    b = List.duplicate(1, m)
    {t, _} = :timer.tc(fn () -> Enum.each(1..k, fn(_) -> a ++ b end) end)
    t
  end
\end{minted}
  

\section{A binary - an array of bytes}

When we represent a string as a binary ytou will 








\end{document}
