\documentclass[a4paper,11pt]{article}

\input ../include/preamble.tex


\begin{document}


\title{Strings}

\author{Johan Montelius}
\date{Spring Term 20xx}

\maketitle

\defaultpagestyle

\section*{Introduction}

There are two ways to represent strings in Elixir and you should get
use to using both and knowing when to prefere one over the other. In
order to learn how to use strings you should do some performance
measurments to uderstand the complexity of different operations.

\section*{The history}

The reason Elixir has two ways to represent strings is historical and
dates back to how Erlang represented strings. The Erlang system was
not targeting text processing so strings were simply respresented as a
linked list of integers that represnted the characters of the
string. This representation allows great flexibility and did not
require any additional data structures.

As the Erlang system matured a new data structure, a byte sequence or
{\em binary}, was introduced. The binary data structure allowed for
efficient representation of large data structures and provided
efficient access operations. Once byte sequences where introduced
these were an obviuous form to represent strings; think about an array
of bytes compared to a linked list of integers.

In Erlang a string was written using quatations marks ({\tt "ABC"})
and was represented as a linked list of integers. Binaries were
written using {\em bit syntax} ({\tt <<65,66,67>>}) and libraries
operating on strings would accept either. To make things even more
flexible srings could be encoded as: linked lists och characyters,
binaries or a linked list of a mixture.

When Elixir was introduced, an Elixir uses Ruby syntax, a slight
change was made to the syntax. Since the byte sequence representation
is more efficient (and maybe what programmers expect), the quoted
syntax ({\tt "a sequence of bytes"}) was used to define byte
sequences. If you really wanted a linked list of characters you
instead used single quotes ({\tt 'a linked list of characters'}).

To complicate things even further strings as byte sequcnes are
represented using UTF8 which means that a character is represented by
a {\em code point} that can be: one, two, three or even four bytes. If
you only use the ASCII alphabet you get away with one byte per
character but the Swedish letters å, ä and ö require two bytes. A
tring represented as a linked list is however encoded as {\em latin1}
i.e. an extended version of ASCII. If you're not confused yet you can
try the following in a Elixir terminal:

\begin{verbatim}
> IO.inspect("åäö", binaries: :as_binaries)

> IO.inspect('åää')

\end{verbatim}

The Elixir shell of course wants to make life easy for you and will
display strings in a way that is readble. This will sometimes couse a
confusion and you will start to debug you program even though it does
return the right result. Try the following in the terminal:

\begin{verbatim}
> [104,101,108,108,111]

\end{verbatim}

If you program should return a list of five integers, Elixir will see
if all integers are between $32$ and $127$, assume that is a list of
ASCII values and print them as the characters they represent. This can
be very confusing of you don't know what is going on.


\section{A charlist - a linked list of characters}

The first thing that we should make clear is the complexity of
accessing a random element in a charlist (Elixir terminology for
linked list of characters). Since a charlist is a linked list the
complexity of different operations should not be a surpise. Use the
following definitions and explore how the execution time changes. Make
a nice table that shows your results and draw the right conclusions. 


\begin{minted}{elixir}
  def access(k, n, i) do
    a = List.duplicate(0, n)
    {t, _} = :timer.tc(fn () -> Enum.each(1..k, fn(_) -> Enum.at(a, i) end) end)
    t
  end
\end{minted}


\begin{minted}{elixir}
  def append(k, n, m) do
    a = List.duplicate(0, n)
    b = List.duplicate(1, m)
    {t, _} = :timer.tc(fn () -> Enum.each(1..k, fn(_) -> a ++ b end) end)
    t
  end
\end{minted}
  

\section{A binary - an array of bytes}

When we represent a string as a binary things become a bit more
complicated; it's more efficient but in order to make use of the
efficiency you need to know how things are impelemented. The following
description is not exactly how things work but it will give you the
right understanding of complexity.

A binary data structure consists of two parts: an array of bytes and a
header. The header keeps two references: one to the beginning of the
first byte of the array and one two the last byte. The array can be
shared amog several binary data structures so similar to a linked list
we do not alwyas have to copy the data structure.

Let's look at an example; the code below allocates a binary structure
using the string syntax and then uses pattern matching to get hold of
the first byte and the rest of the binary.

\begin{verbatim}
>  str = "string"

>  <<char, rest::binary>>  = str

\end{verbatim}

If we draw a image of what this will look like in memory we would en up with someting like this:

\begin{tikzpicture}
  \node[] at (0,6.5) {str : };
  \draw[] (1,6) rectangle (2,7);
  \draw[] (2,6) rectangle (3,7);
  \draw[->] (1.5,6.5) to[in=90, out=270] (2.5,5.2);
  \draw[->] (2.5,6.5) to[in=90, out=270] (7.5,5.2);  

  \draw[] (2,4) rectangle(8,5);
  \foreach \i\c in {3/s, 4/t , 5/r, 6/i, 7/n, 8/g} {
    \draw[] (\i,4) -- (\i,5);
    \node[] at (\i - 0.5, 4.5) {\c};
  };

  \node[] at (1,2.5) {rest : };
  \draw[] (2,2) rectangle (3,3);
  \draw[] (3,2) rectangle (4,3);
  \draw[->] (2.5,2.5) to[out=90, in=270] (3.5,3.8);
  \draw[->] (3.5,2.5) to[out=90, in=270] (7.5,3.8);  
\end{tikzpicture}

The two binary structures share the string "a string" so there is no
copying needed. There is some more magic going on in order to solve
garbage collection etc but to udnerstand the complexity this is fine
for now.

When we concatenate two strings, the general scheme is that the two
strings are both copied into a resulting string. If {\tt a} and {\tt
  b} are two strings then, in general, the call {\tt c = a <> b} would
mean that a new binary is allocated and the content of both {\tt a}
and {\tt b} is copied into the new array. If this is the "general"
scheme you suspect that there is a special case and there is one but
first let's ponder the complexity of the concatention procedure.

Take a look at the two definitions below, they do more or less the
same thing but the first function is operating on charlists and the
second on binaries. Working with linked lists the function will have a
$O(n)$ complexity where $n$ is the length of the first list. 

\begin{minted}{elixir}
  def app_lst([], str) do str end
  def app_lst([head | tail], str) do
    [head |  app_lst(tail, str)]
  end    
  
  def app_str(<<>>, str) do str end
  def app_str(<<head, tail::binary>>, str) do
    <<head, app_str(tail, str)::binary>>
  end  
\end{minted}

The second function of course also has a $O(n)$ component but the
concatenation in each recursion is now not a constant time operation
but an operation that depends on $m$ the length of the second binary;
the complexity is thus $O(n*m)$. 

Knowing this you can benchmark the following function; before you do
try to estimatet the complexity of the function. It does exactly the
same things as {\tt app_str/2} but now bytes are added to the end of
binary one by one to build the final concatenated result.

\begin{minted}{elixir}
  def strange(str, <<>>) do str end
  def strange(str, <<head, tail::binary>>) do
    strange(str <> <<head>>, tail)
  end
\end{minted}

After doing some benchmarking you will probably find that the
complexity os not what you expected - it's alot better, why? This is
the special case that me mentioned earlier and to understand what is
going on we have to explore the representation of binaries.

\section*{The special case}

When a binary os allocted, more room is allocated than is actually
needed. Let's say that we allocate an exponent of $2$ i.e. $8$, $16$, $32$,
$64$ .... bytes to hold a binary. Our string in teh previous example
would then look like this.

\begin{tikzpicture}
  \node[] at (0,6.5) {str : };
  \draw[] (1,6) rectangle (2,7);
  \draw[] (2,6) rectangle (3,7);
  \draw[->] (1.5,6.5) to[in=90, out=270] (2.5,5.2);
  \draw[->] (2.5,6.5) to[in=90, out=270] (7.5,5.2);  

  \draw[] (2,4) rectangle(10,5);
  \foreach \i\c in {3/s, 4/t , 5/r, 6/i, 7/n, 8/g, 9/, 10/} {
    \draw[] (\i,4) -- (\i,5);
    \node[] at (\i - 0.5, 4.5) {\c};
  };
\end{tikzpicture}

Now how can we implemnent an operation such as {\tt c = str <> "x"}?
If the runtime system knows that there is some spare room in the end
of the array we can simply copy the one byte giving us this result.


\begin{tikzpicture}
  \node[] at (0,6.5) {str : };
  \draw[] (1,6) rectangle (2,7);
  \draw[] (2,6) rectangle (3,7);
  \draw[->] (1.5,6.5) to[in=90, out=270] (2.5,5.2);
  \draw[->] (2.5,6.5) to[in=90, out=270] (7.5,5.2);  

  \draw[] (2,4) rectangle(10,5);
  \foreach \i\c in {3/s, 4/t , 5/r, 6/i, 7/n, 8/g, 9/x, 10/} {
    \draw[] (\i,4) -- (\i,5);
    \node[] at (\i - 0.5, 4.5) {\c};
  };

  \node[] at (1,2.5) {c : };
  \draw[] (2,2) rectangle (3,3);
  \draw[] (3,2) rectangle (4,3);
  \draw[->] (2.5,2.5) to[out=90, in=270] (2.5,3.8);
  \draw[->] (3.5,2.5) to[out=90, in=270] (8.5,3.8);    
\end{tikzpicture}

This of course only works once and if we continue to do {\tt d = str
  <> "y"} then the whole string {\tt str} (as well as {\tt "y"}) will
be copied. Note however that we can do {\tt d = c <> "y"} and only
copy the {\tt y} to the free position in the array.

So, what is the complexity of concatenating strings? In the general
case you will have to copy both strings so it's an $O(n+m)$
operation. If you're lucky, or are very carefull when implementing
your functions and you know what you're doing, you can get away with
$O(m)$.

\section*{Conclusions}

To understand why operations take time and correctly estimate the time
complexity of functions you need to understand how things are
implemented. The difference between using an algorithm that runs in
$O(n)$ compared to $O(n^2)$ time could be the difference of a working
implementation and something that is simply to slow to usefull. You
should of course not allways choose a $O(n)$ algorithm over an
$O(n^2)$ algorithm but you should know what you do.

Given your gained knowledge of how binary strcutres are represented,
how would you reverse a binary?

\begin{minted}{elixir}
  def nreverse(<<>>) do <<>> end
  def nreverse(<<h, tail::binary>>) do
    nreverse(tail) <> <<h>>
  end  

  def areverse(<<>>, str) do str end
  def areverse(<<h, tail::binary>>, str) do
    areverse(tail, <<h, str::binary>>)
  end    
\end{minted}






\end{document}
