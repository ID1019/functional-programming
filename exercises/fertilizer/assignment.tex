\documentclass[a4paper,11pt]{article}

\input ../include/preamble.tex


\begin{document}


\title{Give a seed a fertilizer}

\author{Johan Montelius}
\date{Spring Term 2023}

\maketitle

\defaultpagestyle

\section*{Introduction}

This assignment is based on the Advent of Code 2023 day 5 problem. The
problem is divided into two parts, the first part is mandatory while
the second part will bring you closer to a higher grade. This is the
first part of the puzzle.

\section*{Seed to location}

The first task is to given a set of seeds, figure out which seed will
give you the lowest location number. The problem is that you have some
seeds, as in the example below: 79, 14, 55 and 13. And a sequence of
strange lokking maps. The first map tells you how to convert a seed
number to a soil number, the second from a soil number to
fertilizernumber and so on. If you can apply all maps you will have
the location number for the seed and can then figure out which seed to
select (spoiler, it's nr 13).

\begin{verbatim}
seeds: 79 14 55 13

seed-to-soil map:
50 98 2
52 50 48

soil-to-fertilizer map:
0 15 37
37 52 2
39 0 15

fertilizer-to-water map:
49 53 8
0 11 42
42 0 7
57 7 4

water-to-light map:
88 18 7
18 25 70

light-to-temperature map:
45 77 23
81 45 19
68 64 13

temperature-to-humidity map:
0 69 1
1 0 69

humidity-to-location map:
60 56 37
56 93 4
\end{verbatim}

The question is of course how to interpret the map and how to do the
conversion. A map consists of a sequece of conversions, each
consisting of three number: destination ,source and range. The first
conversion of the first map above (50,98,2) means that the numbers 98
and 99 should be converted to 50 and 51 - all other seed numbers are
directly translated to the same soil number.

The first map is thus a function that takes a seed number and converts
it to a soil number and we could of course write it as follows:

\begin{minted}{elixir}
  def seed_to_soil(seed) do
    cond do
     seed < 50 -> seed
     seed < 98 -> 52 + (seed - 50)
     seed < 100 -> 50 + (seed - 98)
     true -> seed
    end
  end
\end{minted}

You will not have time to do this by hand so you have to implement a
function that given a seed number and a map determines the soil number
automatically. Since the behaviour of all translations follow the same
rules you can define one function, call it {\tt dest/2}, that takes a
source number and returns a destination number given a map. Once you
have you {\tt dest/2} function you can use the different maps
recusivly to transform a seed number to a location number.

\section*{Parsing a description}

Your real puzzle will be much larger than the small sample above so
you need to implement a function that reads a description and
transates it into something that you can work with in your
program. Start by defining a function {\tt sample/0} that returns the
text above as one string:

\begin{minted}{elixir}
  def sample() do
"seeds: 79 14 55 13

seed-to-soil map:
50 98 2
52 50 48

soil-to-fertilizer map:
0 15 37
37 52 2
39 0 15

fertilizer-to-water map:
49 53 8
 :
 :
56 93 4"
  end
\end{minted}

Now you implement a function {\tt parse/1} that takes a string that is
a description and returns a tuple {\tt \{seeds, maps\}} where seeds is
a list of seed numbers and maps is a list of maps where each map is a
list of tuples {\tt \{:map, des, src, rng\}}. 

\begin{minted}{elixir}
{:puzzle, [79, 14, 55, 13],
 [
   [{:map, 50, 98,  2}, {:map, 52, 50, 48}],
   [{:map,  0, 15, 37}, {:map, 37, 52,  2}, {:map, 39,  0, 15}],
   :
 ]}
\end{minted}

The parser is easily implementes using the functions {\tt split/2} and
{\tt trim/1} from the {\tt String} module and the {\tt parse/1}
function from that {\tt Integer} module. Try by first splitting the
original string by "\\n\\n" to get the seeds and maps separated. Then
it's small step to parse the descriptions of the seeds and recursivly
parse each of the maps.

\subsection*{Putting it all together}

So once you can parse the input description its time to put it all
together. Implementa a function that parses the sample and then uses
the description to find whiich seed nr results in the smallest
location number. If you done all steps correctly the seeds should
result in the following locations: 82, 43, 86 and 35 and the smallest
is then of course 35 (that came from seed 13). 

Once you have something that works for the sample problem it's time to
try your solution on the real puzzle. You should find a file called
"day5.csv" that is the real description of all seeds and maps. You
should be able to read this file using {\tt File.read!("day5.csv")}
(given that the file is in the same directory as your running
program). The function should return the wole description as a string
that you should be able to parse.

What is the smallest location number that you find given the real
description?


\end{document}
