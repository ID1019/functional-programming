\documentclass[a4paper,11pt]{article}

\input ../include/preamble.tex


\begin{document}


\title{Ranges of seeds}

\author{Johan Montelius}
\date{Spring Term 2023}

\maketitle

\defaultpagestyle

\section*{Introduction}

This assignment is based on the Advent of Code 2023 day 5 problem. The
first task is easier and the second requires some more coding. For the
first task you can choose you own method but for the second task you
 you should solve the problem using an implementation of ranges.


\section*{The first task}

The first part of the puzzle is quite simple. As input you're given
four seeds and a seven transformation maps. Each map tells you how
seed numbers are transformed into soil number, soil numbers to
fertilizer number etc. A transformation is describes as: {\em destination},
{\em source} and {\em range}. A sequence: {\tt 50 98 2} means that number
98 and 99 should be transformed to 50 and 51. You task is to find the
lowest {\em location number} of the seeds.

Given below is a sample problem that you can
work with before trying the true task. 

\begin{verbatim}
seeds: 79 14 55 13

seed-to-soil map:
50 98 2
52 50 48

soil-to-fertilizer map:
0 15 37
37 52 2
39 0 15

fertilizer-to-water map:
49 53 8
0 11 42
42 0 7
57 7 4

water-to-light map:
88 18 7
18 25 70

light-to-temperature map:
45 77 23
81 45 19
68 64 13

temperature-to-humidity map:
0 69 1
1 0 69

humidity-to-location map:
60 56 37
56 93 4
\end{verbatim}

If we take seed nr 79 as an example it will require {\em soil} nr 81
(since all nr 50..98 are mapped to 52..100 i.e. two numbers more. Soil
nr 81 is then mapped to fertilizer 81 (since there is no special rule
). We then end up with water nr 81 but light nr 74. The temperature
change to nr 78. This is also the humidity but the location is 82.

You first job is to parse the input string and turn it in to something
more convenient to work with. Here is some code that uses a sequence
of {\tt String.split/2} functions to turn the input string into a
tuple.

\begin{minted}{elixir}
  def parse(descr) do
    [seeds | maps] = String.split(String.trim(descr), "\n\n")
    [_ | seeds] = String.split(seeds, " ")
    seeds = Enum.map(seeds, fn(str) -> {nr, _} = Integer.parse(String.trim(str)); nr end)
    maps = Enum.map(maps, fn(map) ->
      [_| rows] = String.split(map, "\n")
      Enum.map(rows, fn(row) ->
	[d,s,r] = Enum.map(String.split(row, " "), fn(str) -> {nr, _} = Integer.parse(String.trim(str)); nr end)
	{:map, d,s,r}
      end)
    end)
    {:puzzle, seeds, maps} 
  end
\end{minted}

Once you have the initial seed numbers and maps you should implement a
function that applies the transformations.  If you get the
transformations right the location numbers for each of the seeds will
be: 82, 43, 86 and 35 so the lowest one is 35. If you pass this test
take the real task and give it a try; what is the lowest location
number?

Describe your implementation and justify your choice of data structures.


\section*{The second task}

The input data is the same as in the first task but the
interpretation has changed.  The first row in the specification does
not specify four seeds but two ranges of seeds ( 79..(79+14-1) and
55..(55+13-1)). The seeds are specified as the first seed number and
how many consecutive seed numbers we have (79 14 means 79,80,81...92).

You can use your original program and instead of exploring the
resulting location of four seeds you explore 14 + 13 = 27 seeds. This
works fine, give it a try.

You think that everything works fine and then you try not the sample
but the real problem...... ops. Your program will either crash or run
for ever, the number of seeds to try are millions and
millions, something has to be done.


\section*{Ranges}

In Elixir you have ranges of integers built in. The range of all
integers from 12 to 18 is written {\tt 12..18} and you can then use
this data structure as input to all functions in the {\tt Enum}
module. This is great but the range data structure lacks some
functionality. The ranges must be consecutive so you can not talk
about the range {\tt 1..5 - 8..12} etc.

You task is to implement a new data structure that represent a
sequence of ranges. We could for example have the sequence: 1..5,
11..13 and 18..24. This sequence then includes for example 2, 4 and 19
but not 6,8 nor 15 or 27. It is up to you how this sequence should be
represented but you should also implement some functions over sequences
so you might take a look at them before you decide.

The functions that you should implement will be very useful to when
we solve the puzzle. The functions are:

\begin{itemize}
 \item {\tt empty()} : return an empty sequence.

 \item {\tt range(from,to)} : return a sequence with one interval
   including all number from {\em from} to {\em to}.

 \item {\tt union(a, b)} : given two sequences, return the {\em union}
   i.e. a number is part of the union of {\em a} and {\em b} if, and
   only if, the number is part of {\em a} or {\em b}.

 \item {\tt intersection(a, b)} : given two sequences return the {\em
     intersection} i.e. a number is part of the {\em intersection} if, and
   only if, it is part of {\em a} and of {\em b}.

 \item {\tt difference(a, b)} : given two sequences return the {\em
     difference} i.e. a number is part of the {\em difference} if, and
   only if, it is part of {\em a} but not of {\em b}.

 \item {\tt add(a, n)} : given a sequence, {\em a}, and a number {\em n},
   return a sequence where each element in the original sequence has been
   incremented by {\em n}.
\end{itemize}

Depending on how you choose to represent a range, it becomes more or
less easy to implement these functions. We could of course choose to
represent a range {\tt 3..5 - 8..10} as list of
integers,{\tt \[3,4,5,8,9,10\]}, but that might be an impractical form
if we are dealing with ranges such as {\tt 1..12345678}.

Choosing the right representation should be done with the operations in
mind; some things might be easier to do depending on the
representation.... and, have in mind that the sequences can span over
very large sets. Describe your implementation and justify the
representation that you have chosen.

\subsection*{The second task revisited}

What will happen if you rewrite your program to use sequences of
ranges instead of unique numbers? Using the sample you would start
with the ranges 55..67 and 79..92. These ranges should now be
transformed using the different maps. A map of course consists of a set of
ranges and how those ranges should be transformed.

Take the first map as an example. We see that the range 50..98 should
be transformed by adding 2. Since both our seed ranges are fully
covered by this range we should end end up with the ranges 57..69 and
81..94. The second mapping does not change the ranges but when going from
fertilizer to water we end up with the ranges 53..56, 61..69 and 81..94.

What is the general rule of the transformation? Assume that you have a
range {\em source} that should be incremented by $n$ and a sequence
{\em seeds}, then ... yes, what should be done? If you can describe
what should be done in terms of: {\em union}, {\em intersection}, {\em
  difference} and {\em add} then you have your implementation.

Start by rewriting the parser, or work with the parsed representation,
and produce a new representation of the puzzle. The seeds are now a
sequence of ranges and each rule is a set of ranges with rules of
transformation.

Using the real input for seeds and maps, What is now the minimum
location nr that we end up with? Describe your implementation and the
results you obtain.


\end{document}
