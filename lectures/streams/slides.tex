
\usepackage{fancyhdr}
\usepackage{lastpage}
\usepackage[utf8]{inputenc}

% Minted for syntax highliting
\usepackage{minted}
\usemintedstyle{tango}

% 
\usepackage[T1]{fontenc}
\usepackage{lmodern}

\usepackage{calc}
\usepackage{bytefield}

\usepackage{listings}
\usepackage{amsmath}

\usepackage{tikz}
\usetikzlibrary{automata,arrows,topaths,calc,positioning}
 
\usepackage{syntax}
\grammarindent=2cm


% Headers/footers styling
\pagestyle{fancy}
\fancyhf{}
\renewcommand{\headrulewidth}{0pt}

% Footer
\lfoot{ID1019}
\cfoot{KTH}
\rfoot{\thepage \hspace{1pt} / \pageref{LastPage}}

%\newcommand{\defaultpagestyle}{\thispagestyle{plain}}
\newcommand{\defaultpagestyle}{\thispagestyle{fancy}}



\title[ID1019 Streams]{Streams}

\author{Johan Montelius}
\institute{KTH}
\date{\semester}

\begin{document}

\begin{frame}
\titlepage
\end{frame}

\begin{frame}[fragile]{an infinite list}

\pause\vspace{20pt}

\verb+inf = infinity()+\pause \verb+; [0|inf] = inf.()+\pause \verb+; [1|inf] = inf.()+

\pause
\begin{verbatim}
def infinity() do 
  fn() -> infinity(0) end 
end
\end{verbatim}
\pause
\begin{verbatim}
def infinity(n) do
  [...|...] 
end
\end{verbatim}

\end{frame}


\begin{frame}[fragile]{the list of fibonacci }

A function that returns an infinite list of Fibonacci numbers.

\pause\vspace{20pt}

\begin{verbatim}
def fib() do 
  fn() -> fib(1,1) end 
end
\end{verbatim}
\pause
\begin{verbatim}
def fib(f1, f2) do
  [f1 | fn() -> fib(f2, f1+f2) end]
end
\end{verbatim}

\end{frame}

\begin{frame}{a lazy list from 1 to 10}

  Let's represent a {\em range} of integers from 1 to 10 as:

  \vspace{10pt}\hspace{20pt} {\tt \{:range, 1, 10\}}

  \vspace{10pt}\pause

  Elixir gives us a syntax for this:

    \vspace{10pt}\hspace{20pt} {\tt 1..10}

    \vspace{10pt}\pause
    But we will do our own :-)

    
  \vspace{40pt}\pause
  {\em This is not how Elixir represents it but it's fine for now}

\end{frame}

\begin{frame}[fragile]{sum/1 of a range}

  \pause
\begin{verbatim}
def sum({:range, to, to}) do ... end  
def sum({:range, from, to}) do  ... + sum({:range, ..., to}) end  
\end{verbatim}

  \vspace{20pt}  \pause
  
\begin{verbatim}
def sum(range) do  sum(range, 0) end

def sum({:range, to, to}, acc) do ... end  
def sum({:range, from, to}, acc) do sum({:range, ..., to}, ...) end  
\end{verbatim}
  
  
\end{frame}

\begin{frame}[fragile]{foldl/3 on a range}

  How do we fold-left on a range:  \pause

  \vspace{20pt}

  \verb= foldl({:range, 1, 5}, 0, fn(x,a) -> x + a end)=

\end{frame}


\begin{frame}[fragile]{sum/1 of a range}


\begin{verbatim}

def sum(range) do foldl(range, 0, fn(x,acc) -> x + acc end) end

\end{verbatim}
  
\end{frame}


\begin{frame}[fragile]{map/2 on a range}

  How do we map on a range (let's forget the order):  \pause

  \vspace{20pt}

  \verb= map({:range, 1, 5}, fn(x) -> x + 1 end)= \pause

  \vspace{20pt}
  {\em should we return a list of values or .... a modified range?}

\end{frame}


\begin{frame}[fragile]{filter/2 on a range}

  How do we filter a range (again, f*ck the order):  \pause

  \vspace{20pt}

  \verb+ filter({:range, 1, 5}, fn(x) -> rem(x,2) == 0 end)+ \pause

  \vspace{20pt}
  {\em should we return a list of values or ....}

\end{frame}

\begin{frame}[fragile]{take/2 on a range}

  How do we take n elements from a range (order ... not):  \pause

  \vspace{20pt}

  \verb+ take({:range, 1, 1_000_000}, 5)+ \pause

  \vspace{20pt}

  {\em we don't want to build a list of a million integers}


\end{frame}


\begin{frame}[fragile]{reduce/3: the goto of all}

\begin{verbatim}
  def sum(r) do
    reduce(r,  0, fn(x,a) -> x+a end)
  end
\end{verbatim}

  \vspace{10pt}\pause{\em Our {\tt reduce/3} should work as {\tt foldl/3} (does it make sense to have a {\tt foldr/3}).}
  
\begin{verbatim}
  def reduce({:range, from , to}, acc, fun) do
    if from <= to do
      reduce({:range, from+1, to}, fun.(from, acc), fun)
    else
      acc
    end
  end
\end{verbatim}

  \vspace{20pt}\pause

  {\em ... we're not done!}
  
\vspace{20pt}  \pause
\end{frame}

\begin{frame}{How do we stop in the midle?}

  Implement take/2 using reduce by ..... 

  \vspace{20pt}\pause
  {\em We need to control the reduction.}
  
\end{frame}


\begin{frame}[fragile]{continue}

\begin{verbatim}
  def reduce({:range, from , to}, {:cont, acc}, fun) do
    if from <= to do
      reduce({:range, from+1, to}, fun.(from, acc), fun)
    else
      {:done, acc}
    end
  end
\end{verbatim}

  \vspace{20pt}\pause
  
\begin{verbatim}
  def sum(r) do
    reduce(r, {:cont, 0}, fn(x,a) -> {:cont, x+a} end)
  end
\end{verbatim}

  \vspace{10pt}\pause
  {\em The accumulator is both a value and an instruction to continue.}

  
\end{frame}

\begin{frame}[fragile]{stop in the midle}

\begin{verbatim}
    :
  def reduce(_, {:halt, acc}, _fun) do
    {:halted, acc}
  end
\end{verbatim}

\vspace{20pt}\pause
\begin{verbatim}
  def take(r, n) do
    reduce(r, {:cont, {:sofar, n, [] }},
      fn(x,{:sofar, n, a}) ->
        if n > 0 do
          {:cont, {:sofar, n-1, [x|a]}}
        else
          {:halt, [x|a]}
        end
      end)
  end    
\end{verbatim}
  
\end{frame}



\begin{frame}[fragile]{suspend in the midle:  head and tail }

  
\begin{verbatim}
    :
  def reduce(range, {:suspend, acc}, fun) do
    {:suspended, acc, fn(cmd) -> reduce(range, cmd, fun) end}
  end
\end{verbatim}
  \pause


\begin{verbatim}
  def head(r) do
    reduce(r, {:cont, :na},
      fn (x, _) ->
        {:suspend, x}
      end)
  end
\end{verbatim}
  
\end{frame}

\begin{frame}{Elixir libraries}

  \begin{itemize}
  \item {\tt List} : operatates on lists, returns a list or some value.  \pause
  \item {\tt Enum} : takes an {\tt Enumerable} as argument, returns a list or value. \pause
  \item {\tt Stream} : takes an {\tt Enumerable} as argument, returns an {\tt Enumerable} or value. 
  \end{itemize}

  \vspace{20pt}\pause {\em A datastructure is {\em Enumerable} if it
    implements the {\em enumerable protocol}. Lists and ranges are {\tt
      Enumerable}.}
  
\end{frame}

\begin{frame}{Summary}

\begin{itemize}
\pause\item {range}: representation of a range of integers
\pause\item {streams}: lazy evaluation of sequences
\end{itemize}

\end{frame}


\end{document}



