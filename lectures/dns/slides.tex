
\usepackage{fancyhdr}
\usepackage{lastpage}
\usepackage[utf8]{inputenc}

% Minted for syntax highliting
\usepackage{minted}
\usemintedstyle{tango}

% 
\usepackage[T1]{fontenc}
\usepackage{lmodern}

\usepackage{calc}
\usepackage{bytefield}

\usepackage{listings}
\usepackage{amsmath}

\usepackage{tikz}
\usetikzlibrary{automata,arrows,topaths,calc,positioning}
 
\usepackage{syntax}
\grammarindent=2cm


% Headers/footers styling
\pagestyle{fancy}
\fancyhf{}
\renewcommand{\headrulewidth}{0pt}

% Footer
\lfoot{ID1019}
\cfoot{KTH}
\rfoot{\thepage \hspace{1pt} / \pageref{LastPage}}

%\newcommand{\defaultpagestyle}{\thispagestyle{plain}}
\newcommand{\defaultpagestyle}{\thispagestyle{fancy}}


 
 
\title[ID1019 DNS]{A DNS Resolver}
 

\author{Johan Montelius}
\institute{KTH}
\date{\semester}

\begin{document}

\begin{frame}
\titlepage
\end{frame}

\begin{frame}{Domain Name System}

\end{frame}

\begin{frame}[fragile]{RFC 1035}

\begin{verbatim}
             Local Host                        |  Foreign
                                               |
+---------+               +----------+         |  +--------+
|         | user queries  |          |queries  |  |        |
|  User   |-------------->|          |---------|->|Foreign |
| Program |               | Resolver |         |  |  Name  |
|         |<--------------|          |<--------|--| Server |
|         | user responses|          |responses|  |        |
+---------+               +----------+         |  +--------+
                            |     A            |
            cache additions |     | references |
                            V     |            |
                          +----------+         |
                          |  cache   |         |
                          +----------+         |
\end{verbatim}

\end{frame}


\begin{frame}{the resolver}

\begin{itemize}
\item client: sends request to resolver
\item resolver: receives requests, queries servers/resolvers and caches responses 
\item server: responsible for sub-domain
\end{itemize}

\vspace{10pt}\pause  

{\em The first resolver is most probably running on your laptop.}
\end{frame}


\begin{frame}[fragile]{let's build a DNS resolver}
 
\pause

\begin{verbatim}
defmodule  DNS do

  @server {8,8,8,8}
  @port 53
  @local 5300
\end{verbatim}

\begin{verbatim}
  def start() do
    start(@local, @server, @port)
  end

  def start(local, server, port) do
    spawn(fn() -> init(local, server, port) end)
  end
\end{verbatim}

\vspace{10pt}\pause
{\em The server is the DNS server to which queries are routed.}
\end{frame}

\begin{frame}[fragile]{two datagram sockets}

\begin{verbatim}
  def init(local, server, port) do
    case :gen_udp.open(local, [{:active, true}, :binary]) do
      {:ok, local} ->
        case :gen_udp.open(0, [{:active, true}, :binary]) do
          {:ok, remote} ->
            dns(local, remote, server, port)
          error ->
            :io.format("DNS error opening remote socket: ~w~n", [error])
        end
      error ->
        :io.format("DNS error opening local socket: ~w~n", [error])
    end
  end
\end{verbatim}

\end{frame}

\begin{frame}[fragile]{the server loop}

\begin{verbatim}
  def dns(local, remote, server, port) do
    receive do
      {:udp, ^local, _client, _client_port, _msg} ->
        dns(local, remote, server, port)
      :stop ->
        :ok
      :update ->
        DNS.dns(local, remote, server, port)
      strange ->
        :io.format("strange message ~w~n", [strange])
        dns(local, remote, server, port)
    end
  end
\end{verbatim}

\vspace{20pt}\pause{\em Let's try.}

\end{frame}

\begin{frame}[fragile]{let's decode the message}


  \begin{bytefield}{32}
    \bitheader{0,7-8,15-16,23-24,31} \\

    \begin{rightwordgroup}{transport \\ header}
      \bitbox{16}{identifier} &  \bitbox{16}{flags}\\
      \bitbox{16}{\# query blocks} &  \bitbox{16}{\# answer blocks }\\
      \bitbox{16}{\# authority blocks} &  \bitbox{16}{\# additional blocks }
    \end{rightwordgroup}\\
    \begin{rightwordgroup}{data \\ fields}
      \wordbox[lrt]{1}{%
        \parbox{0.6\width}{\centering \vspace{10pt} query, answer, authority \\ and additional blocks}} \\
      \skippedwords \\
      \wordbox[lrb]{1}{} 
    \end{rightwordgroup}

  \end{bytefield}

\vspace{10pt}\pause
{\em Query and response messages of the same format.}

\end{frame}

 
\begin{frame}[fragile]{message flags}

\begin{itemize}


\item QR: query or reply
\item Op-code: the operation 
\item AA: authoritative answer (if the server is responsible for the domain)
\item TC: message truncated, more to follow
\item RD: recursion desired by client 
\item RA: recursion available by server
\item Resp-code: ok or error message in response
\end{itemize}

\vspace{10pt}\pause

  \begin{bytefield}[bitwidth=2em]{16}
    \bitheader{0,1,4-5,6,7,8-9,11-12,15} \\
    \bitbox{1}{QR} & \bitbox{4}{Op-code} & \bitbox{1}{AA} & \bitbox{1}{TC} & \bitbox{1}{RD} & \bitbox{1}{RA} &  \bitbox{3}{-} &   \bitbox{4}{Resp-code} 
  \end{bytefield}


\vspace{10pt}\pause
{\em This is getting complicated.}

\end{frame}


\begin{frame}[fragile]{the bit syntax}

\begin{verbatim}
def decode(<<id::16, flags::binary-size(2), 
         qdc::16, anc::16, 
         ncs::16, arc::16, 
         body::binary>>=raw) do
\end{verbatim}

\pause
\begin{verbatim}
    <<qr::1, op::4, aa::1, tc::1, rd::1, ra::1, _::3, resp::4>> = flags
\end{verbatim}
\pause
\begin{verbatim}
    decoded = decode_body(qdc, anc, ncs, arc, body, raw)
\end{verbatim}
\pause
\begin{verbatim}
    {id, qr, op, aa, tc, rd, ra, rcode, decoded}
end
\end{verbatim}


{\em Why passing the raw message to the decoding of the body?}


\end{frame}

\begin{frame}[fragile]{decode the body}

The body consists of a number of: query, response, authoritative (server node) and additional sections.

\vspace{10pt}\pause

The answer, authoritative and additional sections follow the same pattern,
the query is slightly different.

\begin{verbatim}
decode_body(qdc, anc, nsc, arc, body, raw) do
    {query, rest} = decode_query(qdc, body, raw)
    {answer, rest} = decode_answer(anc, rest, raw)
    {authority, rest} = decode_answer(nsc, rest, raw)
    {additional, _} = decode_answer(arc, rest, raw)
    {query, answer, authority, additional}
end
\end{verbatim}

\vspace{10pt}\pause

{\em Note the nestling of the reminder of the body.}

\end{frame}

\begin{frame}{decode a query}

A query consists of a sequence of queries (we know from the header how many).

\begin{grammar}
<query> ::= <name> <query type> <query class>

<name> ::= <empty> | <label> <name>

<empty> ::=  {\em 8 bits} 0 

<label> ::=  <length> <byte sequence of length> \\

<query type> ::= {\em 16 bits}  (1 = A, ... 15 = MX, 16 = TXT, ...)

<query class> ::= {\em 16 bits} (1 = Internet)

<length> ::= {\em 8 bits}  (0..63 i.e. the two highest bits are set to zero)

\end{grammar}


\end{frame}

\begin{frame}[fragile]{decode a query}

\begin{verbatim}
def decode_query(0, body, _) do
  {[], body}
end
def decode_query(n, body, raw) do
  {name, <<qtype::16, qclass::16, rest::binary>>} = decode_name(body, raw),
  {decoded, rest} = decode_query(n-1, rest, raw),
  {[{name, qtype, qclass} | decoded], rest}
end
\end{verbatim}

\end{frame}


\begin{frame}[fragile]{decode a name}

\begin{verbatim}
def decode_name(label, raw) do
  decode_name(label, [], raw)
end
\end{verbatim}
\vspace{20pt}\pause

\begin{verbatim}
def decode_name(<<0::1, 0::1, _::6, _::binary>>=label, names, raw) do
  decode_label(label, names, raw)
end
\end{verbatim}
\end{frame}


\begin{frame}[fragile]{decode a label}

\begin{verbatim}
def decode_label(<<0::8, rest::binary>>, names, _) do
  {Enum.reverse(names), rest}
end
def decode_label(<<n::8, rest::binary>>, names, raw) do
  decode_label(n, rest, [], names, raw)
end
\end{verbatim}

\vspace{20pt}\pause

\begin{verbatim}
decode_label(0, rest, sofar, names, raw) ->
  decode_name(rest, [Enum.reverse(sofar)|names], raw)
end
decode_label(n, <<char::8, rest::binary>>, sofar, names, raw) ->
  decode_label(n-1, rest, [char|sofar], names, raw)
end
\end{verbatim}

\end{frame}

\begin{frame}[fragile]{query example}

  Erlang binary:
\begin{verbatim}
  <<4,12, 1, 0,
    0, 1, 0, 0,
    0, 0, 0, 0,
    3,119,119,119,3,107,116,104,2,115,101,0,
    0,1,0,1>>
\end{verbatim}

\vspace{20pt} \pause
  Decoded query:
\begin{verbatim}
  {1036,0,0,0,0,1,0,0,{[{["www","kth","se"],1,1}],[],[],[]}}
\end{verbatim}  

\end{frame}

\begin{frame}{encoding names by offset}

The names in answers may use a more compact form of encoding.  

\vspace{20pt}\pause

Assume we have encoded {\tt www.kth.se} and need to encode {\tt
  mail.kth.se} - then we can reuse the coding of {\tt kth.se}.

\vspace{20pt} \pause

\begin{grammar}
<label> ::=  <length> <byte sequence of length n> | \\
             <offset>

<offset> ::= {\em 16 bits} (two highest bits set to ones)

\end{grammar}


\vspace{20pt} \pause


{\em The length version will always have the top two bits set to {\tt 00} and the offset version will have them set to {\tt 11}.}

\end{frame}

\begin{frame}[fragile]{offset encoding}

\begin{verbatim}
def decode_name(<<0::1, 0::1, _::6, _::binary>>=label, names, raw) do
    ## regular name encoding
    decode_label(label, names, raw)
end
      
def decode_name(<<1::1, 1::1, offset::14, rest::binary>>, names, raw) do
    ## offset encoding
    bits = 8*offset
    <<_::bits, section::binary>> = raw
    {name, _} = decode_label(section, names, raw)
    {name, rest}
end
\end{verbatim}

\end{frame}

\begin{frame}[fragile]{decode an answer}

All answer sections have the same basic structure:

\begin{grammar}
<answer> ::= <name> <type> <class> <ttl> <length> <resource record>
\end{grammar}

\begin{itemize}
\item type 16-bits: A-type, NS-, CNAME-, MX- etc
\item class 16-bits: Internet, ...
\item TTL 32-bits: time in seconds (typical some hours)
\item length 16-bits: the length of the record in bytes
\end{itemize}

\vspace{10pt}\pause
{\em The resource record is coded depending on the type of resource.}

\end{frame}


\begin{frame}{let's try}


\end{frame}

\begin{frame}{forward the reply}

\end{frame}




\end{document}
