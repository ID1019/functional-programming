
\usepackage{fancyhdr}
\usepackage{lastpage}
\usepackage[utf8]{inputenc}

% Minted for syntax highliting
\usepackage{minted}
\usemintedstyle{tango}

% 
\usepackage[T1]{fontenc}
\usepackage{lmodern}

\usepackage{calc}
\usepackage{bytefield}

\usepackage{listings}
\usepackage{amsmath}

\usepackage{tikz}
\usetikzlibrary{automata,arrows,topaths,calc,positioning}
 
\usepackage{syntax}
\grammarindent=2cm


% Headers/footers styling
\pagestyle{fancy}
\fancyhf{}
\renewcommand{\headrulewidth}{0pt}

% Footer
\lfoot{ID1019}
\cfoot{KTH}
\rfoot{\thepage \hspace{1pt} / \pageref{LastPage}}

%\newcommand{\defaultpagestyle}{\thispagestyle{plain}}
\newcommand{\defaultpagestyle}{\thispagestyle{fancy}}



\title[ID1019 Overview]{Overview}


\author{Johan Montelius}
\institute{KTH}
\date{\semester}

\begin{document}

\begin{frame}
\titlepage
\end{frame}


\begin{frame}{Introduction}
  \begin{itemize}
    \pause \item Learning outcomes
    \pause \item Literature
    \pause \item Lectures
    \pause \item Seminars
    \pause \item Exam
    \pause \item More
  \end{itemize}
\end{frame}


\begin{frame}{Learning outcomes}
  The aim of the course:

\pause
\begin{itemize}
\item Functional programming:
\pause
  \begin{itemize}
   \item recursion, pattern matching, functions as first class objects, closures, higher order functions and immutable data structures
\pause
   \item implement selected algorithms in a functional programming language
  \end{itemize}
\pause
\item Concurrent programming:
  \begin{itemize}
\pause
    \item advantages and disadvantages
\pause
    \item message passing, actors model
\pause
    \item design, implement, test and debug 
  \end{itemize}  
\end{itemize}

\end{frame}


\begin{frame}{why functional programming}

Why do we need a course in functional and concurrent programming?

\begin{tikzpicture}[scale=1.0]
\pause
 \node (haskell) at (1,6) {\includegraphics[scale=0.4]{haskell.png}};
\pause
 \node (lisp) at (6,7) {\includegraphics[scale=0.3]{lisp.png}};
\pause
 \node (erlang) at (0,2) {\includegraphics[scale=0.6]{erlang.jpeg}};
 \pause
 \node (f) at (10,6) {\includegraphics[scale=0.4]{f.jpeg}};
 \pause
 \node (elixir) at (8,4) {\includegraphics[scale=0.4]{elixir.png}}; 
\pause 
 \node (scala) at (4,4) {\includegraphics[scale=0.1]{scala.png}};
\pause
 \node (python) at (4,2) {\includegraphics[scale=0.4]{python.png}};
\pause
\node (c11) at (10,2) {\includegraphics[scale=0.3]{c11.jpeg}};
\pause
 \node (rust) at (3,6) {\includegraphics[scale=0.3]{rust.png}};
\end{tikzpicture}

\end{frame}

\begin{frame}{why concurrency}

Two reasons:

\pause\vspace{20pt}\hspace{60pt}A tool to model interactive services.

\pause\vspace{20pt}\hspace{60pt}Hardware can utilize concurrency to speed-up computations.

\pause\vspace{40pt}
{\em To build a good game engine, you need to master concurrency}

\end{frame}


\begin{frame}{Literature}

  This course is not about a particular language but .... \pause

  \vspace{20pt}\hspace{40pt} ... we need a language to experiment with.

\pause
\begin{columns}
 \begin{column}{0.6\linewidth}
   \begin{itemize}
   \item Introducing Elixir - Getting Started in Functional Programming
   \item Simon St. Laurent, J. Eisenberg
   \item O'Reilly Media
   \end{itemize}
 \end{column}
 \begin{column}{0.4\linewidth}
    \includegraphics[scale=0.2]{lrg.jpg}    
 \end{column}
\end{columns}

  \vspace{20pt}\hspace{40pt} {\em Best way to learn the principles of biking is riding a bike.}

\end{frame}


\begin{frame}{Lectures}

   We will have 14 lectures that will cover the following aspects:


  \begin{itemize}
    \item Functional programming
\pause
    \item Programming techniques
\pause
    \item Complexity 
\pause
    \item Concurrency 
\pause
    \item Parallelism
\pause
    \item Servers
  \end{itemize}
\end{frame}


\begin{frame}{Seminars}

  To take part in the seminar sessions, you should:
  \begin{itemize}
  \item solve an assigment before the seminar
  \end{itemize}

  \pause \vspace{20pt}
  The seminars are not compulsory but,\pause if you attend you should have prepared well. 

  \pause \vspace{20pt}
  Collaborate in the implementation.

\end{frame}

\begin{frame}{Seminars}
  \begin{itemize}
    \item Huffman coding: implementing a compression algorithm \pause
    \item A meta-interpreter: Elixir interpreter in Elixir \pause
    \item Dining philosophers: introduction to concurrency \pause
    \item A Mandelbrot image: parallelism \pause
    \item A small web server: easier than you think
  \end{itemize}
\end{frame}

\begin{frame}{Exam}

\pause A closed book exam -- {\bf where writing code by hand will be a large part.}

\vspace{20pt}

\pause Final grade is based on written exam. 

\end{frame}

\begin{frame}{and finally}

  Two student representatives for course board. We will meet a couple
  of times during the course so that you can give feedback.

\pause\vspace{20pt}

Don't use KTH Social for feedback or questions - use Canvas. 

\end{frame}

\end{document}
