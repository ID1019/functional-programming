
\usepackage{fancyhdr}
\usepackage{lastpage}
\usepackage[utf8]{inputenc}

% Minted for syntax highliting
\usepackage{minted}
\usemintedstyle{tango}

% 
\usepackage[T1]{fontenc}
\usepackage{lmodern}

\usepackage{calc}
\usepackage{bytefield}

\usepackage{listings}
\usepackage{amsmath}

\usepackage{tikz}
\usetikzlibrary{automata,arrows,topaths,calc,positioning}
 
\usepackage{syntax}
\grammarindent=2cm


% Headers/footers styling
\pagestyle{fancy}
\fancyhf{}
\renewcommand{\headrulewidth}{0pt}

% Footer
\lfoot{ID1019}
\cfoot{KTH}
\rfoot{\thepage \hspace{1pt} / \pageref{LastPage}}

%\newcommand{\defaultpagestyle}{\thispagestyle{plain}}
\newcommand{\defaultpagestyle}{\thispagestyle{fancy}}



\title[ID1019 Parsing]{Parsing}

\author{Johan Montelius}
\institute{KTH}
\date{\semester}

\begin{document}

\begin{frame}
\titlepage
\end{frame}

\begin{frame}[fragile]{Why parsing?}

$$2x^3 + 4x + 5$$
\pause 

\hspace{20pt} From this:  {\tt "2x\^{}3 + 4x + 5"}
\pause
\vspace{20pt}
\hspace{20pt} To this:

\begin{verbatim}
    {:add, {:mul, {:num, 2}, 
                  {:exp, {:var, :x}, {:num, 3}}},
           {:add, {:mul, {:num, 4}, {:var, :x}},
                  {:num, 5}
           }
    }
\end{verbatim}

\end{frame}

\begin{frame}[fragile]{lexical scanner or tokenizer}

  \pause
  First step - identify which {\em tokens} we use in our language.

  \pause\vspace{20pt}
  \begin{itemize}
  \item numbers : 1, 12, 123 ....
  \item variables :  x, y, z, ....
  \item operators : +, *, \^{}, ...
  \end{itemize}

  \pause\vspace{20pt}
  
  From: {\tt "2x\^{}3 + 4x + 5"}  \pause

  \pause\vspace{20pt}
  To: {\tt [\{:num, 12\}, \{:var, :x\}, :exp, \{:num, 3\}, :plus, \{:num, 4\}, \{:var, :x\}, :plus, \{:num, 5\}]}
  
\end{frame}

\begin{frame}[fragile]{parser}
  \pause
   Second step - identify the grammatic rules given the tokens.

  \pause 
\begin{verbatim}
  expr ::=  number |
            variable |
            expr plus expr |
            expr times expr |
            expr exp expr
\end{verbatim}

\pause\vspace{10pt}
  From: {\tt [\{:num, 12\}, \{:var, :x\}, :exp, \{:num, 3\}, :plus, \{:num, 4\}, \{:var, :x\}, :plus, \{:num, 5\}]}
\pause\vspace{20pt}  

To:
\begin{verbatim}
 {:add, {:mul, {:num, 2}, {:exp, {:var, :x}, {:num, 3}}}, 
        {:add, {:mul, {:num, 4}, {:var, :x}}, 
               {:num, 5}}}
\end{verbatim}
\end{frame}


\begin{frame}{tools}

  You don't want to write your own tokenizer/parse - use a tool that generate the for you. \pause
  
  \begin{itemize}
  \item C/C++ : lex/yacc or flex/bison
  \item Java : JavaCC
  \item Java/Python/.. : ANTLR
  \item Erlang/Elixr : leex/yecc
  \end{itemize}

\end{frame}

\begin{frame}[fragile]{tokenizer.xrl}

\begin{verbatim}
Definitions.
NUM        = [0-9]+
VAR        = [a-z]
EXP        = \^
WHITESPACE = [\s\t\n\r]

Rules.
{NUM}         : {token, {num, TokenLine, list_to_integer(TokenChars)}}.
{VAR}         : {token, {var, TokenLine, list_to_atom(TokenChars)}}.
{EXP}         : {token, {exp, TokenLine}}.
\+            : {token, {plus, TokenLine}}.
\*            : {token, {times, TokenLine}}.
{WHITESPACE}+ : skip_token.

Erlang code.
\end{verbatim}
\end{frame}

\begin{frame}[fragile]{parser.yrl}
\begin{verbatim}
Nonterminals expr.
Trminals num var plus times exp.
Rootsymbol expr.

expr -> num : value('$1').
expr -> var : name('$1').
expr -> expr plus expr  : {add, '$1', '$3'}.
expr -> expr times expr : {mul, '$1', '$3'}.
expr -> expr exp expr   : {exp, '$1', '$3'}.

Erlang code.
value({num, _, Value}) -> Value.
name({var, _, Name}) -> Name.
\end{verbatim}
\end{frame}

\end{document}


