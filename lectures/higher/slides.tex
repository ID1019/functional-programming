
\usepackage{fancyhdr}
\usepackage{lastpage}
\usepackage[utf8]{inputenc}

% Minted for syntax highliting
\usepackage{minted}
\usemintedstyle{tango}

% 
\usepackage[T1]{fontenc}
\usepackage{lmodern}

\usepackage{calc}
\usepackage{bytefield}

\usepackage{listings}
\usepackage{amsmath}

\usepackage{tikz}
\usetikzlibrary{automata,arrows,topaths,calc,positioning}
 
\usepackage{syntax}
\grammarindent=2cm


% Headers/footers styling
\pagestyle{fancy}
\fancyhf{}
\renewcommand{\headrulewidth}{0pt}

% Footer
\lfoot{ID1019}
\cfoot{KTH}
\rfoot{\thepage \hspace{1pt} / \pageref{LastPage}}

%\newcommand{\defaultpagestyle}{\thispagestyle{plain}}
\newcommand{\defaultpagestyle}{\thispagestyle{fancy}}



\title[ID1019 Higher order]{Higher order}


\author{Johan Montelius}
\institute{KTH}
\date{\semester}

\begin{document}

\begin{frame}
\titlepage
\end{frame}

\begin{frame}[fragile]{let's play some cards}

\begin{columns}

 \begin{column}{0.2\linewidth}
   \includegraphics[width=\linewidth]{kung.png}
 \end{column}
 
 \pause

 \begin{column}{0.8\linewidth}
  Properties of a card:
  \begin{itemize}
   \item ${\rm Suit} \in \{{\rm spade}, {\rm heart}, {\rm diamond}, {\rm club}\}$
   \pause
   \item ${\rm Value} \in \{2,3,\ldots 14\}$
   \pause 
   \item ${\rm Card} \in  \{\langle {\rm s}, {\rm v} \rangle | \quad {\rm s} \in {\rm Suit} \wedge {\rm v} \in {\rm Value} \}$
  \end{itemize}

  \vspace{10pt}  \pause
  In Elixir:
  \begin{verbatim}
  @type suite :: :spade | :heart | :diamond | :club
  @type value :: 2..14
  @type card :: {:card, suite, value}
  \end{verbatim}

 \end{column}
\end{columns}

\end{frame}

\begin{frame}[fragile]{order of cards}

\pause
\begin{verbatim}
def lt({:card, s, v1}, {:card, s, v2})  do  v1 < v2 end
\end{verbatim}
\pause
\begin{verbatim}
def lt({:card, :club, _}, _) do  true end
\end{verbatim}
\pause
\begin{verbatim}
def lt({:card, :diamond, _}, {:card, :heart, _}) do true end
\end{verbatim}
\pause
\begin{verbatim}
def lt({:card, :diamond, _}, {:card, :spade, _}) do true end
\end{verbatim}
\pause
\begin{verbatim}
def lt({:card, :heart, _}, {:card, :spade, _}) do true end
\end{verbatim}
\pause
\begin{verbatim}
def lt({:card, _, _}, {:card, _, _}) do false end
\end{verbatim}

\end{frame}

\begin{frame}[fragile]{sorting cards}

\begin{columns}

 \begin{column}{0.4\linewidth}
\begin{verbatim}
@spec sort([card]) :: [card]
\end{verbatim}
\begin{verbatim}
def sort([]) do [] end
def sort([c]) do [c] end
\end{verbatim}
\pause
\begin{verbatim}
def sort(deck) do
  {d1, d2} = split(deck)
  s1 = sort(d1)
  s2 = sort(d2)
  merge(s1, s2)
end
\end{verbatim}
 \end{column}
 
 \pause

 \begin{column}{0.5\linewidth}
\begin{verbatim}
@spec split([card]) :: {[card],[card]}
\end{verbatim}
\pause
\begin{verbatim}
def split([]) do {[], []} end
\end{verbatim}
\begin{verbatim}
def split([c|rest]) do 
  {s1, s2} = split(rest)
  { ..., ...}
end
\end{verbatim}
 \end{column}
\end{columns}

\end{frame}

\begin{frame}[fragile]{tail recursive split}

\begin{columns}

 \begin{column}{0.4\linewidth}
\begin{verbatim}
def split([]) do 
  {[], []} 
end
\end{verbatim}
\pause
\begin{verbatim}
def split([c|rest]) do 
  {s1, s2} = split(rest)
  { ..., ...}
end
\end{verbatim}
 \end{column}
 
 \pause

 \begin{column}{0.6\linewidth}
\begin{verbatim}
def split(deck) do split(deck, [], []) end 
\end{verbatim}
\pause
\begin{verbatim}
@spec split([card], [card], [card]) :: 
   {[card],[card]}
\end{verbatim}   
\pause   

\begin{verbatim}
def split([], d1, d2) do  {d1, d2} end
\end{verbatim}
\pause
\begin{verbatim}
def split([c|rest], d1, d2) do
   split(rest, ..., ...)
end
\end{verbatim}
 \end{column}
\end{columns}

\end{frame}

\begin{frame}[fragile]{sorting cards}

\begin{verbatim}
@spec merge([card], [card]) :: [card]
\end{verbatim}
\pause
\begin{verbatim}
def merge([], s2) do ... end
def merge(s1, []) do ... end
\end{verbatim}
\pause
\begin{verbatim}
def merge([c1|r1]=s1, [c2|r2]=s2) do
  case lt(c1, c2) do
    true ->
      [c1| merge(r1, s2)]
    false ->
      [c2| merge(s1, r2)]     
  end
end
\end{verbatim}

\end{frame}

\begin{frame}{what to do}

\vspace{10pt}\pause Implement function that sorts names of people.

\vspace{10pt}\pause Implement function that sorts a frequency table.

\vspace{10pt}\pause Implement function that sorts ....

\end{frame}

\begin{frame}[fragile]{old friends}

Have we seen this before?

\pause\vspace{20pt}
\begin{columns}
   \begin{column}{.5\linewidth}
     \begin{block}{sum/1}
       \begin{verbatim}
def sum([]) do  0 end
def sum([h|t]) do
    add(h,sum(t))
end
       \end{verbatim}
       \vfill
     \end{block}
   \end{column} 
\pause
   \begin{column}{.5\linewidth}
     \begin{block}{prod/1}
       \begin{verbatim}
def prod([]) do 1 end
def prod([h|t]) do
   mul(h,prod(t))
end
       \end{verbatim}
       \vfill
     \end{block}
   \end{column}
  \end{columns}

\vspace{20pt}{\em There is no built-in add/2, nor mul/2, but we can pretend that there is.}

\end{frame}


\begin{frame}[fragile]{good to have}

We would like to to this:

\pause\vspace{20pt}

\begin{columns}
   \begin{column}{.5\linewidth}
     \begin{block}{foldr/3}
       \begin{verbatim}
def foldr([], acc, op) do acc end
       \end{verbatim}
\pause
       \begin{verbatim}
def foldr([h|t], acc, op) do
   op.(h, foldr(t, acc, op))
end
      \end{verbatim}
       \vfill
     \end{block}
   \end{column}
\pause
   \begin{column}{.5\linewidth}
     \begin{block}{sum/1}
       \begin{verbatim}
def sum(l) do
  add = ... 
  foldr(l, ..., add)
end
       \end{verbatim}
     \end{block}
\pause     
   \begin{block}{prod/1}
       \begin{verbatim}
def prod(l) do
  mul = ... 
  foldr(l, ..., mul)
end
       \end{verbatim}
     \end{block}
   \end{column}
  \end{columns}

\pause\vspace{20pt}
{\em only problem, \ldots How do we express the function?}

\end{frame}

\begin{frame}[fragile]{lambda expressions}

We introduce a new data structure: a closure

  \vspace{20pt}

  \begin{tabular}{r l l}
   {\em Atoms} & = & \{a, b, c, \ldots\} \\
   {\em Closures} & = & \{<p:s:$\theta$> | p $\in $ Parameters $\wedge$ s $\in $ Sequences $\wedge$  $\theta \in $ Environments \}\\
   {\em Structures} & = & {\em Closures} $\cup$ {\em Atoms} $\cup$ \{ \{a, b\} \textbar a $\in$ {\em Structures}  $\wedge$  b $\in$ {\em Structures} \}
  \end{tabular}

\pause\vspace{10pt}
A {\em closure} is a function and an environment.

\pause\vspace{10pt}
We have not really defined what {\em Parameters}, a {\em Sequences} nor {\em Environments} are, but let's forget this for a while.

\end{frame}

\begin{frame}[fragile]{syntax for function expressions}

\begin{code}
   <function> &::= 'fn' '(' <parameters> ')' '->' <sequence> 'end'\\
   <parameters> &::= '  ' | <variables> \\
   <variables> &::= <variable> |  <variable> ',' <variables>\\
\end{code}
\pause
\begin{code}
   <application> &::= <expression> '.(' <arguments> ')'\\
   <arguments> &::= '  ' | <expressions> \\
   <expressions> &::= <expression> |  <expression> ',' <expressions>\\
\end{code}
\pause
\begin{code}
   <expression> &::= <function> | <application> | ...\\
\end{code}

\end{frame}

\begin{frame}[fragile]{function expressions}

\pause\vspace{10pt}
We will write:

\pause\vspace{10pt}\hspace{60pt}\verb!x = 2; f = fn (y) -> x + y end; f.(4) !


\vspace{20pt}\pause
Remember this?

\pause\vspace{10pt}\hspace{60pt} $ \lambda y \rightarrow x + y $

\end{frame}


\begin{frame}[fragile]{evaluation of a function expression}

$$\frac{ \theta = \{ v/s \mid  v/s \in \sigma \wedge v {\rm\ free\  in\ sequence}\}}{
E\sigma(\texttt{fn (} {\rm parameters}\;  \texttt{) ->}\; {\rm sequence}\; \texttt{end} ) \rightarrow \quad \langle{\rm parameters}:{\rm sequence}:\theta\rangle}$$

\vspace{20pt}\pause
\begin{verbatim}
   x = 2; f = fn (y) -> x + y end; f.(4)
\end{verbatim}

\vspace{20pt}\hspace{40pt}What is {\tt f}?

\end{frame}


\begin{frame}[fragile]{evaluation of a function application}

$$\frac{E\sigma(f) \rightarrow <v_1, \ldots:{\rm seq}:\theta > \qquad E\sigma(e_i) \rightarrow s_i \qquad E\{v_1/s_1, \ldots\}\cup\theta({\rm seq}) \rightarrow s}{
E\sigma(f.(e_1, \ldots)) \rightarrow s}$$ 

\vspace{20pt}\pause
\begin{verbatim}
   x = 2; f = fn (y) -> x + y end; f.(4)
\end{verbatim}

\vspace{20pt}\hspace{40pt}What is {\tt f.(4)}?

\end{frame}
 
\begin{frame}[fragile]{example}

\begin{verbatim}
def foo(x) do
  y = 3
  fn (v) -> v + y  + x 
end
\end{verbatim}
\pause\vspace{20pt}

\begin{verbatim}
   f = foo(2); x = 5;  y = 7; f.(1)
\end{verbatim}
\end{frame}

\begin{frame}[fragile]{case closed}

\pause
     \begin{block}{sum/1}
       \begin{verbatim}
def sum(l) do
  add = fn (x,y) -> x + y end
  foldr(l, 0, add)
end
       \end{verbatim}
     \end{block}
   \pause
     \begin{block}{prod/1}
       \begin{verbatim}
def prod(l) do
  mul = fun(x,a) -> x * a end
  foldr(l, 1, mul)
end
       \end{verbatim}
\vfill
     \end{block}

\end{frame}

\begin{frame}[fragile]{example}

\pause What is gurka/1 doing?

\pause\vspace{20pt}

\begin{verbatim}
def gurka(l) do
  f = fn (_, a) -> a + 1 end
  foldr(l, 0, f)
end
\end{verbatim}

\pause\vspace{20pt}
\pause How about tomat/1?

\pause\vspace{10pt}
\begin{verbatim}
def tomat(l) do 
  f = fn (h, a) ->  a ++ [h] end
  foldr(l, [], f)
end
\end{verbatim}
\end{frame}

\begin{frame}[fragile]{example}
     \begin{block}{foldr/3}
       \begin{verbatim}
def foldr([], acc, op) do acc end
def foldr([h|t], acc, op) ->
   op.(h, foldr(t, acc, op))
end
       \end{verbatim}
     \end{block}
\pause
     \begin{block}{foldl/3}
       \begin{verbatim}
def foldl([], acc, op) do acc end
def foldl([h|t], acc, op) do
  foldl(t, op.(h, acc), op)
end
       \end{verbatim}
     \end{block}
\end{frame}


\begin{frame}[fragile]{example}

\pause\vspace{10pt}
\begin{verbatim}
def gurka(l) do
  f = fn (_, a) -> a + 1 end
  foldl(l, 0, f)
end
\end{verbatim}
\pause 
\begin{verbatim}
def morot(l) do
  f = fn (h, a) ->  [h|a] end
  foldl(l, [], f)
end
\end{verbatim}
\end{frame}

\begin{frame}{left or right}

Which one should you use, {\em fold-left} or {\em fold-right}?

\end{frame}

\begin{frame}[fragile]{append all}

\pause Append all lists in a lists.

\vspace{20pt}

\begin{verbatim}
def append_right(l) do
  f = fun(e,a) -> e ++ a end
  foldr(l, [], f)
end
\end{verbatim}
\pause
\begin{verbatim}
def append_left(l) do
  f = fn (e,a) -> a ++ e end
  foldl(l, [], f)
end
\end{verbatim}

\end{frame}



\begin{frame}{patterns}

In the {\tt List} module. 

\begin{itemize}
\item {\tt foldr(list, acc, f)}: fold from right  {\tt f.(x1, .. f.(xn, acc)..) }
\item {\tt foldl(list, acc, f)}: fold from left  {\tt f.(xn, .. f.(x1, acc)..) }
\end{itemize}

\vspace{10pt}\pause
In the {\tt Enum} module. 
\begin{itemize}
\item {\tt map(enum, f)}: return the list of {\tt f.(x)} for each element {\tt x} in the enumeration
\pause 
\item {\tt filter(enum, f)}: return a list of all elements {\tt x}, for which {\tt f.(x)} evaluates to true
\pause
\item {\tt split_with(enum, f)}: partition the enumeration based on the result of  {\tt f.(x)}
\pause
\item {\tt sort(enum, f)}: sort the list given that the function {\t f} is less than or equal
\end{itemize}

\end{frame}


\begin{frame}[fragile]{an infinite list}

\pause\vspace{20pt}

\verb+inf = infinity()+\pause \verb+; [0|inf] = inf.()+\pause \verb+; [1|inf] = inf.()+

\pause
\begin{verbatim}
def infinity() do 
  fn() -> infinity(0) end 
end
\end{verbatim}
\pause
\begin{verbatim}
def infinity(n) -> 
  [...|...] 
end
\end{verbatim}

\end{frame}


\begin{frame}[fragile]{the list of fibonacci }

A function that returns an infinite list of Fibonacci numbers.

\pause\vspace{20pt}

\begin{verbatim}
def fib() do 
  fn() -> fib(1,1) end 
end
\end{verbatim}
\pause
\begin{verbatim}
def fib(f1, f2) do
  [f1 | fn() -> fib(f2, f1+f2) end]
end
\end{verbatim}

\end{frame}

\begin{frame}{a lazy list from 1 to 10}

  Let's represent a {\em range} of integers from 1 to 10 as:

  \vspace{10pt}\hspace{20pt} {\tt \{:range, 1, 10\}}

  \vspace{10pt}\pause

  Elixir gives us a syntax for this:

    \vspace{10pt}\hspace{20pt} {\tt 1..10}
  

    
  \vspace{40pt}\pause
  {\em This is not exactly how Elixir represents it but it's fine for now}

\end{frame}



\begin{frame}[fragile]{let's reduce this range}

\begin{verbatim}
  def sum(r) do
    reduce(r,  0, fn(x,a) -> x+a end)
  end
\end{verbatim}

  \vspace{10pt}\pause{\em Our {\tt reduce/3} should work as {\tt foldl/3} (does it make sense to have a {\tt foldr/3}).}

  
\begin{verbatim}
  def reduce({:range, from , to}, acc, fun) do
    if from <= to do
      reduce({:range, from+1, to}, fun.(from, acc), fun)
    else
      acc
    end
  end
\end{verbatim}

\vspace{20pt}  \pause
  
  

  
\end{frame}

\begin{frame}{Things we would like to do}

  Can we, without traversing the full range:
  \begin{itemize}
  \item take the first five elements
  \item find an element in the range
  \item return the first element and the rest of the range
  \end{itemize}

  \vspace{20pt}\pause
  {\em We need to control the reduction.}
\end{frame}



\begin{frame}[fragile]{continue}

  
\begin{verbatim}
  def reduce({:range, from , to}, {:cont, acc}, fun) do
    if from <= to do
      reduce({:range, from+1, to}, fun.(from, acc), fun)
    else
      {:done, acc}
    end
  end
\end{verbatim}

  \pause
  
\begin{verbatim}
  def sum(r) do
    reduce(r, {:cont, 0}, fn(x,a) -> {:cont, x+a} end)
  end
\end{verbatim}

  
\end{frame}

\begin{frame}[fragile]{stop in the midle}

\begin{verbatim}
    :
  def reduce(_, {:halt, acc}, _fun) do
    {:halted, acc}
  end
\end{verbatim}

\pause
\begin{verbatim}
  def take(r, n) do
    reduce(r, {:cont, {:sofar, 0, [] }},
      fn(x,{:sofar, s, a}) ->
        s = s+1
        if s >= n do
          {:halt, [x|a]}
        else
          {:cont, {:sofar, s, [x|a]}}
        end
      end)
  end    
\end{verbatim}
  
  
\end{frame}

\begin{frame}[fragile]{ head and tail }

\begin{verbatim}
    :
  def reduce(range, {:suspend, acc}, fun) do
    {:suspended, acc, fn(cmd) -> reduce(range, cmd, fun) end}
  end
\end{verbatim}
  \pause


\begin{verbatim}
  def head(r) do
    reduce(r, {:cont, :na},
      fn (x, _) ->
        {:suspend, x}
      end)
  end
\end{verbatim}
  
\end{frame}

\begin{frame}{Elixir libraries}

  \begin{itemize}
  \item {\tt List} : operatates on lists, returns a list or some value.  \pause
  \item {\tt Enum} : takes an {\tt Enumerable} as argument, returns a list or value. \pause
  \item {\tt Stream} : takes an {\tt Enumerable} as argument, returns an {\tt Enumerable} or value. 
  \end{itemize}

  \vspace{20pt}\pause {\em A datastructure is {\em Enumerable} if it
    implements the {\em reduce protocol}. Lists and ranges are {\tt
      Enumerable}.}
  
\end{frame}


\begin{frame}{Higher order}

Order of what?

\pause\vspace{20pt}
A first order function takes a value, a data structures, as argument and returns a value.

\pause\vspace{20pt}
A second order function takes a first order function as argument or returns a first order function.

\pause\vspace{20pt}
A third order function ....

\pause\vspace{20pt}
Higher order functions takes a higher order ...

\pause\vspace{20pt}
Are functions considered to be ``first-class citizen''?
\end{frame}

\begin{frame}[fragile]{not}
    
  \vspace{20pt}Not really - look at this.
    
\begin{verbatim}
 f = fn(x) -> x + 1 end
 g = fn(y) -> y + 1 end
\end{verbatim}
\vspace{20pt}\pause
\begin{verbatim}
\end{verbatim}
  f == g 
\end{frame}


\begin{frame}{Summary}

\pause Higher order programming:

\begin{itemize}
\pause\item {closure}: a function and an environment
\pause\item {generic algorithms}: separate the recursive pattern from the data it operates over
\pause\item {continuations}: powerful technique to handle incomplete information
\pause\item {range}: representation of a range of integers
\pause\item {streams}: lazy evaluation of ranges resulting in 
\end{itemize}


\end{frame}


\end{document}



