
\usepackage{fancyhdr}
\usepackage{lastpage}
\usepackage[utf8]{inputenc}

% Minted for syntax highliting
\usepackage{minted}
\usemintedstyle{tango}

% 
\usepackage[T1]{fontenc}
\usepackage{lmodern}

\usepackage{calc}
\usepackage{bytefield}

\usepackage{listings}
\usepackage{amsmath}

\usepackage{tikz}
\usetikzlibrary{automata,arrows,topaths,calc,positioning}
 
\usepackage{syntax}
\grammarindent=2cm


% Headers/footers styling
\pagestyle{fancy}
\fancyhf{}
\renewcommand{\headrulewidth}{0pt}

% Footer
\lfoot{ID1019}
\cfoot{KTH}
\rfoot{\thepage \hspace{1pt} / \pageref{LastPage}}

%\newcommand{\defaultpagestyle}{\thispagestyle{plain}}
\newcommand{\defaultpagestyle}{\thispagestyle{fancy}}



\title[ID1019 Asynchronous]{Asynchronous vs Synchronous}


\author{Johan Montelius}
\institute{KTH}
\date{\semester}

\begin{document}

\begin{frame}
\titlepage
\end{frame}

\begin{frame}{let's implement a memory}

  The {\em memory} should be a zero indexed {\em mutable data
    structure} of a given size and provide the following functions:

\pause\vspace{20pt}

\begin{itemize}
 \item {\tt @spec new([any]) :: memory} : creates a memory initialized with values from a list \pause
 \item {\tt @spec read(memory, integer) :: any } : returns the value stored in the memory at position $n$ 
 \item {\tt @spec write(memory, n, any) -> :ok} : writes the value at position $n$ in the memory 
\end{itemize}

\pause\vspace{20pt}
{\em what do we mean by read and write}

\end{frame}

\begin{frame}[fragile]{a first try}

\begin{verbatim}
defmodule Mem do

  def new(list) do
    {:mem, List.to_tuple(list)}
  end
\end{verbatim} \pause
\begin{verbatim}
  def read({:mem, mem}, n) do
     elem(mem, n)
  end
\end{verbatim} \pause
\begin{verbatim}
  def write({:mem, mem}, n, val) do
      ???
  end 
\end{verbatim}

\pause\vspace{20pt}
{\em ... hmm, not that easy}

\end{frame}

\begin{frame}[fragile]{the functional way}

Functional programming rules!

\pause\vspace{10pt}
Let {\tt write/3} return a new memory, a copy of the original with the update.

\pause\vspace{10pt}
\begin{verbatim}
def write({:mem, mem}, n, val) ->
   {:mem, put_elem(mem, n, val}}
end
\end{verbatim}
\pause\vspace{40pt}

{\em I love functional programming}

\end{frame}

\begin{frame}[fragile]{life is easy}
\pause
\begin{verbatim}
def test() do
  mem1 = Mem.new([:a, :b, :c, :d, :e, :f])
  mem2 = Mem.write(mem, 3, :foo)
  take_a_look_at_this(mem1)
  and_check_this(mem2)
end
\end{verbatim}

\pause
What if we always write like this:

\begin{verbatim}
def test() do
  mem = Mem.new([:a, :b, :c, :d, :e, :f])
  mem = Mem.write(mem, 3, :foo)
  take_a_look_at_this(mem)
  and_check_this(mem)
end
\end{verbatim}


\end{frame}

\begin{frame}{why not cheat}

Can we cheat, and introduce a mutable data structure?

\pause\vspace{20pt}
Can we use processes to implement mutable data structures?

\end{frame}


\begin{frame}[fragile]{a mutable cell}

\begin{block}{cell/1}
 \begin{verbatim}
def cell(v) do
  receive do
    {:read, pid} ->
       send(pid, {:value, v})
       cell(v)
    {:write, w, pid} ->
       send(pid, :ok)
       cell(w)
    :quit ->
       ok
  end
end
 \end{verbatim}
\end{block}
\end{frame}

\begin{frame}[fragile]{a synchronous interface}

\begin{columns}
\begin{column}{.5\linewidth}
\begin{block}{read/1}
 \begin{verbatim}
def read({:cell, cell}) do
  send(cell , {:read, self()})
  receive do
    {:value, v} ->
       v
  end
end
 \end{verbatim}
\end{block}
\end{column}
\pause
\begin{column}{.5\linewidth}
\begin{block}{write/1}
 \begin{verbatim}
def write({:cell, cell}, val) do
  send(cell, {:write, val, self()})
  receive do
    :ok ->
       :ok
  end
end
 \end{verbatim}
\end{block}
\end{column}
\end{columns}

\end{frame}

\begin{frame}[fragile]{the cell module}
\begin{verbatim}
defmodule Cell do

  def start(val) do       ## things to do in mother process
    {:cell, spawn_link(fn() -> init(val) end)}
  end

  def read({:cell, cell}) do  ... end

  def write({:cell, cell}, V) do ...  end

  def quit({:cell, cell}) do ...  end

  def init(val) do        ## things to do in the child process
     cell(val)
  end
\end{verbatim}
\end{frame}

\begin{frame}[fragile]{the memory}

\begin{verbatim}
defmodule Memory do

def new(list) do
  cells = Enum.map(list, fn(v) -> Cell.start(v) end)
  {:mem, List.to_tuple(cells)}
end
\end{verbatim} \pause
\begin{verbatim}
def read({:mem, mem}, n) do
  Cell.read(elem(mem, n))
end
\end{verbatim} \pause
\begin{verbatim}
def write({:mem, mem}, n, val) do
  Cell.write(elem(mem, n), val)
end
\end{verbatim} \pause
\begin{verbatim}
def delete({:mem, mem}) do
  Enum.each(Tuple.to_list(mem), fn(c) -> Cell.quit(c) end)
end
\end{verbatim}
\end{frame}

\begin{frame}[fragile]{at last}

\pause
\begin{verbatim}
 test() -> 
     array = Memory.new([:a,:b,:c,:d,:e,:f,:g,:h])
     Memory.write(array, 5, :foo)
     Memory.write(array, 2, :bar)
     love_all(array)
     sort_it_for_me(array)
     i_love_imperative_programming(array)
  end
\end{verbatim}
\end{frame}

\begin{frame}{functional vs processes}

By extending our language to handle processes, we have left the
functional world. 

\pause\vspace{10pt}
We can implement {\em mutable data structures}, something
that we agreed was evil.

\pause\vspace{20pt}

Why are mutable data structures evil?

\end{frame}

\begin{frame}[fragile]{the evil of mutability}

\pause
\begin{verbatim}
    :
  this = check_this(mem),
  %% I hope it did not change anything
  that = check_that(mem),
    :
\end{verbatim}

\end{frame}

\begin{frame}[fragile]{truly bad}

\begin{columns}
 \begin{column}{0.5\linewidth}
  \begin{block}{all\_work/2}
   \begin{verbatim}
def all_work(cell, 0, jack) do 
  send(jack, :run)
end
def all_work(cell, n, jack) do
  x = Cell.read(cell)
  Cell.write(cell, x + 1)
  all_work(cell, n-1, jack)
end
   \end{verbatim}
  \end{block}
 \end{column}
 \begin{column}{0.5\linewidth}
  \begin{block}{no\_play/1}
\begin{verbatim}
 def no_play(n) do
    cell = Cell.start(0)
    me = self()
    spawn(fn() -> 
       all_work(cell, n, me) end)
    spawn(fn() -> 
       all_work(cell, n, me) end)    
    receive do :run ->
         receive do :run ->
              Cell.read(cell)
         end
    end
 end
\end{verbatim}
  \end{block}
 \end{column}
\end{columns}


\end{frame}

\begin{frame}{a dull boy}

\begin{columns}
 \begin{column}{0.5\linewidth}
   \includegraphics[width=\linewidth]{shining.jpg}
 \end{column}
 \pause
 \begin{column}{0.5\linewidth}
   \begin {itemize}
    \item without mutable data structures, concurrency would be easy
    \item sharing mutable data structures is the root of all evil
    \item a process is, in one way, a mutable data structure
    \pause
    \item ... It's only a movie.
   \end{itemize}
 \end{column}
 \end{columns}

\end{frame}

\begin{frame}[fragile]{adding atomic operations}

\begin{columns}
 \begin{column}{0.5\linewidth}
\begin{verbatim}
def cell(v) do
  receive do
    {:read, pid} ->
       send(pid, {:value, v})
       cell(v)
    {:write, w, pid} ->
       send(pid, :ok)
       cell(w)
    :quit ->
       ok
  end
end
\end{verbatim}
 \end{column}
\pause
 \begin{column}{0.5\linewidth}
\begin{verbatim}
def cell(v) do
  receive do
    {:read, pid} ->
       send(pid, {:value, v})
       cell(v)
    {:write, w, pid} ->
       send(pid, :ok)
       cell(w)
    {:inc, n} ->
       send(pid, :ok)
       cell(v+n)       
    :quit ->
       ok
  end
end
\end{verbatim}
 \end{column}
\end{columns}

\end{frame}

\begin{frame}[fragile]{a lock}

\vspace{10pt}

We want to avoid processes interfering with each other when intracting
with a process.

\vspace{10pt}
Let's implement a lock using our cell.

\pause
\begin{itemize}
 \item take the lock
 \item relase the lock
 \item at most one process can hold the lock
\end{itemize}

\end{frame}


\begin{frame}[fragile]{using a cell}

\pause\vspace{20pt}

\begin{verbatim}
def critical(danger, lock) do
  case Cell.read(lock) of
    :locked ->
        critical(danger, lock)
    :free ->
        Cell.write(lock, :locked)
        do_it(danger)
        Cell.write(lock, :free)
  end
end 
\end{verbatim}

\pause\vspace{20pt}
{\em hmmm, not so good}

\end{frame}

\begin{frame}[fragile]{a better lock}

\begin{verbatim}
defmodule Lock do

def start() do
   {:lock, spawn_link(fun() -> free() end)}
end
\end{verbatim}
\begin{columns}
  \begin{column}{0.5\linewidth}
\begin{verbatim}
def free() do
  receive do
    {:take, pid} ->
       send(pid, :taken)
       taken()
  end
end
\end{verbatim}
  \end{column}
  \begin{column}{0.5\linewidth}
\begin{verbatim}  
def taken() do
  receive do
   :release ->
     ...
  end
end

\end{verbatim}
  \end{column}
\end{columns}
\end{frame}


\begin{frame}[fragile]{asynchronous communication}

Messages in Elixir/Erlang is a form of {\em asynchronous communication}.

\pause\vspace{20pt}
\begin{verbatim}
  send(pid, {:take, self()})
\end{verbatim}

\pause\vspace{20pt}
The sender does not block and wait for the receiver to accept the message. 
\pause\vspace{20pt}
{\em Asynchronous : not at the same time}
\end{frame}

\begin{frame}[fragile]{synchronous communication}

We can implement {\em synchronous communication}
\pause\vspace{5pt}

\begin{columns}
 \begin{column}{0.5\linewidth}
  \begin{verbatim}
def take(lock) do
  send(lock, {:take, self()})
  receive do
    :taken ->
       :ok
  end 
end
  \end{verbatim}
  \pause
 \end{column}
 \begin{column}{0.5\linewidth}
  \begin{verbatim}
def free() do
  receive do
    {:take, pid} ->
       self(pid, :taken)
       taken()
  end
end
  \end{verbatim}
 \end{column}
\end{columns}
\pause\vspace{10pt}
        
The reply is generated by the application layer.
\pause\vspace{5pt}

\begin{verbatim}
  :ok = take(lock)
\end{verbatim}

The application process sees a {\em synchronous operations},
\pause\vspace{5pt}

{\em Synchronous : at the same time}
\end{frame}

\begin{frame}{pros and cons}

What are the pros and cons of asynchronous communication?

\end{frame}

\begin{frame}[fragile]{synchronous by default}

We could provide synchronous communication by default, for example:
\pause\vspace{10pt}
\begin{verbatim}
send(jack, {:take, this})
%% I now know that the message is in the queue of Jack
  :
  ?
\end{verbatim}

\pause 
\vspace{10pt}... or\vspace{10pt}
\begin{verbatim}
send(jack, {:do, that})
%% I now know that the message has been "received" by Jack
  :
  ?
\end{verbatim}
\pause 
\vspace{10pt}but why not..\vspace{10pt}
\end{frame}

\begin{frame}[fragile]{synchronous when needed}
\begin{verbatim}
send(jack,  {:hello, self()}
%% I need to know that Jack has fun
receive 
  :having_fun ->
     run_as_hell()
end
\end{verbatim}
\pause
\includegraphics[height=0.3\textheight]{fun.jpg}
\end{frame}

\begin{frame}[fragile]{an asynchrounous memory}

Synchronous programing is boring.

\begin{verbatim}
def cell(v) do
  receive do
    {:read, ref, pid} ->
       send(pid, {:value, ref, v})
       cell(v)
    {:write, w, ref, pid} ->
       send(pid, {:ok, ref})
       cell(w)
    :quit ->
       :ok
  end
end
\end{verbatim}

  
\end{frame}

\begin{frame}[fragile]{an asynchrounous memory}

\begin{columns}
 \begin{column}{0.6\linewidth}  
\begin{verbatim}
def redrum({:cell, cell}) do
  ref = make_ref()
  send(cell, {:read, ref, self()})
  ref
end
\end{verbatim} \pause

\begin{verbatim}  
def murder(ref) do
  receive do
   {:value, ^ref, value} -> 
      value
  end
end
\end{verbatim}
 \end{column}
 \pause
 \begin{column}{0.4\linewidth}
\begin{verbatim}  
   :
 ref = redrum(cell) 
   :
   :
 val = murder(ref)
   :
\end{verbatim}

 \end{column}
\end{columns}
\end{frame}


\begin{frame}{Summary}

\begin{itemize}
\item Processes can be used to implement mutable data structures.
\item Same problems needs to be considered. 
\item Made easier since each mutable data structure is a user defined process.
\item Asynchronous vs synchronous message passing - pros and cons.
\end{itemize}

\end{frame}
\end{document}
