
\usepackage{fancyhdr}
\usepackage{lastpage}
\usepackage[utf8]{inputenc}

% Minted for syntax highliting
\usepackage{minted}
\usemintedstyle{tango}

% Headers/footers styling
\pagestyle{fancy}
\fancyhf{}
\renewcommand{\headrulewidth}{0pt}

% Footer
\lfoot{ID1019}
\cfoot{KTH}
\rfoot{\thepage \hspace{1pt} / \pageref{LastPage}}

%\newcommand{\defaultpagestyle}{\thispagestyle{plain}}
\newcommand{\defaultpagestyle}{\thispagestyle{fancy}}


\title[ID1019 Asynchronous]{Asynchronous vs Synchronous}


\author{Johan Montelius}
\institute{KTH}
\date{\semester}

\begin{document}

\begin{frame}
\titlepage
\end{frame}

\begin{frame}{let's implement a memory}

The {\em memory} should be a {\em mutable data structure} of a given size and provide the following functions:

\pause\vspace{20pt}

\begin{itemize}
 \item {\tt new(List) -> Memory} : creates a memory initialized with values from a list
 \item {\tt read(Memory, N) -> \{ok, Value\}} : returns a tuple with the value of memory position N
 \item {\tt write(Memory, N, Value) -> ok} : writes the value at position N in the memory
\end{itemize}

\pause\vspace{20pt}
{\em what do we mean by read and write}

\end{frame}

\begin{frame}[fragile]{a first try}

\begin{verbatim}
-module(mem).

new(List) -> 
  {mem, erlang:list_to_tuple(List)}.

read({mem, M}, N) ->
  {ok, erlang:element(N, M)}.

write({mem, M}, N, V) ->
  ???.
\end{verbatim}

\pause\vspace{20pt}
{\em ... hmm, not that easy}

\end{frame}

\begin{frame}[fragile]{the functional way}

Functional programming rules!

\pause\vspace{10pt}
Let {\tt write/3} return a new memory, a copy of the original with the update.

\pause\vspace{10pt}
\begin{verbatim}
write({mem, M}, N, V) ->
   {ok, {mem, erlang:setelement(N, M, V}}}.
\end{verbatim}
\pause\vspace{10pt}

{\em I love functional programming}

\end{frame}

\begin{frame}[fragile]{life is easy}
\pause
\begin{verbatim}
test() ->
  M0 = mem:new([a,b,c,d,e,f]),
  {ok, M1} = mem:write(M0, 3, foo),
  take_a_look_at_this(M1),
  and_check_this(M0).
\end{verbatim}
\pause
\begin{verbatim}
test() ->
  M0 = mem:new([a,b,c,d,e,f]),
  {ok, M1} = mem:write(M0, 3, foo),
  {ok, M2} = mem:write(M1, 5, bar),
  {ok, V2} = mem:read(M2, 2),
  {ok, V5} = mem:read(M2, 5),
  {it_is_so_easy, V2, V5}.
\end{verbatim}
\pause
{\em try to edit this code}

\end{frame}

\begin{frame}{why not cheat}

Can we cheat, and introduce a mutable data structure?

\pause\vspace{20pt}
Can we introduce syntactical support to help us with the argument passing?

\pause\vspace{20pt}
Can we use processes to implement mutable data structures?

\end{frame}


\begin{frame}[fragile]{a mutable cell}

\begin{block}{cell/1}
 \begin{verbatim}
cell(V) ->
  receive
    {read, Pid} ->
       Pid ! {value, V},
       cell(V);
    {write, W, Pid} ->
       Pid ! ok,
       cell(W)
  end.
 \end{verbatim}
\end{block}
\end{frame}

\begin{frame}[fragile]{a mutable cell}

\begin{columns}
\begin{column}{.5\linewidth}
\begin{block}{read/1}
 \begin{verbatim}
read({cell, M}) ->
  M ! {read, self()},
  receive
    {value, V} ->
       {ok, V}
  end.
 \end{verbatim}
\end{block}
\end{column}
\pause
\begin{column}{.5\linewidth}
\begin{block}{write/1}
 \begin{verbatim}
write({cell, M}, W) ->
  M ! {write, W, self()},
  receive
    ok ->
       ok
  end.
 \end{verbatim}
\end{block}
\end{column}
\end{columns}

\end{frame}

\begin{frame}[fragile]{the cell module}
\begin{verbatim}
-module(cell).
-export([start/1, read/1, write/2]).

start(V) ->
   %% things to do in mother process
   {cell, spawn_link(fun() -> init(V) end)}.

read({cell, C}) -> ...

write({cell, C}, V) -> ...

init(V) -> % things to do in the child process
   cell(V).     

cell(V) ->
   :
\end{verbatim}
\end{frame}

\begin{frame}[fragile]{the memory}

\begin{verbatim}
-module(mem).

new(List) -> 
  Cells = lists:map(fun(E) -> cell:start(E) end, List),
  {mem, erlang:list_to_tuple(Cells)}.

read({mem, M}, N) ->
  Cell = erlang:element(N, M),
  cell:read(Cell).

write({mem, M}, N, V) ->
  Cell = erlang:element(N, M),
  cell:write(Cell, V).
\end{verbatim}
\end{frame}

\begin{frame}[fragile]{at last}

\pause
\begin{verbatim}
 test() -> 
     Array = new:mem([a,a,a,a,a,a,a,a]),
     mem:write(Array, 5, foo),
     mem:write(Array, 2, bar),
     love_all(Array),
     sort_it_for_me(Array)
     i_love_imperative_programming(Array).
\end{verbatim}
\end{frame}

\begin{frame}{functional vs processes}

By extending our language to handle processes, we have left the
functional world. 

\pause\vspace{10pt}
We can implement {\em mutable data structures}, something
that we agreed was evil.

\pause\vspace{20pt}

Why are mutable data structures evil?

\end{frame}

\begin{frame}[fragile]{the evil of mutability}

\pause
\begin{verbatim}
    :
  This = check_this(Mem),
  %% I hope it did not change anything
  That = check_that(Mem),
    :
\end{verbatim}

\end{frame}

\begin{frame}[fragile]{truly bad}

\begin{block}{spawn a process}
\begin{verbatim}
 run() ->
    Mem = mem:new([r,e,d,r,u,m]),
    spawn(fun() -> all_work(Mem) end)
    no_play(Mem).
\end{verbatim}
\end{block}

\pause
\begin{columns}
 \begin{column}{0.5\linewidth}
  \begin{block}{no\_play/1}
   \begin{verbatim}
   :
{ok, X} = mem:read(Mem, 2),
mem:write(Mem, 3, foo),
   :
   \end{verbatim}
  \end{block}
 \end{column}
 \begin{column}{0.5\linewidth}
  \begin{block}{all\_work/1}
   \begin{verbatim}
   :
mem:write(Mem, 3, bar),
{ok, Y} = mem:read(Mem, 2),
   :
   \end{verbatim}
  \end{block}
 \end{column}
\end{columns}


\end{frame}

\begin{frame}{a dull boy}

\begin{columns}
 \begin{column}{0.5\linewidth}
   \includegraphics[width=\linewidth]{shining.jpg}
 \end{column}
 \pause
 \begin{column}{0.5\linewidth}
   \begin {itemize}
    \item without mutable data structures, concurrency would be easy
    \item sharing mutable data structures is the root of all evil
    \item a process i, in one way, a mutable data structure
    \pause
    \item ... but not really
   \end{itemize}
 \end{column}
 \end{columns}

\end{frame}

\begin{frame}[fragile]{the process decides}

\begin{columns}
 \begin{column}{0.5\linewidth}
\begin{verbatim}
cell(V) ->
  receive 
    {read, N, Pid} ->
       Pid ! {value, V},
       cell(V);
    {write, N, W, Pid} ->
       Pid ! ok
       cell(W)
  end.
\end{verbatim}
 \end{column}
\pause
 \begin{column}{0.5\linewidth}

 The cell process is in
 charge; it decides how the state shuld be modified.

 Does not solve the problem of interfering
 processes.

\end{column} \end{columns}

\end{frame}

\begin{frame}[fragile]{a lock}

\vspace{10pt}

We want to avoid processes interfering with each other when intracting
with a process.

\vspace{10pt}
Let's implement a lock using our cell.

\pause
\begin{itemize}
 \item take the lock
 \item relase the lock
 \item at most one process can hold the lock
\end{itemize}

\end{frame}


\begin{frame}[fragile]{using a cell}

\pause\vspace{20pt}

\begin{verbatim}
critical(Danger, Lock) -> 
  case cell:read(Lock) of
    locked ->
        critical(Danger, Lock);
    free ->
        cell:write(Lock, locked),
        do_it(Danger),
        cell:write(Lock, free)
  end.    
\end{verbatim}

\pause\vspace{20pt}
{\em hmmm, not so good}

\end{frame}

\begin{frame}[fragile]{a better lock}

\begin{verbatim}
-module(lock).
-export([start/1, take/1, release/1]).

start() ->
   {cell, spawn_link(fun() -> free() end)}.

free() ->
  receive 
    {take, Pid} ->
       Pid ! taken,
       taken()
  end.

taken() ->
    :
  end.
\end{verbatim}

\end{frame}

\begin{frame}[fragile]{a better idea}

Why not let the critical operation be a process?
\pause\vspace{20pt}
\begin{verbatim}
-module(danger).
-export([start/1, do_it/1]).
 
:

critical(State) ->
  recceive 
    {do_it, Pid} -> 
        Updated = danger(State),
        Pid ! ok,
        critical(Updated)
  end.
\end{verbatim}

\end{frame}

\begin{frame}{similar to Java}

Is this not very similar to how things are done in Java?

\pause\vspace{20pt}
\begin{itemize}
\item in Java an object is passive
\pause
\item a Java thread, starts executing ``in an object''
\pause
\item the tread calls methods of other objects
\pause
\item a shared object needs to be protected (using {\tt synchronized})
\pause
\item typically a handful of threads, thousands of objects
\end{itemize}

\end{frame}

\begin{frame}[fragile]{asynchronous communication}

Messages in Erlang is a form of {\em asynchronous communication}.

\pause\vspace{20pt}
\begin{verbatim}
  M ! {write, W, self()}
\end{verbatim}

\pause\vspace{20pt}
The sender does not block and wait for the receiver to accept the message. 
\pause\vspace{20pt}
{\em Asynchronous : not at the same time}
\end{frame}

\begin{frame}[fragile]{synchronous communication}

We can implement {\em synchronous communication}
\pause\vspace{20pt}

\begin{columns}
 \begin{column}{0.5\linewidth}
  \begin{verbatim}
write({cell, M}, W) ->
  M ! {write, W, self()}
  receive
    ok ->
       ok
  end.
  \end{verbatim}
  \pause
 \end{column}
 \begin{column}{0.5\linewidth}
  \begin{verbatim}
  receive 
     {write, W, Pid} ->
        Pid ! ok;
      :
  end
  \end{verbatim}
 \end{column}
\end{columns}
\pause\vspace{10pt}
        
The reply is generated by the application layer.
\pause\vspace{5pt}

\begin{verbatim}
  Reply = write(Cell, W),
\end{verbatim}

The application process sees a {\em synchronous operations},
\pause\vspace{5pt}

{\em Synchronous : at the same time}
\end{frame}

\begin{frame}{pros and cons}

What are the pros and cons of asynchronous communication?

\end{frame}

\begin{frame}[fragile]{synchronous by default}

We could provide synchronous communication by default, for example:
\pause\vspace{10pt}
\begin{verbatim}
Jack ! {take, This}
%% I now know that the message is in the queue of Jack
  :
\end{verbatim}

\pause 
\vspace{10pt}... or\vspace{10pt}
\begin{verbatim}
Jack ! {do, That}
%% I now know that the message has been "received" by Jack
  :
\end{verbatim}
\pause 
\vspace{10pt}but why not..\vspace{10pt}
\end{frame}

\begin{frame}[fragile]{synchronous when needed}
\begin{verbatim}
Jack ! {go, self()}
%% I need to know that Jack has fun
receive 
  having_fun ->
     run_as_hell()
end
\end{verbatim}
\pause
\includegraphics[height=0.3\textheight]{fun.jpg}
\end{frame}

\begin{frame}[fragile]{the outside world}

A functional program has problems communicating with the ``outside
world''; we evaluate an expression and return a result.

\pause\vspace{20pt}

Once we have processes and process communication, we can treat the
outside world as a process - problem solved.

\pause\vspace{20pt}

\begin{verbatim}
test(Outside) -> 
  Outside ! {format, ``Hello ~n'', []},
  do_this(Outside, foo, bar),
  do_that(Outside, zot).
\end{verbatim}

\end{frame}

\begin{frame}{Summary}

\begin{itemize}
\item Processes can be used to implement mutable data structures.
\item Same problems needs to be considered. 
\item Made easier since each mutable data structure is a user defined process.
\item Asynchronous vs Synchronous message passing - pros and cons.
\item How do we model the outside world?
\end{itemize}

\end{frame}
\end{document}
